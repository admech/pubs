%\section{Примеры движений}

{\bf 5. Примеры движений}
\stepcounter{section}
Численные решения получим для симметричного трехколесного экипажа ($\alpha_i = \frac{2\pi}{N}(i - 1), N = 3$), с $n = 5$ роликами на колесе и следующих движений:
\begin{enumerate}
  \item \label{sol:selfrot} вращение вокруг своей оси ($\nu_1(0) = \nu_2(0) = 0, \nu_3(0) = 1$) (фиг. ~\ref{fig:selfrot}),
  \item \label{sol:straight} движение по прямой в направлении оси первого колеса ($\nu_1(0) = 1, \nu_2(0) = \nu_3(0) = 0$) (фиг.~\ref{fig:straight})
  \item \label{sol:wrench} движение с ненулевой скоростью центра масс и, одновременно, с ненулевой угловой скоростью платформы ($\nu_1(0) = 1, \nu_2(0) = 0, \nu_3(0) = 1$) (фиг.~\ref{fig:wrench}).
\end{enumerate}


Расчеты выполнены при следующих значениях геометрических и массовых параметров: радиус колеса $r = 0.05$, масса колеса $ M_{\text{к}} = 0.15$, масса ролика $m_{\text{рол}} = 0.05$, радиус платформы $R = 0.15$ (?), масса платформы $M_{\text{пл}} = 1$, момент инерции $B=??$. Значения величин здесь и всюду даны в единицах СИ.Для безынерционной модели массово-инерционные характеристики колес положим соответствующими системе с 5 заблокированными роликами.


Во всех трех случаях наблюдаются отличия между двумя постановками: свободные ролики приходят в движение, из-за чего меняется угловая скорость платформы экипажа и скорость центра масс экипажа. Кроме этого, становится заметно влияние введенных предположений о смене контакта: график кинетической энергии приобретает ступенчатый вид в силу изменений в слагаемых (\ref{kin_en}), зависящих от $\chi$ и $\dot{\phi}_{i,j}$: 
\begin{equation}\label{sines_in_kin_en}
    B\sum_{i,j}(\dot{\phi}_{ij}^2 + 2\dot{\theta}\sin(\kappa_j + \chi_i)\dot{\phi}_{ij}),
\end{equation}
происходящих при мгновенном наложении связей. В промежутки времени между сменами роликов энергия остается постоянной. 

В случаях \ref{sol:selfrot} и \ref{sol:straight} траектории центра экипажа $S$ на плоскости $OXY$ и характер вращения вокруг вертикальной оси $SZ$ ($\theta(t)$) отличаются между моделью с роликами и безынерционной несущественно, однако заметны переходные процессы во вращении роликов в начале движения.

В случае вращения вокруг вертикали (движение \ref{sol:selfrot}) угловая скорость платформы $\nu_3$ меняется не монотонно, но в среднем медленно убывает: за первые 1000 секунд угловая скорость уменьшается на 2\%. Скорость центра масс остается равной нулю. Кинетическая энергия системы также медленно убывает. На фиг.~\ref{fig:selfrot} представлены угловые скорости роликов на первом колесе $\dot{\phi}_{1j}$ (номер кривой, указанной на рисунке, совпадает с номером ролика на колесе, поведение роликов на других двух колесах полностью аналогично). Заметим, что при нулевой скорости центра экипажа опорный ролик не вращается (\ref{constraint_wheel_contact}): угловая скорость первого ролика в течение первой секунды движения нулевая. После выхода из контакта ролик начинает раскручиваться в соответствии с первым интегралом (\ref{int_free_roller}). Раскрученный ролик при входе в контакт с опорной плоскостью мгновенно теряет угловую скорость -- на графике угловой скорости первого ролика это происходит при $t=9.6$ c -- что приводит к убыванию кинетической энергии.

При движении по прямой (движение \ref{sol:straight}) угловая скорость остается нулевой.
На фиг.~\ref{fig:straight} слева показаны графики относительного изменения скорости центра масс $\nu_1(t)/\nu_1(0) - 1$ (кривая 1) и кинетической энергии $T/T(0)-1$ (кривая 2). Видно, что на начальном этапе движения при смене контакта кинетическая энергия возрастает, что обусловлено принятой в данной работе моделью наложения связи, но при этом возрастание энергии остается в пределах 4\%. Скорость центра масс (кривая 2, слева) в среднем убывает. Скорость вращения переднего колеса равна нулю, колесо катится, опираясь на один и тот же ролик, остальные ролики не раскручиваются. Угловые скорости роликов на одном из задних колес показаны на фиг.~\ref{fig:straight} справа. Свободные ролики двигаются с постоянной угловой скоростью, ролик в контакте изменяет свою скорость за счет скорости центра масс. После того, как все ролики побывают в контакте, их движение становится квазипериодичным, а энергия убывает с каждой сменой контакта. 

При движении \ref{sol:wrench}, сочетающем поступательное и вращательное движение, угловая скорость экипажа $\nu_3$ растет и выходит на постоянное значение (кривая 1 на фиг.~\ref{fig:wrench} слева вверху), скорость центра экипажа $v = \sqrt{\nu_1^2+\nu_2^2}$ уменьшается до нуля (кривая 2 там же), а кинетическая энергия после короткого начального участка, где происходят маленькие по величине скачки вверх аналогично движению \ref{sol:straight}, убывает. Угловые скорости роликов представляют собой квазипериодические функции времени (характерный участок представлен на фиг.~\ref{fig:wrench} справа вверху, обозначения те же что и на фиг.~\ref{fig:selfrot}). Центр платформы описывает спираль, фиг.~\ref{fig:wrench} внизу. Заметим, что если не учитывать массу роликов на колесе, то при этих начальных условиях скорость центра масс и угловая скорость платформы сохраняется, а центр платформы описывает окружность. Таким образом, даже малая масса роликов приводит к качественным изменениям в движении экипажа.

% Траектории центра экипажа $S$ на плоскости $OXY$ и характер вращения вокруг вертикальной оси $SZ$ ($\theta(t)$) близки или совпадают между двумя постановками (с роликами и без) во всех трех случаях, однако поведение псевдоскоростей и кинетической энергии отличается.

% Движения \ref{sol:selfrot} одинаковы: экипаж вращается вокруг $SZ$ с постоянной скоростью, центр масс остается в покое. В случае \ref{sol:straight}, псевдоскорость $\nu_1$ -- проекция скорости центра масс на ось $\xi$, жестко связанную с платформой экипажа и совпадающую с направлением движения -- оказывается непостоянной в постановке с роликами ("скорость вращения" $\nu_3 = const$). Кроме этого, становится заметно влияние введенных предположений о смене контакта: график кинетической энергии приобретает ступенчатый вид в силу изменений в слагаемых (\ref{kin_en}), зависящих от $\chi$: 
% $$B\sum_{i,j}2\dot{\theta}\sin(\kappa_j + \chi_i)\dot{\phi}_{ij},$$
% происходящих при замене $\chi_i$ с $\chi^+$ на $\chi^-$ или наоборот. В промежутки времени между сменами роликов, впрочем, энергия остается постоянной. Моменты смены контакта становятся заметны и на графике $\nu_3$, но лишь в виде шума в численном решении. Движения \ref{sol:wrench} оказываются близки, но не совпадают, и вид графика кинетической энергии аналогичен случаю \ref{sol:straight}. Также, $\nu_3$ перестает быть постоянной.