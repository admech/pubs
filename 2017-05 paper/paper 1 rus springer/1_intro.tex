\section{Введение}

Омниколеса -- это колеса особой конструкции, позволяющей экипажу, оснащенному ими, двигаться в произвольном направлении, вращая колеса только вокруг их собственных осей, но не вокруг вертикали. На ободе омниколеса располагаются ролики, которые могут свободно вращаться вокруг своих осей, жестко закрепленных в диске колеса (поэтому такие колеса также называют роликонесущими), причем оси эти могут располагаться либо в плоскости колеса (и лежать на касательных к границе диска), либо под углом к ней (такая конструкция называется меканум \cite{mecanum}).

Задачи динамики систем тел, связянные с роликонесущими колесами и экипажами, использующими их, часто становятся объектами исследований, однако, как правило, либо рассматриваются упрощенные модели, пренебрегающие инерцией и формой роликов \cite{ZobovaTatarinov, Martynenko, Borisov, 5to9FromZobova2011}, либо используются формализмы для построения численных моделей систем тел, скрывающие явный вид уравнений движения \cite{KosenkoGerasimov, MaybeTobolar, Others, 3to4FromZobova2011}.

Цель настоящей работы -- получение уравнений движения по инерции экипажа с омни-колесами с массивными роликами в неголономной постановке с помощью подхода \cite{Tatarinov} в явном виде, исследовние их свойств и сравнение поведения такой системы с поведением системы без роликов \cite{Zobova2011}.

В статье показаны отличия в уравнениях движения, возникающие при добавлении роликов -- это явная зависимость правых частей от углов поворота колес (и, следовательно, необходимость добавить уравнения связей для получения замкнутой системы дифференциальных уравнений) и наличие слагаемых, пропорциональных собственному моменту инерции ролика. Также обсуждается смена контакта роликов и возникающие в связи с ней разрывы в правых частях уравнений движения, которые здесь устраняются путем введения дополнительных предположений о характере движения роликов. В заключение проводится сравнение основных типов движения симметричного трехколесного экипажа, полученных численным интегрированием уравнений движения, для рассматриваемых моделей омни-колес.

TODO: РАСКОММЕНТИРОВАТЬ ТУТ КУСОЧЕК

%, в частности, отмечено исчезновение ряда движений и линейного интеграла при добавлении роликов.


