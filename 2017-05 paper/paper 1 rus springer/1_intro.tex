\section{Введение}

% все рассматривают экипажи с колесами без роликов, налагая такие-то связи, и задачи хорошо исследуются (в таких-то работах)
Омни-колесо --- это ...
Задачи динамики систем тел, связянные с роликонесущими колесами и экипажами, использующими их, часто становятся объектами исследований, однако, как правило, либо рассматриваются упрощенные модели, пренебрегающие инерцией и формой роликов \cite{ZobovaTatarinov, Martynenko, Borisov}, либо используются формализмы для построения численных моделей систем тел, скрывающие явный вид уравнений движения \cite{KosenkoGerasimov, MaybeTobolar, Others}.

Цель настоящей работы -- получение уравнений движения по инерции экипажа с омни-колесами с массивными роликами в неголономной постановке с помощью подхода \cite{Tatarinov} в явном виде, исследовние их свойств и сравнение поведения такой системы с поведением системы без роликов \cite{Zobova2011}.

% массы роликов могут достигать такой-то доли от массы всего колеса (ссылки!), и потому имеет смысл рассмотреть полную постановку

% ряд движений исчезнет, изчезнет первый интеграл и вообще мало ли что - про это не буду писать, т.к. движения не исчезли, и мы вообще-то строго не проверяли, что тут с интегралами.

% покажем, что уравнения отличаются наличием слагаемых порядка собственного момента инерции ролика
% возникнут сложности с переходом между роликами, преодолеем их, введя некоторые упрощения.
В статье показаны отличия в уравнениях движения, возникающие при добавлении роликов -- это явная зависимость правых частей от углов поворота колес (и, следовательно, необходимость добавить уравнения связей для получения замкнутой системы дифференциальных уравнений) и наличие слагаемых, пропорциональных собственному моменту инерции ролика. Также обсуждается смена контакта роликов и возникающие в связи с ней разрывы в правых частях уравнений движения, которые здесь устраняются путем введения дополнительных предположений о характере движения роликов. В заключение проводится сравнение основных типов движения симметричного трехколесного экипажа, полученных численным интегрированием уравнений движения, для рассматриваемых моделей омни-колес.


