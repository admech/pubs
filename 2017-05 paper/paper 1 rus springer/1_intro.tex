%\section{Введение}
{\bf 1. Введение.}
\stepcounter{section}
Омниколеса (в русской литературе также используется название роликонесущие колеса) -- это колеса особой конструкции, позволяющей экипажу двигаться в произвольном направлении, вращая колеса вокруг их собственных осей и не поворачивая их вокруг вертикали. На ободе такого колеса располагаются ролики, которые могут свободно вращаться вокруг своих осей, жестко закрепленных в диске колеса. Существуют два варианта расположения осей роликов: первый  (собственно омниколеса) -- оси роликов являются касательными к ободу колеса и, следовательно, лежат в его плоскости; второй (меканум-колеса \cite{mecanum}) -- оси роликов развернуты вокруг нормали к ободу колеса на постоянный угол, обычно $\pi/4$.

Ранее была рассмотрена динамика омни-экипажей с упрощенными моделями омни-колес, в которых не учитывается инерция и форма роликов \cite{ZobovaTatarinovPMM, formalskii, borisov, ZobovaTatarinovAspecty2006, zobova2008svobodnye8020851, Martynenko2010}. В этих работах колеса моделируются как жесткие диски (без роликов), которые могут скользить в одном направлении и катиться без проскальзывания в другом. Далее мы будем называть такую модель безынерционной в том смысле, что инерция собственного вращения роликов в ней не учитывается. Другая часть работ по динамике омни-экипажа \cite{KosenkoGerasimov, Tobolar, Williams2002, Ashmore2002} использует некоторые формализмы для построения численных моделей систем тел, при этом явный вид уравнений движения оказывается скрыт. Это делает невозможным непосредственный  анализ уравнений и затрудняет оценку влияния разных факторов на динамику системы.


Цель настоящей работы -- получение уравнений движения по инерции экипажа с омни-колесами с массивными роликами в неголономной постановке с помощью подхода \cite{Tatarinov} в явном виде, исследование их свойств и сравнение поведения такой системы с поведением системы в безынерционном случае \cite{Zobova2011}.