\section{Приложение}
Матрица кинетической энергии:
$$
M = \begin{bmatrix}
    M &   &               &   &        &   &                        &        &                        \\
      & M &               &   &        &   &                        &        &                        \\
      &   & I_S           &   & \cdots &   & B\sin(\chi_k+\kappa_1) & \cdots & B\sin(\chi_i+\kappa_j) \\
      &   &               & J &        &   &                        &        &                        \\
      &   &               &   & \ddots &   &                        &        &                        \\
      &   &               &   &        & J &                        &        &                        \\
      &   & \text{\huge*} &   &        &   & B                      &        &                        \\
      &   &               &   &        &   &                        & \ddots &                        \\
      &   &               &   &        &   &                        &        & B                      \\
\end{bmatrix}
$$

Матрица связей:
$$
V = \begin{bmatrix}
    \begin{matrix}
        R\cos\theta                  & -R\sin\theta                  & 0                    \\[6pt]
        R\sin\theta                  &  R\cos\theta                  & 0                    \\[6pt]
        0                            & 0                             & \frac{1}{\Lambda}    \\[6pt]
        \frac{R\sin\alpha_i}{l}      & -\frac{R\cos\alpha_i}{l}      & -\frac{R}{\Lambda l} \\[6pt]
        \frac{R\cos\alpha_k}{\rho_k} &  \frac{R\sin\alpha_k}{\rho_k} & 0                    \\[6pt]
    \end{matrix}   & \text{\huge 0}                \\[50pt]
    \text{\huge 0} & \text{\huge E}_{N(n-1)} \quad \\[10pt]
\end{bmatrix}.
$$

Элементы матрицы кинетической энергии с учетом связей:

\begin{equation}\label{mstar}
    \begin{array}{rcl}
        m^*_{11} & = & MR^2 + \sum\limits_i \bigg( J\ddfrac{R^2}{l^2}\sin^2\alpha_i + B\ddfrac{R^2}{\rho_i^2}\cos^2\alpha_i\bigg),
        m^*_{22} & = & MR^2 + \sum\limits_i \bigg(J\ddfrac{R^2}{l^2}\cos^2\alpha_i + B\ddfrac{R^2}{\rho_i^2}\sin^2\alpha_i\bigg),
        m^*_{33} & = & \ddfrac{1}{\Lambda}(I_S + \sum\limits_i J\ddfrac{R^2}{l^2}),
        m^*_{12} & = & \sum\limits_i \bigg(-J\ddfrac{R^2}{l^2} + B\ddfrac{R^2}{\rho_i^2}\bigg)\sin\alpha_i\cos\alpha_i,
        m^*_{13} & = & \ddfrac{1}{\Lambda}\sum\limits_i \bigg(-J\ddfrac{R^2}{l^2}\sin\alpha_i +  B\ddfrac{R\sin\chi_i}{\rho_i}\cos\alpha_i\bigg),
        m^*_{23} & = & \ddfrac{1}{\Lambda}\sum\limits_i \bigg(J\ddfrac{R^2}{l^2}\cos\alpha_i +  B\ddfrac{R\sin\chi_i}{\rho_i}\sin\alpha_i\bigg),
    \end{array}
\end{equation}

Формальные импульсы $\vec{p} = \ddpd{L}{\dot{\vec{q}}}$:

\begin{equation}\label{p}
    \begin{array}{rcl}
        p_x & = & MR(\nu_1\cos\theta-\nu_2\sin\theta),\vsp
        p_y & = & MR(\nu_1\sin\theta+\nu_2\cos\theta),\vsp
        p_\theta & = & BR\sum\limits_i\ddfrac{\sin(\chi_i + \kappa_1)}{\rho_i}(\nu_1\cos\alpha_i + \nu_2\sin\alpha_i) + \ddfrac{I_S}{\Lambda}\nu_3 + B\sum\limits_s\sin(\chi_s)\nu_s,\vsp
        p_{\chi_i} & = & J\ddfrac{R}{l}(\nu_1\sin\alpha_i - \nu_2\cos\alpha_i - \ddfrac{1}{\Lambda}\nu_3),\vsp
        p_{\phi_{k1}} & = & \ddfrac{BR}{\rho_k}(\nu_1\cos\alpha_k + \nu_2\sin\alpha_k) + \ddfrac{B}{\Lambda}\nu_3\sin(\chi_k+\kappa_1),\vsp
        p_{\phi_s} & = & \ddfrac{B}{\Lambda}\nu_3\sigma_s.
    \end{array}
\end{equation}

Формальные импульсы $\vec{P}$:
\begin{equation}\label{P}
    \begin{array}{rcl}
        P_1 & = & R\bigg(p_x\cos\theta + p_y\sin\theta + \sum\limits_{i}\bigg(\ddfrac{\sin\alpha_ip_{\chi_i}}{l} +  \ddfrac{\cos\alpha_ip_{\phi_{i1}}}{\rho_i}\bigg)\bigg),\vsp
        P_2 & = & R\bigg(-p_x\sin\theta + p_y\cos\theta + \sum\limits_{i}\bigg(-\ddfrac{\cos\alpha_ip_{\chi_i}}{l} +  \ddfrac{\sin\alpha_ip_{\phi_{i1}}}{\rho_i}\bigg)\bigg),\vsp
        P_3 & = & \ddfrac{1}{\Lambda}\bigg(p_\theta + \sum\limits_{i}\ddfrac{R}{l}p_{\chi_i}\bigg),\vsp
        P_s & = & p_{\phi_s},
    \end{array}
\end{equation}
где от $\chi_i$ зависят $p_{\phi_{i1}}, p_\theta, \rho_i$ (см.\upr{p}), а потому и $P_1$ и $P_2$, отвечающие проекциям скорости центра масс платформы на подвижные оси, а также $P_3$, соответствующий вращению платформы.

TODO: ДОПИСАТЬ $\ddfrac{d}{dt}\pd{L^{*}}{\nu_\alpha}, \{P_\alpha, L^{*}\}, \{P_\alpha, \nu_\mu P_\mu\}$, см. test.tex.