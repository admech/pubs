\section{Уравнения движения}

Воспользуемся лаконичным методом получения уравнений движения для систем с дифференциальными связями, предложенным Я.В.~Татариновым \cite{Tatarinov}:
\begin{equation}\label{Tatarinov}
    \frac{d}{dt}\frac{\partial L^{*}}{\partial \nu_\alpha}  + \{P_\alpha, L^{*}\} = \{P_\alpha, \nu_\mu P_\mu\}.
\end{equation}
Здесь $L$ -- лагранжиан, $L^*$ -- он же с учетом связей. $P_\alpha$ -- линейные комбинации формальных ``импульсов'' $p_i$, определяемые из соотношения 
$$\nu_\mu P_\mu \equiv \dot{q_i} p_i,$$
 в котором $\dot{q}_i$ выражены через псевдоскорости $\nu_\mu$. Фигурными скобками $\{\cdot, \cdot\}$ обозначена скобка Пуассона по $q_i$, $p_i$. После ее вычисления выполняется подстановка 
$$\hspace{10pt} p_i = \frac{\partial L}{\partial \dot{q}_i}$$
Эти уравнения могут быть получены из уравнений Маджи, подробно их вывод и применение изложено в \cite{Tatarinov,Zobova2011,Zobova_ND}.

Так как потенциальная энергия системы во время движения не меняется, то лагранжиан  равен кинетической энергии:
\begin{equation}\label{kin_en}
    2L = 2T = M\vec{v}_S^2 + I_S\dot{\theta}^2 + J\sum_i\dot{\chi}_i^2 + B\sum_{i,j}(\dot{\phi}_{ij}^2 + 2\dot{\theta}\sin(\kappa_j + \chi_i)\dot{\phi}_{ij})=\dot{\vec{q}}^\mathrm{T}\M\dot{\vec{q}}
\end{equation}
%\sout{Здесь полная масса системы -- $M = \mathring{M} + Nnm$, момент инерции всей системы относительно $SZ$ -- $I_S = \mathring{I_S} + Nn(\frac{A+B}{2} + mR^2 + \frac{mr^2}{2})$, момент инерции колеса (с роликами) относительно его оси $J = \mathring{J} + n(A + mr^2)$, где $\mathring{M}, \mathring{I_S}, \mathring{J}$ -- масса и моменты инерции системы и колес без учета роликов; $m$ -- масса ролика; $A$ -- момент инерции ролика относительно любой оси, перпендикулярной его оси собственного вращения и проходящей через его центр масс; $r$ -- радиус диска колеса (расстояние от центра колеса до центра ролика),}
Здесь $M,\ I_S,\ J$ --- массово-инерционные характеристики экипажа (см. приложение), $B$ -- момент инерции ролика относительно его оси вращения. Лагранжиан с учетом связей имеет вид:
$$ 2L^{*}  = \vec{\nu}^\mathrm{T} V^\mathrm{T}\M V\vec{\nu} = \vec{\nu}^\mathrm{T} \M^*(\chi_i)\vec{\nu} $$
Структура матрицы $\M^*$ следующая:
$$
\M^* = \begin{bmatrix}
        \left(\begin{matrix}&&\\&m^*_{ij}&\\&&\end{matrix}\right)_{3\times3} \quad & \left(\begin{matrix} 0&\ldots& 0 \\ 0&\ldots&0 \\ B\Lambda^{-1}\sin\chi_{11}&\ldots& B\Lambda^{-1}\sin\chi_{nN} \end{matrix}\right) \\[25pt]
        *          & \begin{matrix} B & & \\ & \ddots & \\ & & B \end{matrix} \\
    \end{bmatrix},
$$

%\[ \M^* = \left[ \begin{array}{c|c}
%m^* & c \\ \cline{1-2}
%c^\mathrm{T} & b
%\end{array} \right] \]
Явные формулы для коэффициентов $m^*_{ij}$ главного минора $3\times3$ выписаны в приложении; отметим, что они зависят только от координат $\chi_i$, которые входят в дроби вида $B/\rho_i^2$ и $B\sin\chi_i/\rho_i$, имеющие неустранимый разрыв при смене роликов (см. (\ref{constraint_roller_contact})). Этот минор соответствует псевдоскоростям $\nu_1,\ \nu_2,\ \nu_3$. Остальные элементы матрицы $\M^*$
соответствуют скоростям свободных роликoв $\nu_s$, для которых 
 $\chi_{kl} = \chi_k+\kappa_l$  --- угол между вертикалью и осью ролика. Индекс $k = 1,\dots,N$ означает номер колеса, индекс $l = 1, 2,\ldots,\not{k},\ldots, n$ -- номер ролика на колесе, вычеркнут индекс ролика, находящегося в контакте.
  %($l = 1$ -- ролик, находящийся в контакте).
 
 Выпишем выражения для $P_\alpha$:
\begin{equation}\label{P}
    \begin{array}{rcl}
        P_1 & = & R\bigg(p_x\cos\theta + p_y\sin\theta + \sum\limits_{i}\bigg(\ddfrac{\sin\alpha_ip_{\chi_i}}{l} +  \ddfrac{\cos\alpha_ip_{\phi_{i1}}}{\rho_i}\bigg)\bigg),\vsp
        P_2 & = & R\bigg(-p_x\sin\theta + p_y\cos\theta + \sum\limits_{i}\bigg(-\ddfrac{\cos\alpha_ip_{\chi_i}}{l} +  \ddfrac{\sin\alpha_ip_{\phi_{i1}}}{\rho_i}\bigg)\bigg),\vsp
        P_3 & = & \ddfrac{1}{\Lambda}\bigg(p_\theta - \sum\limits_{i}\ddfrac{R}{l}p_{\chi_i}\bigg),\vsp
        P_s & = & p_{\phi_s},
    \end{array}
\end{equation}

Поскольку $L^{*}$ зависит только от $\chi_i$, то его скобка Пуассона с $P_\alpha$, $\alpha=1,\dots, 3$ --- квадратичная форма псевдоскоростей, пропорциональная моменту инерции ролика $B$ с коэффициентами, зависящими от $\chi_i$:
$$
\{P_1, L^*\} = -\frac{\partial P_1}{\partial p_{\chi_i}}\frac{\partial L^*}{\partial \chi_i} = -\frac{R}{2l}\vec{\nu}^\mathrm{T}\sin\alpha_i\M^*_i\vec{\nu},\text{ где }\M^*_i = \frac{\partial \M^*}{\partial \chi_i}
$$
$$
\{P_2, L^*\} = \frac{R}{2l}\vec{\nu}^\mathrm{T}\cos\alpha_i\M^*_i\vec{\nu},\  
\{P_3, L^*\} = \frac{R}{2l\Lambda}\vec{\nu}^\mathrm{T}\M^*_i\vec{\nu},\quad \{P_s,L^*\} = 0, s>3
$$

Скобки Пуассона в правой части имеют вид (звездочкой обозначена подстановка формальных импульсов $p_i$ -- выражения для них см. в приложении; дополнительно введено обозначение $\tau_k = \ddfrac{sin\chi_k}{\rho_k^2}$):

$$
(\{P_1,P_2\})^* = (-\sum\limits_{k=1}^{N} R^2\tau_kp_{\phi_k})^* =
-BR^2(R\nu_1 S_1^1 + R\nu_2 S_2^1 + \Lambda^{-1}\nu_3S_3^1)$$
$$
S_1^1 = \sum\limits_{k=1}^{N}\frac{\cos\alpha_k}{\rho_k}\tau_k,\
S_2^1 = \sum\limits_{k=1}^{N}\frac{\sin\alpha_k}{\rho_k}\tau_k,\
S_3^1 = \sum\limits_{k=1}^{N}\sin\chi_k\tau_k,
$$

\begin{eqnarray*}
(\{P_1,P_3\})^* &=& \frac{R}{\Lambda}\left(-\sin\theta p_x + \cos\theta p_y - \sum\limits_{k=1}^{N} R\cos\alpha_k\tau_kp_{\phi_k}\right)^* =\\
&=& -\frac{BR^2}{\Lambda}(R\nu_1 S_1^2 + R\nu_2 S_2^2 + \Lambda^{-1}\nu_3S_3^2),
\end{eqnarray*}
$$
S_1^2 = \sum\limits_{k=1}^{N}\cos\alpha_k\frac{\cos\alpha_k}{\rho_k}\tau_k,\
S_2^2 = \sum\limits_{k=1}^{N}\cos\alpha_k\frac{\sin\alpha_k}{\rho_k}\tau_k - \frac{M}{BR},
$$$$
S_3^2 = \sum\limits_{k=1}^{N}\cos\alpha_k\sin\chi_k\tau_k,
$$

\begin{eqnarray*}
(\{P_2,P_3\})^* &=& \frac{R}{\Lambda}\left(-\cos\theta p_x - \sin\theta p_y - \sum\limits_{k=1}^{N} R\sin\alpha_k\tau_kp_{\phi_k}\right)^*  =\\
&=& -\frac{BR^2}{\Lambda}(R\nu_1 S_1^3 + R\nu_2 S_2^3 + \Lambda^{-1}\nu_3S_3^3),
\end{eqnarray*}
$$
S_1^3 = \sum\limits_{k=1}^{N}\sin\alpha_k\frac{\cos\alpha_k}{\rho_k}\tau_k + \frac{M}{BR},\
S_2^3 = \sum\limits_{k=1}^{N}\sin\alpha_k\frac{\sin\alpha_k}{\rho_k}\tau_k,
$$$$
S_3^3 = \sum\limits_{k=1}^{N}\sin\alpha_k\sin\chi_k\tau_k,
$$
$$
\{P_\alpha, P_\beta\} =0,\ \alpha,\beta >3
$$
Собирая все вместе, окончательно получим следующую структуру уравнений:
\begin{eqnarray}\label{eq:full_system}
\M^*\dot{\vec{\nu}} + \frac{R}{l}\sum\limits_{i=1}^{N}
\Big(\M^*_i(\nu_1\sin\alpha_i-\nu_2\cos\alpha_i - \Lambda^{-1}\nu_3)\Big)\vec{\nu} + 
\\ \nonumber
+\frac{R}{2l}\vec{\nu}^\mathrm{T}
\left(\begin{matrix}
-\sin\alpha_i \M^*_i\\
\cos\alpha_i \M^*_i\\
\Lambda^{-1}\M^*_i\\
0\\
\vdots
\\
0
\end{matrix}\right)
\vec{\nu} = 
\frac{MR^2}{\Lambda}\left(
\begin{matrix}
\nu_2\nu_3\\
-\nu_1\nu_3\\
0\\
0\\
\vdots
\\
0
\end{matrix}\right)
-B
\left(
\begin{matrix}
\text{кв. ф. }\nu_1,\nu_2,\nu_2\\
\text{кв. ф. }\nu_1,\nu_2,\nu_2\\
\text{кв. ф. }\nu_1,\nu_2,\nu_2\\
0\\
\vdots
\\
0
\end{matrix}\right)
\end{eqnarray}

Что хочу от такой структуры уравнений:

1) ччтобы было понятно, что $B=0 \Rightarrow \M^*_i =0$ и остаются старые уравнения
2) что решение $\nu_1= 0, \nu_2 = 0, \nu_3 = const,\ \nu_s = 0$, было видно из уравнений. Для тех ограничений, что приняты (колесы перпендикулярны радиусам $SP_i$) оно всегда должно существовать при любых $\alpha_i$. Это наводит на мысль, что какие-то суммы должны быть тождественно равны нулю (как сейчас кажется, при $\nu_3^2$ в последнем столбцы). А может и еще какие-то суммы? Может, Maxima спросить?
3) сейчас кажется, что второе и третье слагаемое в левой части -- подобные слагаемые. Так ли это?
