\section{Уравнения движения}

Воспользуемся лаконичным методом получения уравнений движения для систем с дифференциальными связями, предложенным Я.В.~Татариновым \cite{Tatarinov}:
\begin{equation}\label{Tatarinov}
    \frac{d}{dt}\frac{\partial L^{*}}{\partial \nu_\alpha}  + \{P_\alpha, L^{*}\} = \{P_\alpha, \nu_\mu P_\mu\}.
\end{equation}
Здесь $L$ -- лагранжиан, $L^*$ -- он же с учетом связей. $P_\alpha$ -- линейные комбинации формальных ``импульсов'' $p_i$, определяемые из соотношения 
$$\nu_\mu P_\mu \equiv \dot{q_i} p_i,$$
 в котором $\dot{q}_i$ выражены через псевдоскорости $\nu_\mu$. Фигурными скобками $\{\cdot, \cdot\}$ обозначена скобка Пуассона по $q_i$, $p_i$. После ее вычисления выполняется подстановка 
$$\hspace{10pt} p_i = \frac{\partial L}{\partial \dot{q}_i}$$
Эти уравнения могут быть получены из уравнений Маджи, подробно их вывод и применение изложено в \cite{Tatarinov,Zobova2011,Zobova_ND}.

Так как потенциальная энергия системы во время движения не меняется, то лагранжиан  равен кинетической энергии:
\begin{equation}\label{kin_en}
    2L = 2T = M\vec{v}_S^2 + I_S\dot{\theta}^2 + J\sum_i\dot{\chi}_i^2 + B\sum_{i,j}(\dot{\phi}_{ij}^2 + 2\dot{\theta}\sin(\kappa_j + \chi_i)\dot{\phi}_{ij})=\dot{\vec{q}}^\mathrm{T}\M\dot{\vec{q}}
\end{equation}
%\sout{Здесь полная масса системы -- $M = \mathring{M} + Nnm$, момент инерции всей системы относительно $SZ$ -- $I_S = \mathring{I_S} + Nn(\frac{A+B}{2} + mR^2 + \frac{mr^2}{2})$, момент инерции колеса (с роликами) относительно его оси $J = \mathring{J} + n(A + mr^2)$, где $\mathring{M}, \mathring{I_S}, \mathring{J}$ -- масса и моменты инерции системы и колес без учета роликов; $m$ -- масса ролика; $A$ -- момент инерции ролика относительно любой оси, перпендикулярной его оси собственного вращения и проходящей через его центр масс; $r$ -- радиус диска колеса (расстояние от центра колеса до центра ролика),}
Здесь $M,\ I_S,\ J$ --- массово-инерционные характеристики экипажа (см. приложение), $B$ -- момент инерции ролика относительно его оси вращения. Лагранжиан с учетом связей имеет вид:
$$ 2L^{*}  = \vec{\nu}^\mathrm{T} V^\mathrm{T}\M V\vec{\nu} = \vec{\nu}^\mathrm{T} \M^*(\chi_i)\vec{\nu} $$
Структура матрицы $\M^*$ следующая:
$$
\M^* = \begin{bmatrix}
        \left(\begin{matrix}&&\\&m^*_{ij}&\\&&\end{matrix}\right)_{3\times3} \quad & \left(\begin{matrix} 0&\ldots& 0 \\ 0&\ldots&0 \\ B\Lambda^{-1}\sin\chi_{11}&\ldots& B\Lambda^{-1}\sin\chi_{nN} \end{matrix}\right) \\[25pt]
        *          & \begin{matrix} B & & \\ & \ddots & \\ & & B \end{matrix} \\
    \end{bmatrix},
$$

%\[ \M^* = \left[ \begin{array}{c|c}
%m^* & c \\ \cline{1-2}
%c^\mathrm{T} & b
%\end{array} \right] \]
Явные формулы для коэффициентов $m^*_{ij}$ главного минора $3\times3$ выписаны в приложении; отметим, что они зависят только от координат $\chi_i$, которые входят в дроби вида $B/\rho_i^2$ и $B\sin\chi_i/\rho_i$, имеющие неустранимый разрыв при смене роликов (см. (\ref{constraint_roller_contact})). Этот минор соответствует псевдоскоростям $\nu_1,\ \nu_2,\ \nu_3$. Остальные элементы матрицы $\M^*$
соответствуют скоростям свободных роликoв $\nu_s$, для которых 
 $\chi_{kl} = \chi_k+\kappa_l$  --- угол между вертикалью и осью ролика. Индекс $k = 1,\dots,N$ означает номер колеса, индекс $l = 1, 2,\ldots,\not{k},\ldots, n$ -- номер ролика на колесе, вычеркнут индекс ролика, находящегося в контакте.
  %($l = 1$ -- ролик, находящийся в контакте).
 
 Выпишем выражения для $P_\alpha$:
\begin{equation}\label{P}
    \begin{array}{rcl}
        P_1 & = & R\bigg(p_x\cos\theta + p_y\sin\theta + \sum\limits_{i}\bigg(\ddfrac{\sin\alpha_ip_{\chi_i}}{l} +  \ddfrac{\cos\alpha_ip_{\phi_{i1}}}{\rho_i}\bigg)\bigg),\vsp
        P_2 & = & R\bigg(-p_x\sin\theta + p_y\cos\theta + \sum\limits_{i}\bigg(-\ddfrac{\cos\alpha_ip_{\chi_i}}{l} +  \ddfrac{\sin\alpha_ip_{\phi_{i1}}}{\rho_i}\bigg)\bigg),\vsp
        P_3 & = & \ddfrac{1}{\Lambda}\bigg(p_\theta - \sum\limits_{i}\ddfrac{R}{l}p_{\chi_i}\bigg),\vsp
        P_s & = & p_{\phi_s},
    \end{array}
\end{equation}

Поскольку $L^{*}$ зависит только от $\chi_i$, то его скобка Пуассона с $P_\alpha$, $\alpha=1,\dots, 3$ --- квадратичная форма псевдоскоростей, пропорциональная моменту инерции ролика $B$ с коэффициентами, зависящими от $\chi_i$:
$$
\{P_1, L^*\} = -\frac{\partial P_1}{\partial p_{\chi_i}}\frac{\partial L^*}{\partial \chi_i} = -\frac{R}{2l}\vec{\nu}^\mathrm{T}\sin\alpha_i\M^*_i\vec{\nu},\text{ где }\M^*_i = \frac{\partial \M^*}{\partial \chi_i}
$$
$$
\{P_2, L^*\} = \frac{R}{2l}\vec{\nu}^\mathrm{T}\cos\alpha_i\M^*_i\vec{\nu},\  
\{P_3, L^*\} = \frac{R}{2l\Lambda}\vec{\nu}^\mathrm{T}\M^*_i\vec{\nu},\quad \{P_s,L^*\} = 0, s>3
$$

Cуммы $\{P_\alpha, \nu_\mu P_\mu\}$ в правой части также отличны от нуля лишь для первых трех уравнений, и после подстановки формальных импульсов $p_i$ их можно записать как $$\ddfrac{MR^2}{\Lambda}(\nu_2\nu_3, -\nu_1\nu_3, 0, 0, \ldots, 0)^T - BR^2 \vec{\nu}^T (\Prhs_\alpha)^T \vec{\nu},$$ где матрицы $\Prhs_\alpha$ квадратичных форм задаются следующим образом:

$$
\tau_k = \ddfrac{sin\chi_k}{\rho_k^2}, \quad s_k = ( R\frac{\cos\alpha_k}{\rho_k}, R\frac{\sin\alpha_k}{\rho_k}, \frac{1}{\Lambda}\sin\chi_k )\tau_k,
$$
$$
u_k = ( 1, \frac{1}{\Lambda}\cos\alpha_k, \frac{1}{\Lambda}\sin\alpha_k ), \quad \AntiU_k = \left(\begin{matrix}
    0      & u_k^1  & u_k^2\\
    -u_k^1 & 0      & u_k^3\\
    -u_k^2 & -u_k^3 & 0    \\
\end{matrix}
\right),
$$
$$
(\Prhs_\alpha)_{ij} = \sum_k (\AntiU_k)_{\alpha j}s^i_k, i,j \leq 3; \quad (\Prhs_\alpha)_{ij} = 0, i,j > 3; \quad \Prhs_\alpha = 0, \alpha > 3. \\
$$
Более подробные выражения для $p_i$ и $\{P_i, P_j\}$ см. в приложении.

Собирая все вместе, окончательно получим следующую структуру уравнений:
\begin{eqnarray}\label{eq:full_system}
\M^*\dot{\vec{\nu}} = 
\frac{MR^2}{\Lambda}\left(\begin{matrix}
    \nu_2\nu_3\\
    -\nu_1\nu_3\\
    0\\
    0\\
    \vdots
    \\
    0
\end{matrix}\right)
+\vec{\nu}^\mathrm{T}\left(
\frac{R}{2l}
\left(\begin{matrix}
    -\sin\alpha_i \M^*_i\\
    \cos\alpha_i \M^*_i\\
    \Lambda^{-1}\M^*_i\\
    0\\
    \vdots
    \\
    0
    \end{matrix}
\right)
-BR^2
\left(\begin{matrix}
    \Prhs_1\\
    \Prhs_2\\
    \Prhs_3\\
    0\\
    \vdots
    \\
    0
\end{matrix}\right)
\right)
\vec{\nu}
\end{eqnarray}

Заметим, что:
\begin{enumerate}
    \item при $B = 0$ квадратичные формы в правой части обращаются в ноль, как и все члены, соответствующие роликам, в левой части (см.\upr{mstar}), и остаются уравнения движения для системы без роликов;

    \item для симметричных конфигураций (для которых $\sum_{k=1}^N\cos\alpha_k = \sum_{k=1}^N\sin\alpha_k = 0$, например, $\alpha_k = \ddfrac{2\pi (k-1)}{N}, k = 1,\ldots,N$) существует решение $\nu_1= 0, \nu_2 = 0, \nu_3 = const, \nu_s = 0$: действительно, левая часть и первые два слагаемых правой обращаются в ноль (второе -- т.к.\upr{mstar} $m^*_{33} = const$), третье принимает вид
    $$\sum_{k}(\cos\alpha_k, \sin\alpha_k, 0, 0, \ldots, 0)^T\ddfrac{\sin^2\chi_k}{\Lambda^2\rho^2_k},$$
    где, в силу характера движения и ортогональности плоскостей колес их радиус-векторам, $\chi_k = \chi$ (а потому и $\rho_k = \rho$), $\forall k$, и, следовательно, также равно нулю.

\end{enumerate}


Что хочу от такой структуры уравнений:

что решение $\nu_1= 0, \nu_2 = 0, \nu_3 = const,\ \nu_s = 0$, было видно из уравнений. Для тех ограничений, что приняты (колесы перпендикулярны радиусам $SP_i$) оно всегда должно существовать при любых $\alpha_i$. Это наводит на мысль, что какие-то суммы должны быть тождественно равны нулю (как сейчас кажется, при $\nu_3^2$ в последнем столбцы). А может и еще какие-то суммы? Может, Maxima спросить?
