\section{Уравнения движения}

Воспользуемся лаконичным методом получения уравнений движения для систем с дифференциальными связями, предложенным Я.В.~Татариновым \cite{Tatarinov}:
\begin{equation}\label{Tatarinov}
    \frac{d}{dt}\frac{\partial L^{*}}{\partial \nu_\alpha}  + \{P_\alpha, L^{*}\} = \{P_\alpha, \nu_\mu P_\mu\}.
\end{equation}
Здесь $L$ -- лагранжиан, $L^*$ -- он же с учетом связей. $P_\alpha$ -- линейные комбинации формальных ``импульсов'' $p_i$, определяемые из соотношения 
$$\nu_\mu P_\mu \equiv \dot{q_i} p_i,$$
 в котором $\dot{q}_i$ выражены через псевдоскорости $\nu_\mu$. Фигурными скобками $\{\cdot, \cdot\}$ обозначена скобка Пуассона по $q_i$, $p_i$. После ее вычисления выполняется подстановка 
$$\hspace{10pt} p_i = \frac{\partial L}{\partial \dot{q}_i}$$
Эти уравнения могут быть получены из уравнений Маджи, подробно их вывод и применение изложено в \cite{Tatarinov,Zobova2011,Zobova_ND}.

Так как потенциальная энергия системы во время движения не меняется, то лагранжиан  равен кинетической энергии:
\begin{equation}\label{kin_en}
    2L = 2T = M\vec{v}_S^2 + I_S\dot{\theta}^2 + J\sum_i\dot{\chi}_i^2 + B\sum_{i,j}(\dot{\phi}_{ij}^2 + 2\dot{\theta}\sin(\kappa_j + \chi_i)\dot{\phi}_{ij})=\dot{\vec{q}}^\mathrm{T}M\dot{\vec{q}}
\end{equation}
\sout{Здесь полная масса системы -- $M = \mathring{M} + Nnm$, момент инерции всей системы относительно $SZ$ -- $I_S = \mathring{I_S} + Nn(\frac{A+B}{2} + mR^2 + \frac{mr^2}{2})$, момент инерции колеса (с роликами) относительно его оси $J = \mathring{J} + n(A + mr^2)$, где $\mathring{M}, \mathring{I_S}, \mathring{J}$ -- масса и моменты инерции системы и колес без учета роликов; $m$ -- масса ролика; $A$ -- момент инерции ролика относительно любой оси, перпендикулярной его оси собственного вращения и проходящей через его центр масс; $r$ -- радиус диска колеса (расстояние от центра колеса до центра ролика),}
Здесь $M,\ I_S,\ J$ --- массово-инерционные характеристики экипажа (см. приложение), $B$ -- момент инерции ролика относительно его оси вращения.

Лагранжиан с учетом связей имеет вид:
$$ 2L^{*} = \nu M^*\nu = \nu^T V^TMV\nu, $$
где $M, V$ -- матрицы кинетической энергии и связей (см. приложение).

Структура матрицы $M^*$ следующая:
$$
M^* = \begin{bmatrix}
        (m^*_{ij}) \quad & \begin{matrix} 0 \\ 0 \\ \ddfrac{B}{\Lambda}\sigma_s \end{matrix} \\[25pt]
        *          & BE_{N(n-1)}                                                                  \\
    \end{bmatrix},
$$
Элементы $m_{ii}$ на диагонали постоянны, элементы $m_{ij},\ i\neq j$ зависят от углов поворота колес $\chi_i$, причем содержат функцию $...$ которая имеет разрыв второго рода при смене роликов. 

где $\sigma_s = \sin(\chi_k+\kappa_l)$, и $k, l$ соответствуют свободным роликам, $E$ -- единичная матрица указанной размерности,
$$ m^*_{11} = MR^2 + \sum\limits_i \bigg( J\ddfrac{R^2}{l^2}\sin^2\alpha_i + B\ddfrac{R^2}{\rho_i^2}\cos^2\alpha_i\bigg), $$
$$ m^*_{22} = MR^2 + \sum\limits_i \bigg(J\ddfrac{R^2}{l^2}\cos^2\alpha_i + B\ddfrac{R^2}{\rho_i^2}\sin^2\alpha_i\bigg), $$
$$ m^*_{33} = \ddfrac{1}{\Lambda}(I_S + \sum\limits_i J\ddfrac{R^2}{l^2}), $$
$$ m^*_{12} = \sum\limits_i \bigg(-J\ddfrac{R^2}{l^2} + B\ddfrac{R^2}{\rho_i^2}\bigg)\sin\alpha_i\cos\alpha_i, $$
$$ m^*_{13} = \ddfrac{1}{\Lambda}\sum\limits_i \bigg(-J\ddfrac{R^2}{l^2}\sin\alpha_i +  B\ddfrac{R\sin\chi_i}{\rho_i}\cos\alpha_i\bigg), $$
$$ m^*_{23} = \ddfrac{1}{\Lambda}\sum\limits_i \bigg(J\ddfrac{R^2}{l^2}\cos\alpha_i +  B\ddfrac{R\sin\chi_i}{\rho_i}\sin\alpha_i\bigg),$$
причем диагональные элементы постоянны, а $m^*_{ij} = m^*_{ij}(\chi_1,\ldots,\chi_N), i \neq j$, зависят от углов поворота колес. Формальные импульсы $\vec{P}$:
\begin{equation}\label{P}
    \begin{array}{rcl}
        P_1 & = & R\bigg(p_x\cos\theta + p_y\sin\theta + \sum\limits_{i}\bigg(\ddfrac{\sin\alpha_ip_{\chi_i}}{l} +  \ddfrac{\cos\alpha_ip_{\phi_{i1}}}{\rho_i}\bigg)\bigg),\vsp
        P_2 & = & R\bigg(-p_x\sin\theta + p_y\cos\theta + \sum\limits_{i}\bigg(-\ddfrac{\cos\alpha_ip_{\chi_i}}{l} +  \ddfrac{\sin\alpha_ip_{\phi_{i1}}}{\rho_i}\bigg)\bigg),\vsp
        P_3 & = & \ddfrac{1}{\Lambda}\bigg(p_\theta + \sum\limits_{i}\ddfrac{R}{l}p_{\chi_i}\bigg),\vsp
        P_s & = & p_{\phi_s},
    \end{array}
\end{equation}
где от $\chi_i$ зависят $p_{\phi_{i1}}, p_\theta, \rho_i$ (см.\upr{p}), а потому и $P_1$ и $P_2$, отвечающие проекциям скорости центра масс платформы на подвижные оси, а также $P_3$, соответствующий вращению платформы.

Явные выражения для $\ddfrac{d}{dt}\pd{L^{*}}{\nu_\alpha}, \{P_\alpha, L^{*}\}, \{P_\alpha, \nu_\mu P_\mu\}$ более громоздки и находятся в приложении.