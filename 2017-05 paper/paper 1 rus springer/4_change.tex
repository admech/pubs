\section{Переход между роликами}

\begin{figure}
    \minipage{0.5\textwidth}
        \centering
        \asyinclude{./asy/pic_overlap.asy}
        \caption{Ролики перекрываются}
        \label{fig:overlap}
    \endminipage
    \minipage{0.5\textwidth}
        \centering
        \asyinclude{./asy/pic_change.asy}
        \caption{Переход между роликами}
        \label{fig:change}
    \endminipage
\end{figure}

Чтобы исследовать поведение системы, необходимо описать переход колеса с одного ролика на другой. Здесь мы немедленно сталкиваемся с затруднением, указанным выше: уравнения движения вырождаются на стыках роликов, т.к. квадратичные формы $\boldsymbol{F}_i$ терпят разрыв 2ого рода из-за выражений $\rho_i = l\cos\chi_i-r$ в знаменателе.

Заметим, что в технических реализациях роликонесущих колес ситуация $\rho_i = 0$ никогда не имеет места, т.к. концы роликов усекаются (ФОТО?) (в частности, потому что оси роликов в реальных системах имеют ненулевую толщину и должны быть закреплены в колесах), а чтобы граница проекции колеса на его плоскость оставалась окружностью, ролики располагают либо под углом (как в меканум-колесах, (ФОТО?)), либо в два или больше рядов (ФОТО?). Подобные расположения приводят, впрочем, к скачкам точек контакта в моменты смены роликов, что также создаст разрывы в правых частях уравнений.

Поэтому для начала, мы усечем ролики (см. рис.~\ref{fig:overlap}), но оставим их оси в одной плоскости, допуская пересечение тел роликов в пространстве и пренебрегая им.

Кроме этого, при смене контакта происходит мгновенное наложение связи на вновь вошедший в контакт ролик (и снятие её с освободившегося). В настоящей работе, будем считать, что ролик, входящий в контакт, мгновенно оказывается закручен до той же угловой скорости, до которой закручен освобождающийся ролик, а последний мгновенно прекратит собственное вращение. Таким образом, мы будем рассматривать только движение роликов, находящихся в контакте.

Для этого, отбросим уравнения
$$
\Lambda\dot{\nu}_{ni+j} = -\dot{\nu}_3\sin(\chi_i+\kappa_j) - \dot{\chi_i}\nu_3\cos(\chi_i+\kappa_j), \quad i = 1..n, j = 2..n,
$$
описывающие движение свободных роликов, и при переходе ($\chi_i = \chi_i^+$) сохраним значения $\nu_1$, $\nu_2$, 
$\nu_3$, заменим $\chi_i$ с $\chi_i^+$ на $\chi_i^-$ (см. рис.~\ref{fig:change}), а $\dot\chi_i$ пересчитаем по уравнениям связей.
