\section{Постановка задачи}

\begin{figure}
    \minipage{0.5\textwidth}
        \centering
        \asyinclude{./asy/pic_cart.asy}
        \caption{Экипаж}
        \label{fig:vehicle}
    \endminipage
    \minipage{0.5\textwidth}
        \centering
        \asyinclude{./asy/pic_wheel.asy}
        \caption{Колесо}
        \label{fig:wheel}
    \endminipage
\end{figure}

Рассмотрим экипаж с омни-колесами, движущийся по инерции по неподвижной абсолютно шероховатой горизонтальной плоскости. Экипаж состоит из платформы и $N$ омни-колес, плоскости которых относительно платформы неподвижны. Каждое колесо может свободно вращаться относительно платформы вокруг собственной оси, расположенной горизонтально. Будем считать, что на каждом колесе установлено $n$ массивных роликов, так что оси роликов лежат в плоскостях колёс и направлены по касательной к границам дисков колес (см. рис.~\ref{fig:wheel}). Таким образом, система состоит из $N(n+1) + 1$ абсолютно твердых тел. 

Введем неподвижную систему отсчета так, что ось $OZ$ направлена вертикально вверх, а плоскость $OXY$ совпадает с опорной плоскостью.
Введем также подвижную систему отсчета $S\xi\eta Z$, жестко связанную с платформой экипажа так, что плоскость $S\xi\eta$ горизонтальна и содержит центры всех колес $P_i$. Будем считать, что оси колес лежат на лучах, соединяющих центр платформы $S$ и центры колес (см. рис.~\ref{fig:vehicle}), а расстояния от центров колес до $S$ одинаковы и равны $R$. Геометрию установки колес на платформе зададим углами $\alpha_i$ осями колес и осью $S\xi$
(см. рис.~\ref{fig:wheel}). Введем также орты, жестко связанные с дисками колес: пусть $\vec{n}_i = \vec{SP_i}/|\vec{SP_i}|$ -- единичный орт оси $i$-ого колеса, и орты $\vec{n}_i^\perp$ и $\vec{n}_i^z$, лежащие в плоскости диска колеса, так что вектор $\vec{n}_i^z$ вертикален при нулевом повороте колеса. Положения центров роликов на колесе определим углами $\kappa_j$ между ними и направлением, противоположным вектору $\vec{n}_i^z$. 

Положение экипажа будем задавать следующими координатами:
$x, y$ --- координаты точки $S$ на плоскости $OXY$, $\theta$ -- угол между $OX$ и $S\xi$ (угол курса),
$\chi_i$ ($i = 1\dots N$) -- углы поворота колес вокруг их осей, отсчитываемые против часовой стрелки, если смотреть с конца вектора $\vec{n}_i$, и $\phi_j$ -- углы поворота роликов вокруг их собственных осей.
Таким образом, вектор обобщенных координат имеет вид:
$$\vec{q} = (x, y, \theta, \left.\{\chi_i\}\right|_{i=1}^N , \left.\{\phi_k\}\right|_{k=1}^N, \phi_s)^{\mathop{T}}\in\mathbb{R}^{N(n+1) + 3}$$ 
Будем использовать индекс $k$ для углов поворота роликов, находящихся в данный момент в контакте с опорной плоскостью, a $s$ --- для остальных,  ``cвободных'', роликов.

Введем псевдоскорости
$$\vec{\nu} = (\nu_1, \nu_2, \nu_3, \nu_s), \quad \vec{v}_S = R\nu_1\vec{e}_\xi + R\nu_2\vec{e}_\eta, \quad \nu_3 = \Lambda\dot{\theta},\quad \nu_s = \dot{\phi}_s$$
Их механический смысл таков: $\nu_1$, $\nu_2$ --- проекции скорости точки $S$ на оси $S\xi\eta$, связанные с платформой, $\nu_3$ --- с точностью до множителя угловая скорость платформы, $\nu_s$ --- угловые скорости свободных роликов. Таким образом, имеем
$$ \dot{x} = R \nu_1\cos\theta-R\nu_2\sin\theta, \hspace{15pt} \dot{y} = R\nu_1\sin\theta+R\nu_2\cos\theta,$$

Будем считать, что проскальзывания между опорной плоскостью и роликами в контакте не происходит, т.е.
скорости точек $C_i$ контакта равны нулю:
$$\vec{v}_{C_i} = 0,\quad i = 1\dots N.$$
Выражая скорость точек контакта через введенные псевдоскорости и проектируя на векторы $\vec{n}_i$ и $[\vec{e}_Z,\ \vec{n}_i]$ соответсвенно, получим:
\begin{eqnarray}
\dot{\phi_k} &=& \frac{R}{\rho_k }(\nu_1\cos\alpha_k + \nu_2\sin\alpha_k),\text{ где } \rho_k  = l\cos\chi_k - r \label{constraint_roller_contact}\\
\dot{\chi}_i &=& \frac{R}{l}(\nu_1\sin\alpha_i - \nu_2\cos\alpha_i - \frac{\nu_3}{\Lambda})\label{constraint_wheel_contact}
\end{eqnarray}
Заметим, что знаменатель $\rho_k$ в (\ref{constraint_roller_contact}) есть расстояние от оси ролика до точки контакта, обращающееся в ноль на стыке роликов (см. рис.~\ref{fig:wheel}). Это обстоятельство приводит к неустранимым разрывам правых частей уравнений движения и будет рассмотрено отдельно ниже.
Уравнение (\ref{constraint_wheel_contact}) совпадает со связью в случае безынерционной модели. 

Таким образом, выражение обобщенных скоростей через псевдоскорости, учитывающее связи, наложенные на систему, можно записать в матричном виде (явные выражения компонент матрицы $V$ приведены в приложении):
\begin{equation}
    \dot{\vec{q}} = V\vec{\nu},\quad V = V(\theta,\chi_i)
\end{equation}
