\section{Примеры движений}

Численные решения получим для симметричного трехколесного экипажа ($\alpha_i = \frac{2\pi}{N}(i - 1), N = 3$), с $n = 5$ роликами на колесе (для случая без роликов -- с колесами соответствующей массы) и следующих движений:
\begin{enumerate}
  \item \label{sol:selfrot} вращение вокруг своей оси ($\nu_1(0) = \nu_2(0) = 0, \nu_3(0) = 1$) -- см. рис.~\ref{fig:old_selfrot},~\ref{fig:selfrot},
  \item \label{sol:straight} движение по прямой в направлении оси одного из колес ($\nu_1(0) = 1, \nu_2(0) = \nu_3(0) = 0$) -- см. рис.~\ref{fig:old_straight},~\ref{fig:straight}
  \item \label{sol:wrench} движение с ненулевой скоростью центра масс и, одновременно, с ненулевой угловой скоростью платформы (т.е. ``с закруткой'') ($\nu_1(0) = 1, \nu_2(0) = 0, \nu_3(0) = 1$) -- см. рис.~\ref{fig:old_wrench},~\ref{fig:wrench}~и~\ref{fig:wrench_short}.
\end{enumerate}

Траектории центра экипажа $S$ на плоскости $OXY$ и характер вращения вокруг вертикальной оси $SZ$ ($\theta(t)$) близки или совпадают между двумя постановками (с роликами и без) во всех трех случаях, однако поведение псевдоскоростей и кинетической энергии отличается.

Движения \ref{sol:selfrot} одинаковы: экипаж вращается вокруг $SZ$ с постоянной скоростью, центр масс остается в покое. В случае \ref{sol:straight}, псевдоскорость $\nu_1$ -- проекция скорости центра масс на ось $\xi$, жестко связанную с платформой экипажа и совпадающую с направлением движения -- оказывается непостоянной в постановке с роликами ("скорость вращения" $\nu_3 = const$). Кроме этого, становится заметно влияние введенных предположений о смене контакта: график кинетической энергии приобретает ступенчатый вид в силу изменений в слагаемых (\ref{kin_en}), зависящих от $\chi$: 
$$B\sum_{i,j}2\dot{\theta}\sin(\kappa_j + \chi_i)\dot{\phi}_{ij},$$
происходящих при замене $\chi_i$ с $\chi^+$ на $\chi^-$ или наоборот. В промежутки времени между сменами роликов, впрочем, энергия остается постоянной. Моменты смены контакта становятся заметны и на графике $\nu_3$, но лишь в виде шума в численном решении. Движения \ref{sol:wrench} оказываются близки, но не совпадают, и вид графика кинетической энергии аналогичен случаю \ref{sol:straight}. Также, $\nu_3$ перестает быть постоянной.