%\section{Примеры движений}

{\bf 5. Примеры движений}
\stepcounter{section}
Численные решения получим для симметричного трехколесного экипажа ($\alpha_i = \frac{2\pi}{N}(i - 1), N = 3$), с $n = 5$ роликами на колесе (для случая без роликов -- с колесами соответствующей массы) и следующих движений:
\begin{enumerate}
  \item \label{sol:selfrot} вращение вокруг своей оси ($\nu_1(0) = \nu_2(0) = 0, \nu_3(0) = 1$) -- см. рис.~\ref{fig:old_selfrot},~\ref{fig:selfrot},
  \item \label{sol:straight} движение по прямой в направлении оси одного из колес ($\nu_1(0) = 1, \nu_2(0) = \nu_3(0) = 0$) -- см. рис.~\ref{fig:old_straight},~\ref{fig:straight}
  \item \label{sol:wrench} движение с ненулевой скоростью центра масс и, одновременно, с ненулевой угловой скоростью платформы (т.е. ``с закруткой'') ($\nu_1(0) = 1, \nu_2(0) = 0, \nu_3(0) = 1$) -- см. рис.~\ref{fig:old_wrench},~\ref{fig:wrench}~и~\ref{fig:wrench_short}.
\end{enumerate}

Во всех трех случаях наблюдаются отличия между двумя постановками (с роликами и без): свободные ролики приходят в движение, из-за чего меняется угловая скорость платформы экипажа и скорость центра масс экипажа.

Кроме этого, становится заметно влияние введенных предположений о смене контакта: график кинетической энергии приобретает ступенчатый вид в силу изменений в слагаемых (\ref{kin_en}), зависящих от $\chi$: 
\begin{equation}\label{sines_in_kin_en}
    B\sum_{i,j}2\dot{\theta}\sin(\kappa_j + \chi_i)\dot{\phi}_{ij},
\end{equation}
происходящих при замене $\chi_i$ с $\chi^+$ на $\chi^-$ или наоборот. В промежутки времени между сменами роликов, впрочем, энергия остается постоянной. 

В случаях \ref{sol:selfrot} и \ref{sol:straight} траектории центра экипажа $S$ на плоскости $OXY$ и характер вращения вокруг вертикальной оси $SZ$ ($\theta(t)$) отличаются несущественно, однако заметны переходные процессы во вращении роликов в начале движения.

В \ref{sol:selfrot} свободные ролики раскручиваются до значений угловых скоростей, соответствующих положению роликов на колесах, проходя по всей окружности колеса до первого входа в контакт с опорной плоскостью, см рис. (???), что заметно и на графиках  (с точностью до постоянного множителя) угловой скорости платформы $\nu_3$ и кинетической энергии. Скорость центра масс остается равной нулю. После этого процесса при каждой смене контакта теряется часть энергии, поскольку ролик, входящий в контакт, перестает совершать собственное вращение. 

В \ref{sol:straight} ролики на колесах, находящихся со стороны экипажа, противоположной направлению движения, сперва остаются неподвижными, но когда каждый побывает в контакте, их движение становится ``однородным''. Энергия также убывает с каждой сменой контакта, а раскрутка роликов между сменами происходит за счет скорости центра масс.

Движения \ref{sol:wrench} заметно отличаются между двумя постановками. При учете движения свободных роликов, экипаж описывает не окружность, как в безынерционной модели, а спираль (рис. ???), причем поступательное движение постепенно замедляется, а вращательное ускоряется. Все ролики почти постоянно вращаются вокруг своих осей, направление и скорость вращения меняется. Кинетическая энергия имеет ступенчатый вид из-за многовенного наложения связей на вновь входящие в контакт ролики.

% Траектории центра экипажа $S$ на плоскости $OXY$ и характер вращения вокруг вертикальной оси $SZ$ ($\theta(t)$) близки или совпадают между двумя постановками (с роликами и без) во всех трех случаях, однако поведение псевдоскоростей и кинетической энергии отличается.

% Движения \ref{sol:selfrot} одинаковы: экипаж вращается вокруг $SZ$ с постоянной скоростью, центр масс остается в покое. В случае \ref{sol:straight}, псевдоскорость $\nu_1$ -- проекция скорости центра масс на ось $\xi$, жестко связанную с платформой экипажа и совпадающую с направлением движения -- оказывается непостоянной в постановке с роликами ("скорость вращения" $\nu_3 = const$). Кроме этого, становится заметно влияние введенных предположений о смене контакта: график кинетической энергии приобретает ступенчатый вид в силу изменений в слагаемых (\ref{kin_en}), зависящих от $\chi$: 
% $$B\sum_{i,j}2\dot{\theta}\sin(\kappa_j + \chi_i)\dot{\phi}_{ij},$$
% происходящих при замене $\chi_i$ с $\chi^+$ на $\chi^-$ или наоборот. В промежутки времени между сменами роликов, впрочем, энергия остается постоянной. Моменты смены контакта становятся заметны и на графике $\nu_3$, но лишь в виде шума в численном решении. Движения \ref{sol:wrench} оказываются близки, но не совпадают, и вид графика кинетической энергии аналогичен случаю \ref{sol:straight}. Также, $\nu_3$ перестает быть постоянной.