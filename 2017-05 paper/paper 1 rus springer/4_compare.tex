%\section{Сравнение уравнений с уравнениями безынерционной модели}
{\bf Сравнение уравнений с уравнениями безынерционной модели.}
\stepcounter{section}

При $B=0$ структура кинетической энергии совпадает с  \cite{Zobova2011}, где масса роликов не учитывается. То же верно для лагранжиана с учетом связей:
$$ 2L^{*} = \mathring{\nu}^T \mathring{V}^T \mathring{M} \mathring{V} \mathring{\nu} + $$
$$ + B\sum_{i}(
	\frac{(\nu_2\sin\alpha_i+\nu_1\cos\alpha_i)^2R^2}
	{\rho_i^2} +
	\frac{2R\nu_3(\nu_2\sin\alpha_i+\nu_1\cos\alpha_i)\sin\chi_i}
	{\rho_i\Lambda}
) + $$
$$+ B\sum_{i,j}(
	\frac{2\nu_3\nu_{3+ni+j}\sin(\kappa_j+\chi_i)}
	{\Lambda}
	+
	\nu_{3+ni+j}^2
)
$$
где $\frac{1}{2}\mathring{\nu}^T \mathring{V}^T \mathring{M} \mathring{V} \mathring{\nu} = \mathring{L}^{*}$ -- лагранжиан системы без роликов, $\mathring{M}, \mathring{V}$ -- матрицы кинетической энергии и связей для системы без роликов:
$$
\mathring{M} = diag(M, M, I_S, J...J),
\quad
\mathring{V} = \left(\begin{matrix}
    R\cos\theta & -R\sin\theta & 0 \\
    R\sin\theta & R\cos\theta  & 0 \\
    0           & 0            & \frac{1}{\Lambda} \\
    \frac{R}{l}\sin\alpha_i & -\frac{R}{l}\cos\alpha_i & -\frac{R}{l\Lambda} \\
\end{matrix}\right),
$$
$\nu_{3+nu+j} = \nu_s$ соответствуют свободным роликам. Заметим также, что в выражениях\upr{P} для ``импульсов'' присутствуют слагаемые, пропорциональные $p_{\phi_{i1}} = Bg(\chi_{i})$ (см.\upr{p}).

Таким образом, лагранжиан и ``импульсы'' отличаются от оных в случае без роликов аддитивными членами:
$$ L^{*} = \mathring{L}^{*} + BL^{*}_\Delta(\nu, \chi),$$
$$ P_\alpha = \mathring{P_\alpha}(\theta, p_x, p_y, p_\chi) + P_\Delta(p_{\phi_i}, \chi).$$
Поэтому имеет место следующий факт.

\begin{stmt}
    Учет массы роликов приводит к появлению в правой части дифференциальных уравнений, описывающих динамику экипажа, слагаемых, пропорциональных собственному моменту инерции роликов $B$ и квадратично зависящих от псевдоскоростей. Эти новые слагаемые явно зависят от углов поворота колес $\chi_i$.
    $$\boldsymbol{A}\dot{\boldsymbol{\nu}} = \frac1\Lambda
    \left(
    \begin{array}{c}
         \nu_2\nu_3  \\
         -\nu_1\nu_3 \\
         0
    \end{array}
    \right) + B
     \left(
    \begin{array}{c}
         \boldsymbol{\nu}^T\boldsymbol{F}_1(\chi_i)\boldsymbol{\nu}  \\
         \boldsymbol{\nu}^T\boldsymbol{F}_2(\chi_i)\boldsymbol{\nu} \\
         \boldsymbol{\nu}^T\boldsymbol{F}_3(\chi_i)\boldsymbol{\nu}
    \end{array}
    \right),
    $$
    $$
    \dot{\chi}_i = \frac{R\sin\alpha_i}{l}\nu_1 - \frac{R\cos(\alpha_i)}{l}\nu_2 - \frac{R}{l\Lambda}\nu_3, \quad i = 1..N,
    $$
    $$
    \Lambda\dot{\nu}_{ni+j} = -\dot{\nu}_3\sin(\chi_i+\kappa_j) - \dot{\chi_i}\nu_3\cos(\chi_i+\kappa_j), \quad j = 2..n.
    $$
\end{stmt}

Для доказательства достаточно рассмотреть по очереди члены в лаконичной форме уравнений \ref{Tatarinov}:

$$ \frac{d}{dt}\frac{\partial }{\partial \nu_\alpha}(L^{*} - \mathring{L^{*}}) = B\frac{d}{dt}\frac{\partial}{\partial \nu_\alpha}L^{*}_\Delta(\nu, \chi), $$
$$ \{P_\alpha, L^{*}\} - \{\mathring{P}_\alpha, \mathring{L}^{*}\} = B\{ P_\alpha, L^{*}_\Delta(\nu, \chi) \} $$
$$\{P_\alpha, P_\mu\} - \{\mathring{P}_\alpha, \mathring{P}_\mu\} = B\frac{R^2}{\Lambda}\sum_i\frac{f_\alpha(\nu, \chi)}{\rho^2_i}(\frac{R}{\rho_i}(\nu_1\cos\alpha_i + \nu_2\sin\alpha_i) + \frac{\sin\chi_i}{\Lambda}\nu_3).$$
