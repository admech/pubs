\documentclass{article}

\RequirePackage{fix-cm}

\documentclass[12pt]{extarticle}

% \smartqed  % flush right qed marks, e.g. at end of proof

\usepackage{amsmath}
\usepackage{amsfonts}

% Specially for PMM: to have straight (upright) greek letters for vectors
%--------------------------------------
% \usepackage{upgreek}
% \usepackage[artemisia]{textgreek}
\usepackage[euler]{textgreek} %use upsilon instead of nu
%--------------------------------------

% Russian-specific packages
%--------------------------------------
\usepackage[T2A]{fontenc}
\usepackage[utf8]{inputenc}
\usepackage[russian]{babel}
%--------------------------------------

% Asymptote for pictures
%--------------------------------------
\usepackage{asymptote} %% comes with options inline and attach
%--------------------------------------

% graphicx for graphs
%--------------------------------------
\usepackage{graphicx}
%--------------------------------------

% so that it was possible to fix a figure's placement with [H]: \begin{figure}[H]
%--------------------------------------
\usepackage{float}
%--------------------------------------

% to put Fig.N on the margins
%-------------------------------------
\usepackage{marginnote}
\reversemarginpar
%-------------------------------------


%--------------------------------------
% Specially for PMM: make all imported EPS grayscale:
%--------------------------------------
\usepackage[gray]{epspdfconversion}
%--------------------------------------

% \usepackage{subfig} % incompatible with subcaption package
% \graphicspath{ % not used here
    % {./pic/,./asy/}
% }
%--------------------------------------

% subcaption for many figures under one big caption
% each having its own small caption
%--------------------------------------
%\usepackage{caption}
%\usepackage{subcaption}
%--------------------------------------

% so that refs were [1-10], not [1,2,3,4,5,...]
%--------------------------------------
\usepackage{cite}
%--------------------------------------

% to get \bigstar
%--------------------------------------
\usepackage{amssymb}
%--------------------------------------

% \ddfrac command to show big fractions, not cramped up
% https://tex.stackexchange.com/questions/173899/
%--------------------------------------
\newcommand\ddfrac[2]{\displaystyle\frac{\displaystyle #1}{\displaystyle #2}}
%--------------------------------------

% \vsp command to make a spacey newline
% useful for equations arrays
%--------------------------------------
\newcommand\vsp[1][10]{\\[#1pt]}
%--------------------------------------

% to customize itemize
%--------------------------------------
\usepackage{enumitem}
%--------------------------------------

% to set text width
%--------------------------------------
% \usepackage{geometry}
\usepackage{changepage}
%--------------------------------------



% partial derivatives (can \usepackage{physics}, but only one command so far, so no)
%--------------------------------------
\newcommand\pd[2]{\frac{\partial #1}{\partial #2}}
\newcommand\ddpd[2]{\ddfrac{\partial #1}{\partial #2}}
\newcommand\ddt[1]{\frac{d #1}{dt}}
\newcommand\ddddt[1]{\ddfrac{d #1}{dt}}
%--------------------------------------

% unbreakable space parenthesized reference
%--------------------------------------
\newcommand\upr[1]{~(\ref{#1})}
%--------------------------------------

% Nice letters
%--------------------------------------
\newcommand\M[0]{\mathcal{M}} % Matrix of intertia
\newcommand\const{\mathrm{const}} %константа
\newcommand\AntiU[0]{\mathcal{U}} % Helper antisymmetric matrix for eqs' RHS
\newcommand\Rhs[0]{\mathcal{R}} % RHS
\newcommand\Prhs[0]{\mathcal{P}} % The family of matrices for RHS
\newcommand\prhs[0]{\mathbf{p}} % Poisson brackets
\newcommand\vnu[0]{\text{\textbf{\textupsilon}}} % Upright greek vector nu for PMM
%--------------------------------------

% change line spacing mid doc (affects global line spacing)
%--------------------------------------
% \usepackage{setspace}
%--------------------------------------


\renewcommand{\vec}[1]{\boldsymbol{\mathbf{#1}}}
%\renewcommand{\figurename}{Фиг.}
\usepackage[labelsep=period]{caption}
\addto\captionsrussian{\renewcommand{\figurename}{Фиг. }}

\newtheorem{stmt}{Утверждение}
\newtheorem{prblm}{Затруднение}

% biblio hacks -- noindent bibitems
\makeatletter
\renewenvironment{thebibliography}[1]
      {\section*{\refname}%
      \@mkboth{\MakeUppercase\refname}{\MakeUppercase\refname}%
      \list{\@biblabel{\@arabic\c@enumiv}}%
            {\settowidth\labelwidth{\@biblabel{#1}}%
             \leftmargin\labelwidth
             \advance\leftmargin-25pt% change 20 pt according to your needs
             \advance\leftmargin\labelsep
             \setlength\itemindent{25pt}% change using the inverse of the length used before
             \@openbib@code
             \usecounter{enumiv}%
             \let\p@enumiv\@empty
             \renewcommand\theenumiv{\@arabic\c@enumiv}}%
      \sloppy
      \clubpenalty4000
      \@clubpenalty \clubpenalty
      \widowpenalty4000%
      \sfcode`\.\@m}
      {\def\@noitemerr
        {\@latex@warning{Empty `thebibliography' environment}}%
      \endlist}
\renewcommand\newblock{\hskip .11em\@plus.33em\@minus.07em}
\makeatother


\voffset=-15mm \textwidth=17cm \textheight=24cm
\oddsidemargin=0cm \topmargin=+0cm \headsep=10pt \evensidemargin=0mm
\renewcommand{\baselinestretch}{2}

\makeatletter \@addtoreset{equation}{section} \makeatother
\makeatletter

% bibliography hacks
\renewcommand{\@biblabel}[1]{#1. \hfill}
\makeatother
\addto\captionsrussian{\def\refname{Литература}}
\renewcommand{\refname}{}

\renewcommand{\thesection}{\arabic{section}}
\renewcommand{\theequation}{\arabic{section}.\arabic{equation}}



\begin{document}


% \begin{figure}
%         \asyinclude{./asy/pic_cart.asy}
%         \asyinclude{./asy/pic_wheel.asy}
%         \asyinclude{./asy/pic_overlap.asy}
%         \asyinclude{./asy/pic_change.asy}
% \end{figure}

\begin{figure}[H]
    \hspace{-0.6cm}
    % \minipage{0.5\textwidth}
        % \centering
        \scalebox{1.5}{\asyinclude{./asy/pic_wheel.asy}}
        %\caption{Колесо}
        % \caption{\ }
        % \label{fig:wheel}
    % \endminipage
    \quad
    % \minipage{0.55\textwidth}
        % \centering
        \scalebox{1.5}{\asyinclude{./asy/pic_cart.asy}}
        %\caption{Экипаж}
        \caption{\ }
        % \label{fig:vehicle}
        \label{fig:wheel}
    % \endminipage
\end{figure}

\begin{figure}[H]
    \centering
    \scalebox{1.5}{\asyinclude{./asy/pic_overlap.asy}}
    \quad
    \scalebox{1.5}{\asyinclude{./asy/pic_change.asy}}
    \caption{\ }
    \label{fig:overlap_and_change}
\end{figure}

% $\ddfrac{d}{dt}\pd{L^{*}}{\nu_\alpha}, \{P_\alpha, L^{*}\}, \{P_\alpha, \nu_\mu P_\mu\}$
% $$
%     \ddddt{}\ddpd{L^*}{\nu_\alpha} = M^*_{\alpha\beta}\dot{\nu}^\beta + \ddpd{M^*_{\alpha\beta}}{\chi_\gamma}V^\gamma_\xi\nu^\xi\nu^\beta,
% $$
% $$
%     \begin{array}{rcl}
%         \ddddt{}\ddpd{L^*}{\nu_1} & = & B\sum\limits_{i}\bigg(\bigg),\vsp
%         \ddddt{}\ddpd{L^*}{\nu_2} & = & ,\vsp
%         \ddddt{}\ddpd{L^*}{\nu_3} & = & ,\vsp
%         \ddddt{}\ddpd{L^*}{\nu_s} & = & ,\vsp
%     \end{array}
% $$

% В наших уравнениях

% \begin{equation}
%     \frac{d}{dt}\frac{\partial L^{*}}{\partial \nu_\alpha}  + \{P_\alpha, L^{*}\} = \{P_\alpha, \nu_\mu P_\mu\}
% \end{equation}

% интерес представляет первое слагаемое, в котором:

% \begin{equation}
%     L^{*} = \frac{1}{2}\vec{\nu}^\mathrm{T} \M^*(\chi_i)\vec{\nu},
% \end{equation}

% что при дифференцировании даёт:

% \begin{equation}
%     \frac{d}{dt}\frac{\partial L^{*}}{\partial \nu_\alpha} = \frac{d}{dt}(\M^*(\chi_i)\vec{\nu}_\alpha) = 
%     \M^*(\chi_i)\dot{\vec{\nu}_\alpha} +
%     \color{blue}
%     \left(\frac{d}{dt}(\M^*(\chi_i))\vec{\nu}\right)_\alpha
%     \color{black},
% \end{equation}

% где для взятия производной в последнем слагаемом нужно воспользоваться связями $\dot{\chi_i} = (V\nu)_{3+i}$:

% \begin{equation}
%     \ddfrac{d}{dt}\M^*(\chi_i)\vec{\nu} = \sum_{i=1}^{N}\M^*_i(V\nu)_{3+i}\vec{\nu},
% \end{equation}

% что для $\nu_{s}$ равно:

% \begin{equation}
%     \frac{\cos\left( \chi_i+\kappa_j\right) \nu_3 B \left( -\ddfrac{\nu_3 R}{l \Lambda}-\ddfrac{\cos\alpha_i \nu_2 R}{l}+\ddfrac{\sin\alpha_i \nu_1 R}{l}\right) }{\Lambda},
% \end{equation}

% где $i,j$ соответствуют $s$. \\

% Эта штука оказывается в правой части (с противоположным знаком) и вероломно раскручивает нам ролики! \\

% Для $\nu_{1\ldots3}$, конечно, тоже свои слагаемые. Допишу это в структуру уравнений, стало быть, и выводы наши, наверно, немного уточнятся.

\end{document}



