
\subsection{Изменение кинетической энергии}
Выясним, как меняется кинетическая энергия при смене ролика в контакте:
$$
2\Delta\ke  =  2\left(\ke^+ - \ke^-\right) = \dotp{\mke\dqposle}{\dqposle} - \dotp{\mke\dqdo}{\dqdo} =
$$
$$
= \dotp{\mke\deltadq}{\deltadq} + 2\dotp{\mke\dqdo}{\deltadq} = -\dotp{\mke\deltadq}{\deltadq} + 2\dotp{\mke\dqposle}{\deltadq}
$$
Последнее слагаемое равно нулю в силу идеальности связей\upr{eq:constraints_ideal} и основного уравнения удара\upr{eq:udar_general}, т.е. равенства нулю мощности ударных импульсов на перемещениях, допускаемых связями
$$\logicWorkZero.$$
% (заметим: $\edMkeSim$) или, что то же, ортогональности в кинетической метрике вектора потерянных скоростей $\deltadq$ подпространству $\subspace$ пространства возможных перемещений, определяемому связями.
Таким образом, потеря кинетической энергии системы равна энергии потерянных скоростей $\deltadq = \dqposle - \dqdo$:
\begin{equation*}
\eqDeltaT,
\end{equation*}
что соответствует теореме Карно \cite{Vilke}.
