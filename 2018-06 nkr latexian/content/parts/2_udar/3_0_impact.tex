\section{Наложение связи при смене ролика в контакте}

В реальной системе при смене контакта имеет место скольжение роликов относительно плоскости, и при этом полная энергия системы рассеивается. В данной работе будем считать, что трение достаточно велико, и прекращение проскальзывания вновь вошедшего в контакт ролика происходит мгновенно. Это взаимодействие будет рассматриваться как абсолютно неупругий удар, происходящий при мгновенном наложении связи.
Освободившийся ролик начинает свободно вращаться вокруг своей оси.
Будем предполагать следующее:
\begin{itemize}
    \item удар происходит за бесконечно малый интервал времени $\Delta t << 1$, так что изменения обобщенных координат пренебрежимо малы $\Delta \q \sim \dotq\Delta t << 1$, а изменения обобщенных скоростей конечны $\Delta \dotq < \infty$;
    \item взаимодействие экипажа с опорной полоскостью во время удара сводится к действию в точках контакта нормальных и касательных реакций $\mathbf{R}_i = \mathbf{N}_i + \mathbf{F}_i$ с нулевым моментом $\mathbf{M}_i = 0$;
    \item к моменту окончания удара $t^*+\Delta t$ уравнения связей выполнены $\dqposle = \cstr(\q)\nuposle$, т.е. за время $\Delta t$ проскальзывание вошедшего в контакт ролика заканчивается.
\end{itemize}

% Таким образом, решение задачи представляется в виде чередования гладких участков, определяемых уравнениями движения, и пересчетов значений обобщенных скоростей в моменты смен контакта.

Исходя из этих предположений, в следующих разделах получим системы алгебраических уравнений, связывающих значения обобщенных скоростей непосредственно перед ударом $\dqdo$ и значения псевдоскоростей сразу после удара $\nuposle$ двумя разными способами: в первом случае, будем  вводить ударные реакции, действующие в точках контакта, а во втором, будем рассматривать неупругий удар как проецирование вектора обобщенных скоростей на плоскость, задаваемую уравнениями вновь налагаемых связей.

Таким образом, моделирование системы состоит в решении задачи Коши системы обыкновенных дифференциальных уравнений в интервалах между моментами смены роликов в контактах и решения систем алгебраических уравнений в эти моменты для получения начальных условий для следующего безударного интервала.

\begin{figure}[h]
    \minipage{0.5\textwidth}
        \centering
        \asyinclude{content/pic/asy/pic_react}
        \caption{Компоненты векторов ударных реакций в точках контакта}
        \label{fig:react}
    \endminipage
    \minipage{0.5\textwidth}
        \centering
        \asyinclude{content/pic/asy/pic_project}
        \caption{Проецирование вектора обобщенных скоростей на ядро дифференциальной формы связей}
        \label{fig:project}
    \endminipage
\end{figure}
