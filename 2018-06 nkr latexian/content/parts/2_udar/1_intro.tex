\chapter{Смена ролика в контакте с опорной плоскостью}

При рассмотрении динамики роликов отдельного внимания заслуживает момент перехода колеса с одного ролика на другой, поскольку вращение ролика, входящего в контакт, может не быть согласовано с условием отсутствия скольжения в контакте.

\begin{figure}
    \minipage{\textwidth}
        \centering
        \asyinclude{./content/pic/asy/pic_overlap.asy}
        \caption{Перекрытие}
        \label{fig:overlap}
    \endminipage
\end{figure}

В тех интервалах времени, когда ролик в контакте с опорной плоскостью не меняется, динамика системы описывается уравнениями движения системы (см. \cite{GerasimovZobovaPMM2018}). Смена контакта на $i$-том колесе происходит при значении угла $\chi_i = \ddfrac{2\pi}{n}$. При этом, во-первых, правая часть уравнений движения терпит разрыв второго рода из-за равенства нулю выражений $\rho_i = l\cos\chi_i-r$ в знаменателе. Во-вторых, происходит мгновенное снятие и наложение связей: условие отсутствия проскальзывания для ролика, выходящего из контакта, снимается, и аналогичное ему мгновенно налагается на вновь входящий в контакт ролик.

На практике первое обстоятельство никогда не реализуется, поскольку оси роликов в реальных системах имеют ненулевую толщину, а значит, концы роликов усекаются. При этом ролики располагают рядами в двух и более плоскостях, чтобы в каждый момент гладкая сторона хотя бы одного ролика была в контакте с плоскостью. В данной работе рассматриваются усеченные ролики (см. фиг.~\ref{fig:overlap}), но их оси расположены в одной плоскости, и допускается пересечение тел роликов в пространстве. Ось ролика находится на расстоянии $r = l\cos\ddfrac{\pi}{n-1}$ от центра колеса. Ролик представляет собой тело вращения относительно этой оси дуги окружности радиуса $l$ с углом раствора $\ddfrac{2\pi}{n}$.

В данной главе проведено детальное рассмотрение момента смены ролика в контакте с учетом ударного характера взаимодействия с опорной плоскостью. Также, получены численные решения, состоящие из участков, определяемых уравнениями движения, и моментов смены контакта, моделируемых с точки зрения теории удара.
