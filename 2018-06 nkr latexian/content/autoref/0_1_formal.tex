%\section{Коротко о главном}

\section*{Общая характеристика работы}

\textbf{Актуальность темы}

Роликонесущие колеса – это колеса особой конструкции, позволяющей экипажу двигаться в произвольном направлении, не поворачиваясь вокруг вертикальной оси. Колеса при этом вращаются лишь вокруг их собственных осей и не поворачиваются вокруг вертикали. Экипажи с такими колесами обладают повышенной маневренностью и используются, например, в качестве погрузочных платформ в авиастроении.
Ранее была рассмотрена динамика роликонесущих экипажей с использованием упрощенной модели омни-колес без учета инерции и формы роликов (безынерционная модель). В этой модели колеса  представляют собой жесткие диски, которые могут скользить в одном направлении и катиться без проскальзывания в другом. Динамические эффекты, порождаемые вращением роликов, не учитываются. В другой части работ по динамике роликонесущего экипажа используются  алгоритмы для построения численных моделей систем твердых тел. При этом явный вид уравнений движения оказывается скрытым, что делает невозможным непосредственный анализ уравнений и затрудняет оценку влияния разных факторов на динамику системы. В этом случае авторами также применяются существенные упрощения геометрии роликов, либо их динамика также не учитывается.
В настоящей работе динамические эффекты, возникающие из-за собственного вращения роликов, учитываются, и форма роликов приближена к применяемой на практике.

\textbf{Цель работы}

Целью работы является изучение неуправляемого движения роликонесущего экипажа по горизонтальной плоскости с учетом инерции роликов и трения в двух постановках. В первой постановке опорная плоскость абсолютно шероховата, т.е. проскальзывание между роликом в контакте и плоскостью отсутствует. При этом предполагается, что при смене ролика в контакте происходит мгновенное согласование скоростей системы в соответствии с новыми связями (удар связями). Во второй между контактным роликом и опорной плоскостью действует сила сухого трения Кулона. 

\textbf{Методы исследования}

Для получения уравнения движения экипажа по абсолютно шероховатой плоскости используется метод, предложенный Я.В. Татариновым. Рассмотрение перехода колеса с одного ролика на другой проведено с точки зрения теории удара. Решения составленной системы уравнений получены для случая трех колес и пяти роликов на каждом с помощью системы компьютерной алгебры \texit{Maxima}. При построении модели экипажа на плоскости с сухим трением кинематика вращательного движения твердых тел описывается в алгебре кватернионов. Сила сухого трения Кулона в точечном твердотельном контакте роликов и опорной плоскости регуляризуется, основываясь на теории И.В. Новожилова. Компьютерная модель экипажа с трением построена с помощью технологии объектно-ориентированного моделирования \textit{Modelica}. СТОИТ ЛИ О СЛЕДУЮЩЕМ ??? Верификация данной модели в сравнении с безынерционной моделью проведена, опираясь на теорию А.В. Карапетяна о близости решений неголономных систем и систем с достаточно сильным вязким трением.

\textbf{Достоверность и обоснованность результатов}

КАК В ЗАКЛЮЧЕНИИ КАФЕДРЫ

Все основные результаты первой и второй главы диссертации получены строгими аналитическими методами и базируются на основных положениях механики систем тел и теории удара. Решения уравнений движения из первой и второй главы, а также результаты третьей главы получены при помощи численных методов.

ЛИБО

Основные результаты глав 1 и 2 получены аналитически с помощью методов Я.В. Татаринова, теории удара, аналитической механики и линейной алгебры. Построение компьютерной модели в главе 3 основано на теориях И.В. Новожилова, НАДО ЛИ ??? А.В. Карапетяна и Синьорини. Численные решения верифицируются с использованием хорошо изученных аналитических моделей, либо свойств моделируемых систем, выявленных аналитически. Аналитические результаты подтверждены и проиллюстрированы с помощью численного анализа.

\textbf{Научная новизна}

Все основные результаты, полученные в диссертационной работе, являются новыми. Впервые получены уравнения движения экипажа на омни-колесах по абсолютно шероховатой плоскости с учетом инерции всех роликов. Построен способ расчета изменения обобщенных скоростей при смене ролика в контакте согласно теории удара. Проведено численное моделирование движений экипажа. Построена динамическая модель экипажа на плоскости с регуляризованным сухим трением с учетом геометрии роликов, приближенной к применяемой на практике, показаны отличия движений этой модели от движений безынерционной модели.

\textbf{Положения, выдвинутые на защиту}

\begin{itemize}

	\item Получены уравнения движения экипажа на омни-колесах по абсолютно шероховатой плоскости с учетом инерции роликов.

	\item Изучены свойства этих уравнений движения и проведено сравнение их с уравнениями движения безынерционной модели.

	\item Построен способ расчета изменения обобщенных скоростей при смене ролика в контакте, в предположении о мгновенном выполнении условия отсутствия проскальзывания между роликом и опорной плоскостью.

	\item Получены численные решения для симметричной конфигурации экипажа с омни-колесами с учетом ударного взаимодействия роликов и опорной плоскости.

	\item Построена динамическая модель экипажа с омни-колесами на плоскости с регуляризованным сухим трением. Показана возможность гладкого безударного переключения роликов в контакте в процессе качения/скольжения омни-колеса. Модель описывает оба варианта омни-колес: обыкновенные, с осями роликов в плоскости колеса, и \textit{mecanum}, где оси роликов повернуты вокруг радиус-векторов их центров.

	\item Выполнена верификация динамической модели омни-экипажа с использованием безынерционной модели в качестве предельного случая (когда суммарная масса роликов равна нулю).
\end{itemize}

\textbf{Научная и практическая значимость}

Диссертация носит теоретический характер. Результаты диссертации могут найти применения при проведении исследований в МГУ имени М.В. Ломоносова, Институте проблем механики имени А.Ю. Ишлинского РАН, Институте прикладной математики имени М.В. Келдыша РАН и научно -- исследовательских центрах, занимающихся проектированием и исследованием колесных систем различного назначения.

\textbf{Апробация работы}

По результатам работы опубликованы и приняты к печати в рецензируемых журналах, реферируемых в международных базах WebOfScience, Scopus, RSCI и входящих в список ВАК, следующие статьи: 

\begin{enumerate}
	\item Герасимов К.В., Зобова А.А. Движение симметричного экипажа на омни-колесах с массивными роликами // ПММ. 2018. Т. 82, No 4. прин. к печ.
	\item Герасимов К.В., Зобова А.А. Динамика экипажа на омни-колесах с массивными роликами с учетом смены ролика в контакте с опорной плоскостью // Труды МАИ. Сент. 2018. прин. к печ.
	\item Косенко И.И., Герасимов К.В. Физически-ориентированное моделирование динамики омнитележки // Нелин. дин. 2016. Т. 12, No 2. С. 251–262.
	\item Kosenko I.I., Stepanov S.Y., Gerasimov K.V. Improved contact tracking algorithm for the omni wheel in general case of roller orientation // The Proceedings of the Asian Conference on Multibody Dynamics. 2016.8. The Japan Society of Mechanical Engineers. — 2017. — no. July 01 — P. 2424-2985.
	\item Kosenko I.I., Gerasimov K.V. Object-oriented approach to the construction of an omni vehicle dynamical model // Journal of Mechanical Science and Technology. — 2015. — Vol. 29, no. 7. — P. 2593–2599.
\end{enumerate}

Результаты докладывались автором на ряде международных и всероссийских конференций:

\begin{enumerate}
	\item Международная конференция по дифференциальным уравнениям и динамическим системам 2018, Суздаль, Россия, 6-11 июля 2018
	\item 20-е Международное рабочее совещание по компьютерной алгебре (Дубна, 21-22 мая 2018), Дубна, Россия, 21-22 мая 2018
	\item 20-е Международное рабочее совещание по компьютерной алгебре (Дубна, 21-22 мая 2018), Дубна, Россия, 21-22 мая 2018
	\item Ломоносовские Чтения - 2018, МГУ имени М.В. Ломоносова, Россия, 16-25 апреля 2018
	\item Ломоносовские Чтения - 2017, МГУ имени М.В. Ломоносова, Россия, 17-26 апреля 2017
	\item 11th International Modelica Conference Versailles, France, September 21-23, 2015
\end{enumerate}

Результаты также были представлены автором на семинаре Аналитическая механика и теория устойчивости (имени В.В. Румянцева) под руководством д.ф.-м.н. проф. А.В. Карапетяна в 2017 и 2018 г. и на семинаре Динамика относительного движения под руководством д.ф.-м.н. проф. В.Е. Павловского на механико-математическом факультете МГУ им. М.В. Ломоносова в 2018 г.

\textbf{Личный вклад}

Научные руководители предложили постановку всех задач и указали методы их исследования. Все представленные в диссертации результаты получены лично соискателем. 

\textbf{Структура и объем работы}

Диссертационная работа состоит из введения, трех глав, заключения и списка литературы. Полный объем диссертации КХМ! страниц текста с КХЕ-КХЕ! рисунками. Список литературы содержит АХХМ! наименований.

\section*{Содержание работы}

Во \textbf{введении} приведен обзор литературы, обоснована актуальность решаемых задач, дан обзор литературы по предметной области и применяемым методам, приведено краткое содержание работы, даны ссылки на статьи, опубликованные по результатам работы и названы семинары и конференции, где результаты были представлены.

