
\section{Постановка задачи}

\begin{figure}[h]
    \minipage{0.3\textwidth}
        \centering
        \asyinclude{./asy/pic_wheel.asy}
        \caption{Колесо}
        \label{fig:wheel}
    \endminipage
    \minipage{0.4\textwidth}
        \centering
        \asyinclude{./asy/pic_cart.asy}
        \caption{Экипаж}
        \label{fig:vehicle}
    \endminipage
    \minipage{0.3\textwidth}
        \centering
        \asyinclude{./asy/pic_overlap.asy}
        \caption{Перекрытие}
        \label{fig:overlap}
    \endminipage
\end{figure}

( Копия из odno-koleso-udar ) \newline
Рассмотрим движение одного свободного омниколеса по абсолютно шероховатой плоскости (всего экипажа нет). На колесо действует сила тяжести, а также реакция связи, приложенная в наинизшей точке колеса.

Примем следующие предположения:
\begin{itemize}
\item Время удара $\Delta t$ мало настолько, что за время удара координаты системы не меняются. При этом скорости меняются на конечную величину.
\item Изменение скоростей происходит вследствие действия реакции связи, сводящейся к нормальной реакции и силе трения, действующей на входящий в контакт ролик. Будем считать, что момент сил трения относительно граничной точки входящего в контакт ролика равен нулю. Сила трения имеет обе горизонтальные компоненты -- вдоль плоскости колеса и перпендикулярно ей.
\item Все время движения колеса на его диск действует некоторый горизонтальный момент, сохраняющий плоскость диска колеса вертикальной.
\item Ролик усечен, так что вектор, идущий из центра ролика в точку контакта, не коллинеарен оси ролика (это необходимо, чтобы сила трения смогла раскрутить ролик до нужной нам скорости)
\item Трения в оси ролика нет
\item Трения в оси колеса нет
\item За время $\Delta t$ угловая скорость вторго ролика и скорость собственного вращения колеса меняются так, что проскальзывание в точке контакта исчезает (то есть в момент окончания удара становятся выполнеными дифференциальные уравнения связи. В течение удара $[0,\Delta t)$ они нарушены).
\end{itemize}


