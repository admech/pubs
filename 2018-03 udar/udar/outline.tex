% ------------------------------------------------------
\section{Введение}

Известны модели либо неголономные без роликов, либо с роликами и трением, но без уравнений.
Построим неголономную модель с роликами и понятными уравнениями, но и с реалистичной сменой контакта.

% ------------------------------------------------------
\section{Постановка задачи}

\begin{itemize}
    \item рисунки, обозначения и предположения
    \item уравнения движения
    \item усечение и перекрытие роликов
    \item снятие связей перед сменой контакта
    \item эффект рамки и ожидания об убывании энергии
\end{itemize}

% ------------------------------------------------------
\section{Удар}

% ------------------------------------------------------
\subsection{Способ 1. Классическая механика}

Мотивация -- получить величины реакций.

\begin{itemize}
    \item введение реакций в контакте, рисунки
    \item составление линейной системы, ранг
    \item решение системы
\end{itemize}

% ------------------------------------------------------
\subsection{Способ 2. Аналитическая механика}

Мотивация -- строго определяется матрицами кинетической энергии и связей, не требует произвольного введения векторов реакций.

\begin{itemize}
    \item постановка задачи теории удара как проецирования вектора обобщенных скоростей на плоскость в пространстве виртуальных перемещений, определяемую вновь налагаемыми связями
    \item выражение для скоростей после удара
\end{itemize}

% ------------------------------------------------------
\subsection{Изменение кинетической энергии}

\begin{itemize}
    \item проверка, что оба способа дают один результат
    \item проверка, что результат соответствует теореме Карно (потеря энергии равна энергии потерянных скоростей)
\end{itemize}

% ------------------------------------------------------
\section{Примеры движений}

Замечание о перенумеровании роликов.
\begin{itemize}
    \item вокруг себя
    \item по прямой
    \item с закруткой
\end{itemize}

Обратить внимание на монотонное убывание кинетической энергии.

% ------------------------------------------------------
\section{Бонус: реакции}

Получить, в каких конусах остаются реакции и сравнить с конусами трения Кулона.

% ------------------------------------------------------
\section{Бонус: время движения системы}

Энергия всякий раз убывает на ненулевую конечную величину из-за эффекта рамки. Оценить время движения или количество ``шагов'', исходя из значений этой величины.
