
\section{Примеры движений}

Замечание о перенумеровании роликов.
\begin{itemize}
    \item вокруг себя
    \item по прямой
    \item с закруткой
\end{itemize}

Обратить внимание на монотонное убывание кинетической энергии.

Численные решения на участках между сменами контакта получаются интегрированием уравнений движения.

В момент смены новые значения псевдоскоростей $\nuposle$ вычисляются в соответствии с предыдущими разделами.

При смене контакта система переходит в новую конфигурацию, в которой в контакте находится ролик со следующим по порядку номером, и систему уравнений движения необходимо поменять соответственно. Однако в силу геометрической симметрии колес, эта конфигурация отличается от предыдущей только нумерацией роликов, значениями углов поворота колес $\chi_i$ и обозначениями улов поворота роликов $\phi_{ij}$.

Вычисления в момент смены контакта организуем следующим образом.

Сперва перенумеруем ролики циклически так, чтобы ролик в контакте имел номер $1$, и заменим значения углов поворота колес следующим образом: $ADJUST!$.

На этом этапе получается система, находящаяся в ``запрещенной'' области: нижний ролик может проскальзывать.

Затем восстановим значения обобщенных скоростей $\dqdo$ по псевдоскоростям $\nudo$ и решим задачу теории удара, получая новый значения псевдоскоростей $\nuposle$.

После этого система удовлетворяет вновь наложенным неголономным связям, и выполняется решение уравнений движения на следующем гладком участке.