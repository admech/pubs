
\subsection{Способ 2. Аналитическая механика}
Мгновенное наложение связи можно рассматривать как абсолютно неупругий удар: сразу после окончания удара обобщенные скорости согласованы с вновь налагаемыми связями. Получим  значения псевдоскоростей $\nuposle$ после удара, не вводя в рассмотрение ударные импульсы $\mathbf{Q}$.

Cвязи, налагаемые в момент смены контакта, задают подпространство $\subspace$ в пространстве виртуальных перемещений, и вектор псевдоскоростей после удара $\nuposle$ является проекцией  вектора обобщенных скоростей до удара $\dqdo$ на подпространство $\subspace$, разложенной по базису, состоящему из столбцов $\cstr$ (см. фиг.\upr{fig:project}). В силу предположения об идеальности налагаемых связей, при проецировании теряется только компонента $\deltadq$ вектора обобщенных скоростей, ортогональная подпространству $\subspace$ в метрике, задаваемой матрицей кинетической энергии $\mke$. Таким образом, для вектора обобщенных скоростей после удара верно 
$$\dqposle = \dqdo - \deltadq = \mathbf{V}\nuposle \in \subspace$$ Пользуясь этим, выразим $\nuposle$ из условия ортогональности $\deltadq$ и $\subspace$:
% $\eqDelta$
\begin{equation*}
\logicDeltaOrth,
\end{equation*}
откуда

\begin{equation*}
\eqnuposleproj,
\end{equation*}
где матрица $\cstr^T\mke\cstr$ обратима как матрица Грама набора линейно-независимых столбцов матрицы $\cstr$ в метрике $\mke$.

Таким образом, для получения значений обобщенных скоростей после удара $\dqposle$ не требуется вводить реакции в точках контакта и находить их.