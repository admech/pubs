
\subsection{Способ 2. Аналитическая механика}

Мотивация -- строго определяется матрицами кинетической энергии и связей, не требует произвольного введения векторов реакций.

\begin{itemize}
    \item постановка задачи теории удара как проецирования вектора обобщенных скоростей на плоскость в пространстве виртуальных перемещений, определяемую вновь налагаемыми связями
    \item выражение для скоростей после удара
\end{itemize}

Связи, налагаемые в момент смены контакта, задают подпространство $\subspace$ в пространстве виртуальных перемещений.

Вектор обобщенных скоростей до удара $\dqdo$ может иметь составляющую $\deltadq$, ортогональную этому подпространству (в кинетической метрике). (А ПОЧЕМУ ИМЕННО В КИНЕТИЧЕСКОЙ?! КАК-ТО КРАСИВО НАДО СКАЗАТЬ)

В силу идеальности налагаемых связей, значения обобщенных скоростей после удара можно получить вычитанием этой компоненты из $\dqdo$ или, что то же, проецированием $\dqdo$ на подпространство $\subspace$.

Найдем вектор $\deltadq$, ортогональный $\subspace$ в метрике, задаваемой матрицей кинетической энергии $\mke$, такой что $\dqposle = \dqdo - \deltadq = \mathbf{V}\nuposle \in \subspace$

\begin{equation*}
\eqDelta
\end{equation*}
\begin{equation*}
\logicDeltaOrth
\end{equation*}

откуда
\begin{equation*}
\eqnuposleproj
\end{equation*}
где матрица $\cstr^T\mke\cstr$ обратима, поскольку имеет максимальный ранг.