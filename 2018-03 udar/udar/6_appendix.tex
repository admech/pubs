\section{Приложение}

Покажем невырожденность матрицы $(\mke\cstr; -\A^T)$ основного уравнения удара\upr{eq:udar_mat_At}.
С одной стороны, каждая из систем столбцов $\{\cstr_{\cdot i}\}$ и $\{\A^T_{\cdot j}\}$ линейно-независима,
а с другой, эти две системы взаимно ортогональны\upr{eq:constraints_orth}.
Поскольку матрица $\mke$ положительно определена, ее действие на $\cstr$ не нарушает линейной независимости системы столбцов в целом.
Последнее можно показать это можно от противного.
Предположим существование таких чисел $x_i$ и $y_j$, не равных нулю одновременно, что $\sum_i x_i \mke \cstr_{\cdot i} + \sum_j y_j \J^T_{\cdot j} = 0$.
Домножим это равенство скалярно на каждый из столбцов $\{\cstr_{\cdot \alpha}\}$ и рассмотрим сумму полученных выражений: $\sum_{\alpha, i} x_i \cstr_{\cdot \alpha}^T \mke \cstr_{\cdot i} + \sum_{\alpha, j} y_j \cstr_{\cdot \alpha}^T \mke \J^T_{\cdot j} = \cstr^T \mke \cstr \mathbf{x} + \cstr^T\J^T\mathbf{y}$.
Последнее слагаемое здесь равно нулю в силу\upr{eq:constraints_orth},
а первое отлично от нуля, поскольку матрица $\cstr^T \mke \cstr$ является матрицей Грама линейно-независимой системы столбцов $\{\cstr_{\cdot i}\}$ в метрике $\mke$, а значит, невырожденна.
Таким образом, все выражение не может быть равным нулю.
Полученное противоречие завершает доказательство.