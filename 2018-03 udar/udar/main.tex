% \smartqed  % flush right qed marks, e.g. at end of proof

\usepackage{amsmath}
\usepackage{amsfonts}

% Russian-specific packages
%--------------------------------------
\usepackage[T2A]{fontenc}
\usepackage[utf8]{inputenc}
\usepackage[russian]{babel}
%--------------------------------------

% Asymptote for pictures
%--------------------------------------
\usepackage{asymptote} %% comes with options inline and attach
%--------------------------------------

% graphicx for graphs
%--------------------------------------
\usepackage{graphicx}

%--------------------------------------
% Specially for PMM: make all imported EPS grayscale:
%--------------------------------------
\usepackage[gray]{epspdfconversion}
%--------------------------------------

% \usepackage{subfig} % incompatible with subcaption package
% \graphicspath{ % not used here
    % {./pic/,./asy/}
% }
%--------------------------------------

% subcaption for many figures under one big caption
% each having its own small caption
%--------------------------------------
\usepackage{caption}
\usepackage{subcaption}
%--------------------------------------

% so that refs were [1-10], not [1,2,3,4,5,...]
%--------------------------------------
\usepackage{cite}
%--------------------------------------

% \ddfrac command to show big fractions, not cramped up
% https://tex.stackexchange.com/questions/173899/
%--------------------------------------
\newcommand\ddfrac[2]{\displaystyle\frac{\displaystyle #1}{\displaystyle #2}}
%--------------------------------------

% \vsp command to make a spacey newline
% useful for equations arrays
%--------------------------------------
\newcommand\vsp[1][10]{\\[#1pt]}
%--------------------------------------

% partial derivatives (can \usepackage{physics}, but only one command so far, so no)
%--------------------------------------
\newcommand\pd[2]{\frac{\partial #1}{\partial #2}}
\newcommand\ddpd[2]{\ddfrac{\partial #1}{\partial #2}}
\newcommand\ddt[1]{\frac{d #1}{dt}}
\newcommand\ddddt[1]{\ddfrac{d #1}{dt}}
%--------------------------------------

% unbreakable space parenthesized reference
%--------------------------------------
\newcommand\upr[1]{~(\ref{#1})}
%--------------------------------------

% Nice letters
%--------------------------------------
\newcommand\M[0]{\mathcal{M}} % Matrix of intertia
\newcommand\AntiU[0]{\mathcal{U}} % Helper antisymmetric matrix for eqs' RHS
\newcommand\Rhs[0]{\mathcal{R}} % RHS
\newcommand\Prhs[0]{\mathcal{P}} % The family of matrices for RHS
\newcommand\prhs[0]{\mathbf{p}} % Poisson brackets
%--------------------------------------

\renewcommand{\vec}[1]{\boldsymbol{\mathbf{#1}}}

\newtheorem{stmt}{Утверждение}
\newtheorem{prblm}{Затруднение}

\begin{document}

{\huge Структура статьи}
% ------------------------------------------------------
\section{Введение}

Известны модели либо неголономные без роликов, либо с роликами и трением, но без уравнений.
Построим неголономную модель с роликами и понятными уравнениями, но и с реалистичной сменой контакта.

% ------------------------------------------------------
\section{Постановка задачи}

\begin{itemize}
    \item рисунки, обозначения и предположения
    \item уравнения движения
    \item усечение и перекрытие роликов
    \item снятие связей перед сменой контакта
    \item эффект рамки и ожидания об убывании энергии
\end{itemize}

% ------------------------------------------------------
\section{Удар}

% ------------------------------------------------------
\subsection{Способ 1. Классическая механика}

Мотивация -- получить величины реакций.

\begin{itemize}
    \item введение реакций в контакте, рисунки
    \item составление линейной системы, ранг
    \item решение системы
\end{itemize}

% ------------------------------------------------------
\subsection{Способ 2. Аналитическая механика}

Мотивация -- строго определяется матрицами кинетической энергии и связей, не требует произвольного введения векторов реакций.

\begin{itemize}
    \item постановка задачи теории удара как проецирования вектора обобщенных скоростей на плоскость в пространстве виртуальных перемещений, определяемую вновь налагаемыми связями
    \item выражение для скоростей после удара
\end{itemize}

% ------------------------------------------------------
\subsection{Изменение кинетической энергии}

\begin{itemize}
    \item проверка, что оба способа дают один результат
    \item проверка, что результат соответствует теореме Карно (потеря энергии равна энергии потерянных скоростей)
\end{itemize}

% ------------------------------------------------------
\section{Примеры движений}

Замечание о перенумеровании роликов.
\begin{itemize}
    \item вокруг себя
    \item по прямой
    \item с закруткой
\end{itemize}

Обратить внимание на монотонное убывание кинетической энергии.

% ------------------------------------------------------
\section{Бонус: реакции}

Получить, в каких конусах остаются реакции и сравнить с конусами трения Кулона.

% ------------------------------------------------------
\section{Бонус: время движения системы}

Энергия всякий раз убывает на ненулевую конечную величину из-за эффекта рамки. Оценить время движения или количество ``шагов'', исходя из значений этой величины.


\vspace{20pt}

{\huge Открытые вопросы}
\begin{itemize}
    \item \sout{Почему обратима матрица с реакциями и откуда она берется ?} -- потому что якобиан.
    \item \sout{Почему обратима матрица с проекцией ?} -- потому что это матрица Грама л.н.з. с-мы столбцов $\cstr$.
    \item \sout{Почему два способа дают один результат ?} -- потому что второй следует из первого
    \item \sout{Каковы значения реакций ?} -- посчитано
    \item \sout{Как назвать подразделы ?} -- сделано
\end{itemize}

Оценка скорости убывания кинетической энергии:
$$
T^+ = T^- + \Delta T,
$$
Надо выписать матрицу $A$ (надо другое обозначение, придумаете?), которая определена так:
$$
\Delta T = -\frac12(M\Delta\dot{q},\Delta\dot{q}) = -\frac12(A^T M A\dot{q}^-, \dot{q}^{-})
$$
и найти собственные значения матрицы $A^T M A$ в метрике $M$, решив уравнение
$$
A^T M A - \lambda M = 0
$$
тогда 
$$
\Delta T = -\frac12(A^T M A\dot{q}^-, \dot{q}^{-}) \leq -|\lambda| T^{-}
$$
$\lambda$ -- наименьшее по модулю собственное значение.
$$
T_{i+1}\leq (1-|\lambda|) T_{i}
$$
Проблема, которая может возникнуть -- нулевые собственные значения. Надо проверять.
Отсюда можно оценить число ударов до полной остановки.
Остается вопрос про время между ударами.
Проблема в том, что кинетическая энергия может вся ``сидеть'' в энергии свободных роликов. Наверное, надо как-то сказать -- пусть что-то из $\nu_1$, $\nu_2$, $\nu_3$ не равно нулю. Тогда по энергии вот этих скоростей можно (наверное) оценить $\dot{\chi}$ из связей, а угол, на который колесо поворачивается между ударами, задан геометрией -- отсюда время между ударами.

Тут есть мысль про полную остановку: чтобы она произошла, в какой-то момент часть вектора $\dotq$ должна оказаться прямо вертикальной на картинке с проекцией, то есть у той линейной системы должно оказаться решение вот прямо с тремя нулями первыми компонентами. Звучит так, что что-то особенное должно произойти, чтобы система всё-таки остановилась совсем. Что-нибудь с чем-нибудь соизмеримо там оказаться, например... Как-то странно ожидать, что $(\dot{x}, \dot{y}, \dot{\theta})^T$ вот сами возьмут и окажутся целиком ортогональны $\subspace$, не~так~ли?

\vspace{20pt}

{\huge Ниже -- части статьи}
\setcounter{section}{1}
\newpage

% 
\section{Постановка задачи}

\begin{figure}
    \minipage{0.5\textwidth}
        \centering
        \asyinclude{./asy/pic_cart.asy}
        \caption{Экипаж}
        \label{fig:vehicle}
    \endminipage
    \minipage{0.5\textwidth}
        \centering
        \asyinclude{./asy/pic_wheel.asy}
        \caption{Колесо}
        \label{fig:wheel}
    \endminipage
\end{figure}

Рассмотрим экипаж с омни-колесами, движущийся по инерции по неподвижной абсолютно шероховатой горизонтальной плоскости. Экипаж состоит из платформы и $N$ омни-колес, плоскости которых относительно платформы неподвижны. Каждое колесо может свободно вращаться относительно платформы вокруг собственной оси, расположенной горизонтально. Будем считать, что на каждом колесе установлено $n$ массивных роликов, так что оси роликов лежат в плоскостях колёс и направлены по касательной к границам дисков колес (см. рис.~\ref{fig:wheel}). Таким образом, система состоит из $N(n+1) + 1$ абсолютно твердых тел. 

Введем неподвижную систему отсчета так, что ось $OZ$ направлена вертикально вверх, а плоскость $OXY$ совпадает с опорной плоскостью.
Введем также подвижную систему отсчета $S\xi\eta Z$, жестко связанную с платформой экипажа так, что плоскость $S\xi\eta$ горизонтальна и содержит центры всех колес $P_i$. Будем считать, что оси колес лежат на лучах, соединяющих центр платформы $S$ и центры колес (см. рис.~\ref{fig:vehicle}), а расстояния от центров колес до $S$ одинаковы и равны $R$. Геометрию установки колес на платформе зададим углами $\alpha_i$ осями колес и осью $S\xi$
(см. рис.~\ref{fig:wheel}). Введем также орты, жестко связанные с дисками колес: пусть $\vec{n}_i = \vec{SP_i}/|\vec{SP_i}|$ -- единичный орт оси $i$-ого колеса, и орты $\vec{n}_i^\perp$ и $\vec{n}_i^z$, лежащие в плоскости диска колеса, так что вектор $\vec{n}_i^z$ вертикален при нулевом повороте колеса. Положения центров роликов на колесе определим углами $\kappa_j$ между ними и направлением, противоположным вектору $\vec{n}_i^z$. 

Положение экипажа будем задавать следующими координатами:
$x, y$ --- координаты точки $S$ на плоскости $OXY$, $\theta$ -- угол между $OX$ и $S\xi$ (угол курса),
$\chi_i$ ($i = 1\dots N$) -- углы поворота колес вокруг их осей, отсчитываемые против часовой стрелки, если смотреть с конца вектора $\vec{n}_i$, и $\phi_j$ -- углы поворота роликов вокруг их собственных осей.
Таким образом, вектор обобщенных координат имеет вид:
$$\vec{q} = (x, y, \theta, \left.\{\chi_i\}\right|_{i=1}^N , \left.\{\phi_k\}\right|_{k=1}^N, \phi_s)^{\mathop{T}}\in\mathbb{R}^{N(n+1) + 3}$$ 
Будем использовать индекс $k$ для углов поворота роликов, находящихся в данный момент в контакте с опорной плоскостью, a $s$ --- для остальных,  ``cвободных'', роликов.

Введем псевдоскорости
$$\vec{\nu} = (\nu_1, \nu_2, \nu_3, \nu_s), \quad \vec{v}_S = R\nu_1\vec{e}_\xi + R\nu_2\vec{e}_\eta, \quad \nu_3 = \Lambda\dot{\theta},\quad \nu_s = \dot{\phi}_s$$
Их механический смысл таков: $\nu_1$, $\nu_2$ --- проекции скорости точки $S$ на оси $S\xi\eta$, связанные с платформой, $\nu_3$ --- с точностью до множителя угловая скорость платформы, $\nu_s$ --- угловые скорости свободных роликов. Таким образом, имеем
$$ \dot{x} = R \nu_1\cos\theta-R\nu_2\sin\theta, \hspace{15pt} \dot{y} = R\nu_1\sin\theta+R\nu_2\cos\theta,$$

Будем считать, что проскальзывания между опорной плоскостью и роликами в контакте не происходит, т.е.
скорости точек $C_i$ контакта равны нулю:
$$\vec{v}_{C_i} = 0,\quad i = 1\dots N.$$
Выражая скорость точек контакта через введенные псевдоскорости и проектируя на векторы $\vec{n}_i$ и $[\vec{e}_Z,\ \vec{n}_i]$ соответсвенно, получим:
\begin{eqnarray}
\dot{\phi_k} &=& \frac{R}{\rho_k }(\nu_1\cos\alpha_k + \nu_2\sin\alpha_k),\text{ где } \rho_k  = l\cos\chi_k - r \label{constraint_roller_contact}\\
\dot{\chi}_i &=& \frac{R}{l}(\nu_1\sin\alpha_i - \nu_2\cos\alpha_i - \frac{\nu_3}{\Lambda})\label{constraint_wheel_contact}
\end{eqnarray}
Заметим, что знаменатель $\rho_k$ в (\ref{constraint_roller_contact}) есть расстояние от оси ролика до точки контакта, обращающееся в ноль на стыке роликов (см. рис.~\ref{fig:wheel}). Это обстоятельство приводит к неустранимым разрывам правых частей уравнений движения и будет рассмотрено отдельно ниже.
Уравнение (\ref{constraint_wheel_contact}) совпадает со связью в случае безынерционной модели. 

Таким образом, выражение обобщенных скоростей через псевдоскорости, учитывающее связи, наложенные на систему, можно записать в матричном виде (явные выражения компонент матрицы $V$ приведены в приложении):
\begin{equation}
    \dot{\vec{q}} = V\vec{\nu},\quad V = V(\theta,\chi_i)
\end{equation}


\section{Удар}

\subsection{Основное уравнение теории удара}\label{sect:impact_classical}

Составим алгебраические уравнения, связывающие значения псевдоскоростей после удара и величины ударных импульсов. В течение бесконечно малого времени $\Delta t$ наложены только геометрические связи, так что скорости $\dot{\mathbf{q}}$ независимы. Запишем основное уравнение удара в обобщенных координатах \cite{Vilke}:
\begin{equation}\label{eq:udar_general}
\eqDeltaqQ,
\end{equation}
где $\mke$ -- матрица кинетической энергии без учета связей (так что матрица кинетической энергии с учетом связей $\M^* = \cstr^T \mke \cstr$ (см. также раздел \ref{sect:eqs} главы 1)), а $\Q$ -- вектор импульсов ударных реакций в обобщенных координатах:
\begin{equation*}
\eqQ
\end{equation*}

Исходя из геометрии системы, найдем связь между компонентами этого вектора и касательными составляющими ударных реакций.
Компоненты касательных реакций вдоль неподвижных осей $OX$, $OY$ показаны на рис.~\ref{fig:react}. Компоненты ударных реакций вдоль оси $OZ$ равны нулю, поскольку в рассматриваемой конфигурации колеса внешняя граница проекции тел всех роликов на плоскость колеса есть окружность, и скорость точки ролика, первой оказывающейся в контакте с опорной плоскостью, параллельна опорной плоскости. Верхний индекс в $F_i^\cdot$ означает проекцию на соответствующую ось неподвижной системы отсчета. Тогда имеем
\begin{eqnarray*}
\eqQiOne \\
\eqQiTwo \\
\eqQiTheta \\
\eqQChii \\
\eqQPhii \\
\eqQs
\end{eqnarray*}
В матричном виде:
\begin{equation*}
    % \eqQKFmat
    \eqQKF
\end{equation*}
где
\begin{eqnarray*}
    \hspace{-45pt}
    \mK = \K
\end{eqnarray*}
Размер матрицы $\mK$ равен $(3 + N(n+1)) \times 2N$, её ранг максимален и равен $(3 + N(n+1))$, что можно показать непосредственным вычислением.

В момент удара происходит мгновенное снятие связей, запрещающих проскальзывание роликов, выходящих из контакта и мгновенное наложение аналогичных связей на вновь входящие в контакт ролики:
\begin{equation*}
\eqqnu
\end{equation*}
Отсюда уравнение \upr{eq:udar_general} принимает вид:
\begin{equation}\label{eq:udar_mat}
\eqMVnuKF.
\end{equation}
В следющем пункте будет доказано, что полученная линейная система алгебраических уравнений относительно $\nuposle$ и $\F$ допускает единственное решение.

% Либо в матричной форме
% \begin{equation*}
% \eqMVnuKFmat
% \end{equation*}

% Решение тогда получается следующим образом
% \begin{equation*}
% \eqMVnuKFmatres
% \end{equation*}
% где матрица $\mvk$ обратима в силу геометрии системы.


\subsection{Разрешимость основного уравнения теории удара при наложении дифференциальных связей}

Обоснуем существование и единственность решения уравнения\upr{eq:udar_mat}.

Наложение на систему связей, аналогичных\upr{eq:constraints_vec}, равносильно заданию на конфигурационном многообразии дифференциальной формы $\J$ и требованию к вектору обобщенных скоростей находиться в ядре этой формы:
\begin{equation}\label{eq:constraints_form}
    \J(\q)\dqposle = 0,
\end{equation}
то есть в линейном подпространстве $\subspace = \ker \J$ пространства виртуальных перемещений $T_\q\M$. В этом подпространстве размерности $\dim \q - \rk \J$ можно выбрать базис и компоненты вектора обобщенных скоростей в этом базисе использовать как псевдоскорости: $\dqposle = \cstr\nuposle$, при этом для матрицы дифференциальной формы связей $\J$ и матрицы $\cstr$, связывающей псевдоскорости и обобщенные скорости, будет выполнено равенство:
\begin{equation}\label{eq:constraints_orth}
    \J\cstr = 0.    
\end{equation}
Налагаемые связи будем считать идеальными, то есть работу ударных импульсов на допустимых связями перемещениях равной нулю:
\begin{equation}\label{eq:constraints_ideal}
    0 = \Q^T \delta\qposle = \Q^T \cstr\delta\nuposle.
\end{equation}
Таким образом, ударные импульсы находятся в подпространстве, ортогональном $\subspace$, а значит, вектор $\Q$ может быть представлен в базисе, образованном столбцами матрицы $\J^T$: $\Q = \J^T\F$.

Для экипажа с омни-колесами, рассматриваемого здесь, матрица дифференциальной формы связей\upr{eq:constraints_vec} может быть получена, в частности, как якобиан зависимости вектора из координат точек нижних роликов колес экипажа, находящихся в контакте с опорной плоскостью, в неподвижной системе отсчета $OXYZ$, от обобщенных координат $\q$: $\J(\q) = \ddfrac{\partial \mathbf{r}(\q)}{\partial \q}$, где $\mathbf{r} = ( x_1, y_1, \ldots, x_N, y_N )^T$ (координаты вдоль оси $Z$ опущены, поскольку они учтены при введении обобщенных координат, и соответствующие им компоненты якобиана равны нулю). В этом случае $\J^T$ в точности совпадает с матрицей $\mK$ из~раздела~\ref{sect:impact_classical}.

Итак, основное уравнение удара\upr{eq:udar_general} можно записать в более общем, чем полученный для изучаемого экипажа\upr{eq:udar_mat}, виде:
\begin{equation}\label{eq:udar_mat_Jt}
    \mke\cstr\nuposle - \mathbf{\J}^T\F = \mke\dqdo,
\end{equation}
где вместо матрицы $\mK$, приведенной в разделе~\ref{sect:impact_classical}, стоит любая матрица дифференциальной формы связей.

Равенство\upr{eq:udar_mat_Jt} есть система алгебраических уравнений относительно вектора неизвестных $(\nuposle, \F)^T$. Матрица $(\mke\cstr; -\J^T)$ этой системы -- квадратная размерности $\dim \q \times \dim \q$, поскольку столбцы $\cstr$ и $\J^T$ образуют базисы в дополнительных подпространствах $\mathbb{R}^{\dim \q}$. По той же причине она невырожденна. Действительно, поскольку с одной стороны, каждая из систем столбцов $\{\cstr_{\cdot i}\}$ и $\{\J^T_{\cdot j}\}$ линейно-независима, и с другой, эти две системы взаимно ортогональны\upr{eq:constraints_orth}, то действие положительно определенного оператора $\mke$ на $\cstr$ не нарушает линейной независимости системы столбцов в целом. Показать это можно от противного, предположив существование таких чисел $x_i$ и $y_j$, не равных нулю одновременно, что $\sum_i x_i \mke \cstr_{\cdot i} + \sum_j y_j \J^T_{\cdot j} = 0$, домножив это равенство скалярно на каждый из столбцов $\{\cstr_{\cdot \alpha}\}$ и рассмотрев сумму полученных выражений: $\sum_{\alpha, i} x_i \cstr_{\cdot \alpha}^T \mke \cstr_{\cdot i} + \sum_{\alpha, j} y_j \cstr_{\cdot \alpha}^T \mke \J^T_{\cdot j} = \cstr^T \mke \cstr \mathbf{x} + \cstr^T\J^T\mathbf{y}$, где последнее слагаемое равно нулю в силу\upr{eq:constraints_orth}, а первое отлично от нуля, поскольку матрица $\cstr^T \mke \cstr$ является матрицей Грама линейно-независимой системы столбцов матрицы $\cstr$ в метрике $\mke$, а значит, невырожденна -- противоречие. Таким образом, система\upr{eq:udar_mat_Jt} имеет единственное решение.

Отметим также, что мгновенное наложение связи можно рассматривать как абсолютно неупругий удар, при котором теряется компонента $\deltadq$ вектора обобщенных скоростей $\dqdo$, ортогональная подпространству $\subspace$ в кинетической метрике, и соответственно, находить вектор обобщенных скоростей после удара $\dqposle = \dqdo - \deltadq = \mathbf{V}\nuposle \in \subspace$ можно непосредственно из условия идеальности связей\upr{eq:constraints_ideal} и основного уравнения удара\upr{eq:udar_general}, минуя вычисление величин реакций $\Q$ или $\F$ и необходимой для этого матрицы $\J$:
\begin{equation*}
    0 = \cstr^T\Q = \cstr^T\mke\deltadq = \cstr^T\mke(\cstr\nuposle - \dqdo) = \cstr^T\mke\cstr\nuposle - \cstr^T\mke\dqdo,
\end{equation*}
откуда:
\begin{equation*}
\eqnuposleproj.
\end{equation*}
Эту же формулу можно получить и из уравнения\upr{eq:udar_mat_Jt}, домножая его слева на $\cstr^T$. Симметрично, при умножении\upr{eq:udar_mat_Jt} слева на $\J\mke^{-1}$, имеем выражение для реакций:
\begin{equation*}
    \F = -(\J \mke \J^T)^{-1}\J\dqdo,
\end{equation*}
не включающее явно матрицу связей $\cstr$.



\subsection{Изменение кинетической энергии}

\begin{itemize}
    \item проверка, что оба способа дают один результат
    \item проверка, что результат соответствует теореме Карно (потеря энергии равна энергии потерянных скоростей)
\end{itemize}

Кинетическая энергия в системе без неголономных связей -- квадратичная форма.
\begin{equation*}
\eqTquad
\end{equation*}

Связи
\begin{equation*}
\eqDqVnu
\end{equation*}

В силу идеальности неголономных связей, потеря энергии при ударе равна кинетической энергии потерянных скоростей, в соответствии с теоремой Карно.
Воспользуемся следующими фактами: $\edQposleDelta$, $\eqDqposleInSubspace$, $\eqDeltaOrth$, $\edMkeSim$, работа ударных испульсов на перемещениях, допускаемых вновь налагаемыми связями, равна нулю $\logicWorkZero$
\begin{eqnarray*}
\logicDeltaT
\end{eqnarray*}

таким образом,
\begin{equation*}
\eqDeltaT
\end{equation*}



\ifdefined\IMPORTS  \else % \smartqed  % flush right qed marks, e.g. at end of proof

\usepackage{amsmath}
\usepackage{amsfonts}

% Russian-specific packages
%--------------------------------------
\usepackage[T2A]{fontenc}
\usepackage[utf8]{inputenc}
\usepackage[russian]{babel}
%--------------------------------------

% Asymptote for pictures
%--------------------------------------
\usepackage{asymptote} %% comes with options inline and attach
%--------------------------------------

% graphicx for graphs
%--------------------------------------
\usepackage{graphicx}

%--------------------------------------
% Specially for PMM: make all imported EPS grayscale:
%--------------------------------------
\usepackage[gray]{epspdfconversion}
%--------------------------------------

% \usepackage{subfig} % incompatible with subcaption package
% \graphicspath{ % not used here
    % {./pic/,./asy/}
% }
%--------------------------------------

% subcaption for many figures under one big caption
% each having its own small caption
%--------------------------------------
\usepackage{caption}
\usepackage{subcaption}
%--------------------------------------

% so that refs were [1-10], not [1,2,3,4,5,...]
%--------------------------------------
\usepackage{cite}
%--------------------------------------

% \ddfrac command to show big fractions, not cramped up
% https://tex.stackexchange.com/questions/173899/
%--------------------------------------
\newcommand\ddfrac[2]{\displaystyle\frac{\displaystyle #1}{\displaystyle #2}}
%--------------------------------------

% \vsp command to make a spacey newline
% useful for equations arrays
%--------------------------------------
\newcommand\vsp[1][10]{\\[#1pt]}
%--------------------------------------

% partial derivatives (can \usepackage{physics}, but only one command so far, so no)
%--------------------------------------
\newcommand\pd[2]{\frac{\partial #1}{\partial #2}}
\newcommand\ddpd[2]{\ddfrac{\partial #1}{\partial #2}}
\newcommand\ddt[1]{\frac{d #1}{dt}}
\newcommand\ddddt[1]{\ddfrac{d #1}{dt}}
%--------------------------------------

% unbreakable space parenthesized reference
%--------------------------------------
\newcommand\upr[1]{~(\ref{#1})}
%--------------------------------------

% Nice letters
%--------------------------------------
\newcommand\M[0]{\mathcal{M}} % Matrix of intertia
\newcommand\AntiU[0]{\mathcal{U}} % Helper antisymmetric matrix for eqs' RHS
\newcommand\Rhs[0]{\mathcal{R}} % RHS
\newcommand\Prhs[0]{\mathcal{P}} % The family of matrices for RHS
\newcommand\prhs[0]{\mathbf{p}} % Poisson brackets
%--------------------------------------

\renewcommand{\vec}[1]{\boldsymbol{\mathbf{#1}}}

\newtheorem{stmt}{Утверждение}
\newtheorem{prblm}{Затруднение} \fi






\chapter{Динамика экипажа на омни-колесах с трением}

В предыдущих главах рассмотрена динамика омни-колесного экипажа на абсолютно шероховатой плоскости. Однако, как было показано ранее, в этом случае на интервалах движения без смены ролика в контакте полная механическая энергия сохраняется, что никогда не происходит в реальных системах. Отдельного рассмотрения заслуживает вопрос о том, прекращается ли в реальных системах скольжение вновь вошедшего в контакт ролика. Таким образом, 
интерес представляет изучение динамики экипажа по плоскости с трением.

В данной главе вместо идеальных неголомных связей отсутствия проскальзывания используются голономные неидеальные неудерживающие связи. Для касательных составляющих реакций опорной плоскости используются две модели: сухое трение Амонтона -- Кулона, регуляризованного в окрестности нуля по скорости проскальзывания линейной функцией с насыщением; вязкое трением.

Построение модели выполнено таким образом, что изменение в ней модели контактного взаимодействия требует изменения всего одной алгебраической формулы. 
Также подробно рассмотрен вопрос отслеживания контакта роликов и горизонтальной плоскости и алгоритмической реализации процесса  переключения контакта от ролика к ролику при качении  роликонесущего колеса.

Динамические свойства построенной модели экипажа проиллюстрированы при помощи численных экспериментов.
Проведена верификация построенной модели с сухим трением в сравнении с безынерционной моделью при стремлении суммарной массы роликов к нулю. Модель с вязким трением представлена в сравнивнении с неголономной моделью, построенной в главах 1 и 2.

\section{Постановка задачи}

\begin{figure}
    \minipage{0.5\textwidth}
        \centering
        \asyinclude{./asy/pic_cart.asy}
        \caption{Экипаж}
        \label{fig:vehicle}
    \endminipage
    \minipage{0.5\textwidth}
        \centering
        \asyinclude{./asy/pic_wheel.asy}
        \caption{Колесо}
        \label{fig:wheel}
    \endminipage
\end{figure}

Рассмотрим экипаж с омни-колесами, движущийся по инерции по неподвижной абсолютно шероховатой горизонтальной плоскости. Экипаж состоит из платформы и $N$ омни-колес, плоскости которых относительно платформы неподвижны. Каждое колесо может свободно вращаться относительно платформы вокруг собственной оси, расположенной горизонтально. Будем считать, что на каждом колесе установлено $n$ массивных роликов, так что оси роликов лежат в плоскостях колёс и направлены по касательной к границам дисков колес (см. рис.~\ref{fig:wheel}). Таким образом, система состоит из $N(n+1) + 1$ абсолютно твердых тел. 

Введем неподвижную систему отсчета так, что ось $OZ$ направлена вертикально вверх, а плоскость $OXY$ совпадает с опорной плоскостью.
Введем также подвижную систему отсчета $S\xi\eta Z$, жестко связанную с платформой экипажа так, что плоскость $S\xi\eta$ горизонтальна и содержит центры всех колес $P_i$. Будем считать, что оси колес лежат на лучах, соединяющих центр платформы $S$ и центры колес (см. рис.~\ref{fig:vehicle}), а расстояния от центров колес до $S$ одинаковы и равны $R$. Геометрию установки колес на платформе зададим углами $\alpha_i$ осями колес и осью $S\xi$
(см. рис.~\ref{fig:wheel}). Введем также орты, жестко связанные с дисками колес: пусть $\vec{n}_i = \vec{SP_i}/|\vec{SP_i}|$ -- единичный орт оси $i$-ого колеса, и орты $\vec{n}_i^\perp$ и $\vec{n}_i^z$, лежащие в плоскости диска колеса, так что вектор $\vec{n}_i^z$ вертикален при нулевом повороте колеса. Положения центров роликов на колесе определим углами $\kappa_j$ между ними и направлением, противоположным вектору $\vec{n}_i^z$. 

Положение экипажа будем задавать следующими координатами:
$x, y$ --- координаты точки $S$ на плоскости $OXY$, $\theta$ -- угол между $OX$ и $S\xi$ (угол курса),
$\chi_i$ ($i = 1\dots N$) -- углы поворота колес вокруг их осей, отсчитываемые против часовой стрелки, если смотреть с конца вектора $\vec{n}_i$, и $\phi_j$ -- углы поворота роликов вокруг их собственных осей.
Таким образом, вектор обобщенных координат имеет вид:
$$\vec{q} = (x, y, \theta, \left.\{\chi_i\}\right|_{i=1}^N , \left.\{\phi_k\}\right|_{k=1}^N, \phi_s)^{\mathop{T}}\in\mathbb{R}^{N(n+1) + 3}$$ 
Будем использовать индекс $k$ для углов поворота роликов, находящихся в данный момент в контакте с опорной плоскостью, a $s$ --- для остальных,  ``cвободных'', роликов.

Введем псевдоскорости
$$\vec{\nu} = (\nu_1, \nu_2, \nu_3, \nu_s), \quad \vec{v}_S = R\nu_1\vec{e}_\xi + R\nu_2\vec{e}_\eta, \quad \nu_3 = \Lambda\dot{\theta},\quad \nu_s = \dot{\phi}_s$$
Их механический смысл таков: $\nu_1$, $\nu_2$ --- проекции скорости точки $S$ на оси $S\xi\eta$, связанные с платформой, $\nu_3$ --- с точностью до множителя угловая скорость платформы, $\nu_s$ --- угловые скорости свободных роликов. Таким образом, имеем
$$ \dot{x} = R \nu_1\cos\theta-R\nu_2\sin\theta, \hspace{15pt} \dot{y} = R\nu_1\sin\theta+R\nu_2\cos\theta,$$

Будем считать, что проскальзывания между опорной плоскостью и роликами в контакте не происходит, т.е.
скорости точек $C_i$ контакта равны нулю:
$$\vec{v}_{C_i} = 0,\quad i = 1\dots N.$$
Выражая скорость точек контакта через введенные псевдоскорости и проектируя на векторы $\vec{n}_i$ и $[\vec{e}_Z,\ \vec{n}_i]$ соответсвенно, получим:
\begin{eqnarray}
\dot{\phi_k} &=& \frac{R}{\rho_k }(\nu_1\cos\alpha_k + \nu_2\sin\alpha_k),\text{ где } \rho_k  = l\cos\chi_k - r \label{constraint_roller_contact}\\
\dot{\chi}_i &=& \frac{R}{l}(\nu_1\sin\alpha_i - \nu_2\cos\alpha_i - \frac{\nu_3}{\Lambda})\label{constraint_wheel_contact}
\end{eqnarray}
Заметим, что знаменатель $\rho_k$ в (\ref{constraint_roller_contact}) есть расстояние от оси ролика до точки контакта, обращающееся в ноль на стыке роликов (см. рис.~\ref{fig:wheel}). Это обстоятельство приводит к неустранимым разрывам правых частей уравнений движения и будет рассмотрено отдельно ниже.
Уравнение (\ref{constraint_wheel_contact}) совпадает со связью в случае безынерционной модели. 

Таким образом, выражение обобщенных скоростей через псевдоскорости, учитывающее связи, наложенные на систему, можно записать в матричном виде (явные выражения компонент матрицы $V$ приведены в приложении):
\begin{equation}
    \dot{\vec{q}} = V\vec{\nu},\quad V = V(\theta,\chi_i)
\end{equation}


\section{Наложение связи при смене ролика в контакте}

\begin{figure}[h]
    \minipage{0.5\textwidth}
        \centering
        \asyinclude{./asy/pic_react}
        \caption{Реакции}
        \label{fig:react}
    \endminipage
    \minipage{0.5\textwidth}
        \centering
        \asyinclude{./asy/pic_project}
        \caption{Проекция}
        \label{fig:project}
    \endminipage
\end{figure}

В реальной системе при смене контакта имеет место скольжение роликов относительно плоскости, и при этом полная энергия системы рассеивается. В данной работе будем считать, что трение достаточно велико, и прекращение проскальзывания вновь вошедшего в контакт ролика происходит мгновенно. Это взаимодействие будет рассматриваться как абсолютно неупругий удар, происходящий при мгновенном наложении связи.
Освободившийся ролик начинает свободно вращаться вокруг своей оси.
Будем предполагать следующее:
\begin{itemize}
    \item удар происходит за бесконечно малый интервал времени $\Delta t << 1$, так что изменения обобщенных координат пренебрежимо малы $\Delta \q \sim \dotq\Delta t << 1$, а изменения обобщенных скоростей конечны $\Delta \dotq < \infty$;
    \item взаимодействие экипажа с опорной полоскостью во время удара сводится к действию в точках контакта нормальных и касательных реакций $\mathbf{R}_i = \mathbf{N}_i + \mathbf{F}_i$ с нулевым относительно точек касания моментом $\mathbf{M}_i = 0$;
    \item к моменту окончания удара $t^*+\Delta t$ уравнения связей выполнены $\dqposle = \cstr(\q)\nuposle$, т.е. за время $\Delta t$ проскальзывание вошедшего в контакт ролика заканчивается.
\end{itemize}

% Таким образом, решение задачи представляется в виде чередования гладких участков, определяемых уравнениями движения, и пересчетов значений обобщенных скоростей в моменты смен контакта.

Исходя из этих предположений, в следующих разделах получим системы алгебраических уравнений, связывающих значения обобщенных скоростей непосредственно перед ударом $\dqdo$ и значения псевдоскоростей сразу после удара $\nuposle$ двумя разными способами: в первом случае, будем  вводить ударные реакции, действующие в точках контакта, а во втором, будем рассматривать неупругий удар как проецирование вектора обобщенных скоростей на плоскость, задаваемую уравнениями вновь налагаемых связей.

Таким образом, моделирование системы состоит в решении задачи Коши системы обыкновенных дифференциальных уравнений в интервалах между моментами смены роликов в контактах и решения систем алгебраических уравнений в эти моменты для получения начальных условий для следующего безударного интервала.


\subsection{Основное уравнение теории удара}\label{sect:impact_classical}

Составим алгебраические уравнения, связывающие значения псевдоскоростей после удара и величины ударных импульсов. В течение бесконечно малого времени $\Delta t$ наложены только геометрические связи, так что скорости $\dot{\mathbf{q}}$ независимы. Запишем основное уравнение удара в обобщенных координатах \cite{Vilke}:
\begin{equation}\label{eq:udar_general}
\eqDeltaqQ,
\end{equation}
где $\mke$ -- матрица кинетической энергии без учета связей (так что матрица кинетической энергии с учетом связей $\M^* = \cstr^T \mke \cstr$ (см. также раздел \ref{sect:eqs} главы 1)), а $\Q$ -- вектор импульсов ударных реакций в обобщенных координатах:
\begin{equation*}
\eqQ
\end{equation*}

Исходя из геометрии системы, найдем связь между компонентами этого вектора и касательными составляющими ударных реакций.
Компоненты касательных реакций вдоль неподвижных осей $OX$, $OY$ показаны на рис.~\ref{fig:react}. Компоненты ударных реакций вдоль оси $OZ$ равны нулю, поскольку в рассматриваемой конфигурации колеса внешняя граница проекции тел всех роликов на плоскость колеса есть окружность, и скорость точки ролика, первой оказывающейся в контакте с опорной плоскостью, параллельна опорной плоскости. Верхний индекс в $F_i^\cdot$ означает проекцию на соответствующую ось неподвижной системы отсчета. Тогда имеем
\begin{eqnarray*}
\eqQiOne \\
\eqQiTwo \\
\eqQiTheta \\
\eqQChii \\
\eqQPhii \\
\eqQs
\end{eqnarray*}
В матричном виде:
\begin{equation*}
    % \eqQKFmat
    \eqQKF
\end{equation*}
где
\begin{eqnarray*}
    \hspace{-45pt}
    \mK = \K
\end{eqnarray*}
Размер матрицы $\mK$ равен $(3 + N(n+1)) \times 2N$, её ранг максимален и равен $(3 + N(n+1))$, что можно показать непосредственным вычислением.

В момент удара происходит мгновенное снятие связей, запрещающих проскальзывание роликов, выходящих из контакта и мгновенное наложение аналогичных связей на вновь входящие в контакт ролики:
\begin{equation*}
\eqqnu
\end{equation*}
Отсюда уравнение \upr{eq:udar_general} принимает вид:
\begin{equation}\label{eq:udar_mat}
\eqMVnuKF.
\end{equation}
В следющем пункте будет доказано, что полученная линейная система алгебраических уравнений относительно $\nuposle$ и $\F$ допускает единственное решение.

% Либо в матричной форме
% \begin{equation*}
% \eqMVnuKFmat
% \end{equation*}

% Решение тогда получается следующим образом
% \begin{equation*}
% \eqMVnuKFmatres
% \end{equation*}
% где матрица $\mvk$ обратима в силу геометрии системы.


\subsection{Разрешимость основного уравнения теории удара при наложении дифференциальных связей}

Обоснуем существование и единственность решения уравнения\upr{eq:udar_mat}.

Наложение на систему связей, аналогичных\upr{eq:constraints_vec}, равносильно заданию на конфигурационном многообразии дифференциальной формы $\J$ и требованию к вектору обобщенных скоростей находиться в ядре этой формы:
\begin{equation}\label{eq:constraints_form}
    \J(\q)\dqposle = 0,
\end{equation}
то есть в линейном подпространстве $\subspace = \ker \J$ пространства виртуальных перемещений $T_\q\M$. В этом подпространстве размерности $\dim \q - \rk \J$ можно выбрать базис и компоненты вектора обобщенных скоростей в этом базисе использовать как псевдоскорости: $\dqposle = \cstr\nuposle$, при этом для матрицы дифференциальной формы связей $\J$ и матрицы $\cstr$, связывающей псевдоскорости и обобщенные скорости, будет выполнено равенство:
\begin{equation}\label{eq:constraints_orth}
    \J\cstr = 0.    
\end{equation}
Налагаемые связи будем считать идеальными, то есть работу ударных импульсов на допустимых связями перемещениях равной нулю:
\begin{equation}\label{eq:constraints_ideal}
    0 = \Q^T \delta\qposle = \Q^T \cstr\delta\nuposle.
\end{equation}
Таким образом, ударные импульсы находятся в подпространстве, ортогональном $\subspace$, а значит, вектор $\Q$ может быть представлен в базисе, образованном столбцами матрицы $\J^T$: $\Q = \J^T\F$.

Для экипажа с омни-колесами, рассматриваемого здесь, матрица дифференциальной формы связей\upr{eq:constraints_vec} может быть получена, в частности, как якобиан зависимости вектора из координат точек нижних роликов колес экипажа, находящихся в контакте с опорной плоскостью, в неподвижной системе отсчета $OXYZ$, от обобщенных координат $\q$: $\J(\q) = \ddfrac{\partial \mathbf{r}(\q)}{\partial \q}$, где $\mathbf{r} = ( x_1, y_1, \ldots, x_N, y_N )^T$ (координаты вдоль оси $Z$ опущены, поскольку они учтены при введении обобщенных координат, и соответствующие им компоненты якобиана равны нулю). В этом случае $\J^T$ в точности совпадает с матрицей $\mK$ из~раздела~\ref{sect:impact_classical}.

Итак, основное уравнение удара\upr{eq:udar_general} можно записать в более общем, чем полученный для изучаемого экипажа\upr{eq:udar_mat}, виде:
\begin{equation}\label{eq:udar_mat_Jt}
    \mke\cstr\nuposle - \mathbf{\J}^T\F = \mke\dqdo,
\end{equation}
где вместо матрицы $\mK$, приведенной в разделе~\ref{sect:impact_classical}, стоит любая матрица дифференциальной формы связей.

Равенство\upr{eq:udar_mat_Jt} есть система алгебраических уравнений относительно вектора неизвестных $(\nuposle, \F)^T$. Матрица $(\mke\cstr; -\J^T)$ этой системы -- квадратная размерности $\dim \q \times \dim \q$, поскольку столбцы $\cstr$ и $\J^T$ образуют базисы в дополнительных подпространствах $\mathbb{R}^{\dim \q}$. По той же причине она невырожденна. Действительно, поскольку с одной стороны, каждая из систем столбцов $\{\cstr_{\cdot i}\}$ и $\{\J^T_{\cdot j}\}$ линейно-независима, и с другой, эти две системы взаимно ортогональны\upr{eq:constraints_orth}, то действие положительно определенного оператора $\mke$ на $\cstr$ не нарушает линейной независимости системы столбцов в целом. Показать это можно от противного, предположив существование таких чисел $x_i$ и $y_j$, не равных нулю одновременно, что $\sum_i x_i \mke \cstr_{\cdot i} + \sum_j y_j \J^T_{\cdot j} = 0$, домножив это равенство скалярно на каждый из столбцов $\{\cstr_{\cdot \alpha}\}$ и рассмотрев сумму полученных выражений: $\sum_{\alpha, i} x_i \cstr_{\cdot \alpha}^T \mke \cstr_{\cdot i} + \sum_{\alpha, j} y_j \cstr_{\cdot \alpha}^T \mke \J^T_{\cdot j} = \cstr^T \mke \cstr \mathbf{x} + \cstr^T\J^T\mathbf{y}$, где последнее слагаемое равно нулю в силу\upr{eq:constraints_orth}, а первое отлично от нуля, поскольку матрица $\cstr^T \mke \cstr$ является матрицей Грама линейно-независимой системы столбцов матрицы $\cstr$ в метрике $\mke$, а значит, невырожденна -- противоречие. Таким образом, система\upr{eq:udar_mat_Jt} имеет единственное решение.

Отметим также, что мгновенное наложение связи можно рассматривать как абсолютно неупругий удар, при котором теряется компонента $\deltadq$ вектора обобщенных скоростей $\dqdo$, ортогональная подпространству $\subspace$ в кинетической метрике, и соответственно, находить вектор обобщенных скоростей после удара $\dqposle = \dqdo - \deltadq = \mathbf{V}\nuposle \in \subspace$ можно непосредственно из условия идеальности связей\upr{eq:constraints_ideal} и основного уравнения удара\upr{eq:udar_general}, минуя вычисление величин реакций $\Q$ или $\F$ и необходимой для этого матрицы $\J$:
\begin{equation*}
    0 = \cstr^T\Q = \cstr^T\mke\deltadq = \cstr^T\mke(\cstr\nuposle - \dqdo) = \cstr^T\mke\cstr\nuposle - \cstr^T\mke\dqdo,
\end{equation*}
откуда:
\begin{equation*}
\eqnuposleproj.
\end{equation*}
Эту же формулу можно получить и из уравнения\upr{eq:udar_mat_Jt}, домножая его слева на $\cstr^T$. Симметрично, при умножении\upr{eq:udar_mat_Jt} слева на $\J\mke^{-1}$, имеем выражение для реакций:
\begin{equation*}
    \F = -(\J \mke \J^T)^{-1}\J\dqdo,
\end{equation*}
не включающее явно матрицу связей $\cstr$.



\subsection{Изменение кинетической энергии}

\begin{itemize}
    \item проверка, что оба способа дают один результат
    \item проверка, что результат соответствует теореме Карно (потеря энергии равна энергии потерянных скоростей)
\end{itemize}

Кинетическая энергия в системе без неголономных связей -- квадратичная форма.
\begin{equation*}
\eqTquad
\end{equation*}

Связи
\begin{equation*}
\eqDqVnu
\end{equation*}

В силу идеальности неголономных связей, потеря энергии при ударе равна кинетической энергии потерянных скоростей, в соответствии с теоремой Карно.
Воспользуемся следующими фактами: $\edQposleDelta$, $\eqDqposleInSubspace$, $\eqDeltaOrth$, $\edMkeSim$, работа ударных испульсов на перемещениях, допускаемых вновь налагаемыми связями, равна нулю $\logicWorkZero$
\begin{eqnarray*}
\logicDeltaT
\end{eqnarray*}

таким образом,
\begin{equation*}
\eqDeltaT
\end{equation*}



\ifdefined\IMPORTS  \else % \smartqed  % flush right qed marks, e.g. at end of proof

\usepackage{amsmath}
\usepackage{amsfonts}

% Russian-specific packages
%--------------------------------------
\usepackage[T2A]{fontenc}
\usepackage[utf8]{inputenc}
\usepackage[russian]{babel}
%--------------------------------------

% Asymptote for pictures
%--------------------------------------
\usepackage{asymptote} %% comes with options inline and attach
%--------------------------------------

% graphicx for graphs
%--------------------------------------
\usepackage{graphicx}

%--------------------------------------
% Specially for PMM: make all imported EPS grayscale:
%--------------------------------------
\usepackage[gray]{epspdfconversion}
%--------------------------------------

% \usepackage{subfig} % incompatible with subcaption package
% \graphicspath{ % not used here
    % {./pic/,./asy/}
% }
%--------------------------------------

% subcaption for many figures under one big caption
% each having its own small caption
%--------------------------------------
\usepackage{caption}
\usepackage{subcaption}
%--------------------------------------

% so that refs were [1-10], not [1,2,3,4,5,...]
%--------------------------------------
\usepackage{cite}
%--------------------------------------

% \ddfrac command to show big fractions, not cramped up
% https://tex.stackexchange.com/questions/173899/
%--------------------------------------
\newcommand\ddfrac[2]{\displaystyle\frac{\displaystyle #1}{\displaystyle #2}}
%--------------------------------------

% \vsp command to make a spacey newline
% useful for equations arrays
%--------------------------------------
\newcommand\vsp[1][10]{\\[#1pt]}
%--------------------------------------

% partial derivatives (can \usepackage{physics}, but only one command so far, so no)
%--------------------------------------
\newcommand\pd[2]{\frac{\partial #1}{\partial #2}}
\newcommand\ddpd[2]{\ddfrac{\partial #1}{\partial #2}}
\newcommand\ddt[1]{\frac{d #1}{dt}}
\newcommand\ddddt[1]{\ddfrac{d #1}{dt}}
%--------------------------------------

% unbreakable space parenthesized reference
%--------------------------------------
\newcommand\upr[1]{~(\ref{#1})}
%--------------------------------------

% Nice letters
%--------------------------------------
\newcommand\M[0]{\mathcal{M}} % Matrix of intertia
\newcommand\AntiU[0]{\mathcal{U}} % Helper antisymmetric matrix for eqs' RHS
\newcommand\Rhs[0]{\mathcal{R}} % RHS
\newcommand\Prhs[0]{\mathcal{P}} % The family of matrices for RHS
\newcommand\prhs[0]{\mathbf{p}} % Poisson brackets
%--------------------------------------

\renewcommand{\vec}[1]{\boldsymbol{\mathbf{#1}}}

\newtheorem{stmt}{Утверждение}
\newtheorem{prblm}{Затруднение} \fi





\section{Выводы}

В ходе работы построен способ расчета изменения обобщенных скоростей при смене ролика в контакте, в предположении о мгновенном выполнении условия отсутствия проскальзывания между роликом и опорной плоскостью. Смена контакта рассмотрена с точки зрения классической механики, и приведено обоснование разрешимости задачи теории удара при мгновенном наложении дифференциальных связей на натуральную систему. Получены численные решения для симметричной конфигурации экипажа с омни-колесами с учетом ударного взаимодействия роликов и опорной плоскости.


% % \smartqed  % flush right qed marks, e.g. at end of proof

\usepackage{amsmath}
\usepackage{amsfonts}

% Russian-specific packages
%--------------------------------------
\usepackage[T2A]{fontenc}
\usepackage[utf8]{inputenc}
\usepackage[russian]{babel}
%--------------------------------------

% Asymptote for pictures
%--------------------------------------
\usepackage{asymptote} %% comes with options inline and attach
%--------------------------------------

% graphicx for graphs
%--------------------------------------
\usepackage{graphicx}

%--------------------------------------
% Specially for PMM: make all imported EPS grayscale:
%--------------------------------------
\usepackage[gray]{epspdfconversion}
%--------------------------------------

% \usepackage{subfig} % incompatible with subcaption package
% \graphicspath{ % not used here
    % {./pic/,./asy/}
% }
%--------------------------------------

% subcaption for many figures under one big caption
% each having its own small caption
%--------------------------------------
\usepackage{caption}
\usepackage{subcaption}
%--------------------------------------

% so that refs were [1-10], not [1,2,3,4,5,...]
%--------------------------------------
\usepackage{cite}
%--------------------------------------

% \ddfrac command to show big fractions, not cramped up
% https://tex.stackexchange.com/questions/173899/
%--------------------------------------
\newcommand\ddfrac[2]{\displaystyle\frac{\displaystyle #1}{\displaystyle #2}}
%--------------------------------------

% \vsp command to make a spacey newline
% useful for equations arrays
%--------------------------------------
\newcommand\vsp[1][10]{\\[#1pt]}
%--------------------------------------

% partial derivatives (can \usepackage{physics}, but only one command so far, so no)
%--------------------------------------
\newcommand\pd[2]{\frac{\partial #1}{\partial #2}}
\newcommand\ddpd[2]{\ddfrac{\partial #1}{\partial #2}}
\newcommand\ddt[1]{\frac{d #1}{dt}}
\newcommand\ddddt[1]{\ddfrac{d #1}{dt}}
%--------------------------------------

% unbreakable space parenthesized reference
%--------------------------------------
\newcommand\upr[1]{~(\ref{#1})}
%--------------------------------------

% Nice letters
%--------------------------------------
\newcommand\M[0]{\mathcal{M}} % Matrix of intertia
\newcommand\AntiU[0]{\mathcal{U}} % Helper antisymmetric matrix for eqs' RHS
\newcommand\Rhs[0]{\mathcal{R}} % RHS
\newcommand\Prhs[0]{\mathcal{P}} % The family of matrices for RHS
\newcommand\prhs[0]{\mathbf{p}} % Poisson brackets
%--------------------------------------

\renewcommand{\vec}[1]{\boldsymbol{\mathbf{#1}}}

\newtheorem{stmt}{Утверждение}
\newtheorem{prblm}{Затруднение}

\begin{document}

% \asyinclude{./asy/pic_react}

\ifdefined\IMPORTS  \else % \smartqed  % flush right qed marks, e.g. at end of proof

\usepackage{amsmath}
\usepackage{amsfonts}

% Russian-specific packages
%--------------------------------------
\usepackage[T2A]{fontenc}
\usepackage[utf8]{inputenc}
\usepackage[russian]{babel}
%--------------------------------------

% Asymptote for pictures
%--------------------------------------
\usepackage{asymptote} %% comes with options inline and attach
%--------------------------------------

% graphicx for graphs
%--------------------------------------
\usepackage{graphicx}

%--------------------------------------
% Specially for PMM: make all imported EPS grayscale:
%--------------------------------------
\usepackage[gray]{epspdfconversion}
%--------------------------------------

% \usepackage{subfig} % incompatible with subcaption package
% \graphicspath{ % not used here
    % {./pic/,./asy/}
% }
%--------------------------------------

% subcaption for many figures under one big caption
% each having its own small caption
%--------------------------------------
\usepackage{caption}
\usepackage{subcaption}
%--------------------------------------

% so that refs were [1-10], not [1,2,3,4,5,...]
%--------------------------------------
\usepackage{cite}
%--------------------------------------

% \ddfrac command to show big fractions, not cramped up
% https://tex.stackexchange.com/questions/173899/
%--------------------------------------
\newcommand\ddfrac[2]{\displaystyle\frac{\displaystyle #1}{\displaystyle #2}}
%--------------------------------------

% \vsp command to make a spacey newline
% useful for equations arrays
%--------------------------------------
\newcommand\vsp[1][10]{\\[#1pt]}
%--------------------------------------

% partial derivatives (can \usepackage{physics}, but only one command so far, so no)
%--------------------------------------
\newcommand\pd[2]{\frac{\partial #1}{\partial #2}}
\newcommand\ddpd[2]{\ddfrac{\partial #1}{\partial #2}}
\newcommand\ddt[1]{\frac{d #1}{dt}}
\newcommand\ddddt[1]{\ddfrac{d #1}{dt}}
%--------------------------------------

% unbreakable space parenthesized reference
%--------------------------------------
\newcommand\upr[1]{~(\ref{#1})}
%--------------------------------------

% Nice letters
%--------------------------------------
\newcommand\M[0]{\mathcal{M}} % Matrix of intertia
\newcommand\AntiU[0]{\mathcal{U}} % Helper antisymmetric matrix for eqs' RHS
\newcommand\Rhs[0]{\mathcal{R}} % RHS
\newcommand\Prhs[0]{\mathcal{P}} % The family of matrices for RHS
\newcommand\prhs[0]{\mathbf{p}} % Poisson brackets
%--------------------------------------

\renewcommand{\vec}[1]{\boldsymbol{\mathbf{#1}}}

\newtheorem{stmt}{Утверждение}
\newtheorem{prblm}{Затруднение} \fi




\end{document}





\end{document}
