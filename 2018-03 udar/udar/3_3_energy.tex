
\subsection{Изменение кинетической энергии}

\begin{itemize}
    \item проверка, что оба способа дают один результат
    \item проверка, что результат соответствует теореме Карно (потеря энергии равна энергии потерянных скоростей)
\end{itemize}

Кинетическая энергия в системе без неголономных связей -- квадратичная форма.
\begin{equation*}
\ke = \ddfrac{1}{2}\dotp{\mke\dotq}{\dotq}
\end{equation*}

Связи
\begin{equation*}
\dqposle = \cstr\nuposle
\end{equation*}

В силу идеальности неголономных связей, потеря энергии при ударе равна кинетической энергии потерянных скоростей, в соответствии с теоремой Карно.
Воспользуемся следующими фактами: $\dqposle = \dqdo + \deltadq$, $\dqposle \in \subspace$, $\deltadq \perp_\mke \subspace$, $\mke = \mke^T$, работа ударных испульсов на перемещениях, допускаемых вновь налагаемыми связями, равна нулю $\dotp{\mke\dqposle}{\deltadq} = \dotp{\mke\deltadq}{\dqposle} = \dotp{\mathbf{P}}{\dqposle} = 0$
\begin{eqnarray*}
2\Delta\ke & = & 2\left(\ke^+ - \ke^-\right) = \dotp{\mke\dqposle}{\dqposle} - \dotp{\mke\dqdo}{\dqdo} = \dotp{\mke\deltadq}{\deltadq} + 2\dotp{\mke\dqdo}{\deltadq} \\
 & = & -\dotp{\mke\deltadq}{\deltadq} + 2\dotp{\mke\dqposle}{\deltadq} = -\dotp{\mke\deltadq}{\deltadq}
\end{eqnarray*}

таким образом,
\begin{equation*}
\Delta\ke = -\ddfrac{1}{2}\dotp{\mke\deltadq}{\deltadq} < 0
\end{equation*}
