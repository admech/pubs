
\subsection{Изменение кинетической энергии}

\begin{itemize}
    \item проверка, что оба способа дают один результат
    \item проверка, что результат соответствует теореме Карно (потеря энергии равна энергии потерянных скоростей)
\end{itemize}

Кинетическая энергия в системе без неголономных связей -- квадратичная форма.
\begin{equation*}
\eqTquad
\end{equation*}

Связи
\begin{equation*}
\eqDqVnu
\end{equation*}

В силу идеальности неголономных связей, потеря энергии при ударе равна кинетической энергии потерянных скоростей, в соответствии с теоремой Карно.
Воспользуемся следующими фактами: $\edQposleDelta$, $\eqDqposleInSubspace$, $\eqDeltaOrth$, $\edMkeSim$, работа ударных испульсов на перемещениях, допускаемых вновь налагаемыми связями, равна нулю $\logicWorkZero$
\begin{eqnarray*}
\logicDeltaT
\end{eqnarray*}

таким образом,
\begin{equation*}
\eqDeltaT
\end{equation*}

