\subsection{Способ 1. Классическая механика}

Мотивация -- получить величины реакций.

\begin{itemize}
    \item введение реакций в контакте, рисунки
    \item составление линейной системы, ранг
    \item решение системы
\end{itemize}

Уравнение изменения импульса в момент удара
\begin{equation*}
\eqDeltaqQ
\end{equation*}

Ударные импульсы выражаются через реакции, действующие в точках касания
\begin{equation*}
\eqQKF
\end{equation*}

Непосредственно перед ударом неголономные связи, запрещающие проскальзывание в точках касания роликов, все еще находящихся в контакте в этот момент, снимаются.
В момент сразу после удара аналогичные связи налагаются на вновь входящие в контакт ролики.
\begin{equation*}
\eqqnu
\end{equation*}

Тогда уравнение изменения импульса можно записать в виде
\begin{equation*}
\eqMVnuKF
\end{equation*}

Либо в матричной форме
\begin{equation*}
\eqMVnuKFmat
\end{equation*}

Решение тогда получается следующим образом
\begin{equation*}
\eqMVnuKFmatres
\end{equation*}
где матрица $\mvk$ обратима в силу геометрии системы (ИЛИ ПОЧЕМУ!? вроде можно показать, что имеет максимальный ранг)


