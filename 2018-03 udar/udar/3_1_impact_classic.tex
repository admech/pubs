\subsection{Способ 1. Классическая механика}

Мотивация -- получить величины реакций.

\begin{itemize}
    \item введение реакций в контакте, рисунки
    \item составление линейной системы, ранг
    \item решение системы
\end{itemize}

Уравнение изменения импульса в момент удара
\begin{equation*}
\mathbf{M} (\dot{\mathbf{q}}^+ - \dot{\mathbf{q}}^-) = \mathbf{Q}
\end{equation*}

Ударные импульсы выражаются через реакции, действующие в точках касания
\begin{equation*}
\mathbf{Q} = \mathbf{K}\mathbf{F}
\end{equation*}

Непосредственно перед ударом неголономные связи, запрещающие проскальзывание в точках касания роликов, все еще находящихся в контакте в этот момент, снимаются.
В момент сразу после удара аналогичные связи налагаются на вновь входящие в контакт ролики.
\begin{equation*}
\dot{\mathbf{\nu}}^- = \mathbf{V}\mathbf{\nu}^+
\end{equation*}

Тогда уравнение изменения импульса можно записать в виде
\begin{equation*}
\mathbf{M}\mathbf{V}\mathbf{\nu}^+ - \mathbf{K}\mathbf{F} = \mathbf{M}\dot{\mathbf{q}}^-
\end{equation*}

Либо в матричной форме
\begin{equation*}
\left(\mathbf{M}\mathbf{V} \enspace \mathbf{K}\right) \left(\mathbf{\nu}^+ \enspace \mathbf{F}\right)^T = \mathbf{M}\dot{\mathbf{q}}^-
\end{equation*}

Решение тогда получается следующим образом
\begin{equation*}
\left(\nuposle \enspace \mathbf{F}\right)^T = \inv{\mke\cstr \enspace \mathbf{K}} \mke\dqdo
\end{equation*}
где матрица $\left(\mathbf{M}\mathbf{V} \enspace \mathbf{K}\right)$ обратима в силу геометрии системы (ИЛИ ПОЧЕМУ!? вроде можно показать, что имеет максимальный ранг)
