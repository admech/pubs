
\section{Введение}

Омни-колесо -- это колесо особой конструкции, способное двигаться вдоль опорной поверхности не только за счет вращения вокруг собственной оси в направлении, лежащем в плоскости колеса, но и в направлении, перпендикулярном этой плоскости. Таким свойством они обладают благодаря роликам, располагаемым вдоль колеса. Оси роликов касаются средней плоскости колеса, и ролики свободно вращаются вокруг своих осей. Используется и другой вариант конструкции (\textit{mecanum wheels}), в котором оси роликов составляют с плоскостью колеса некоторый фиксированный угол, как правило, $\ddfrac{\pi}{4}$. Экипаж с омни-колесами способен двигаться в произвольном направлении, не поворачиваясь вокруг вертикали, и не поворачивая вокруг вертикали колеса, то есть обладает повышенной маневренностью.

Исследования динамики экипажей с омни-колесами обычно следуют одному из двух подходов: либо не учитывается динамика роликов, и колеса моделируются как диски, способные скользить в заданном направлении \cite{Bezinercionnaya}, либо применяются формализмы для построения компьютерных моделей систем тел \cite{Compjuternaya}. В первом случае не учитываются эффекты, связанные с собственным вращением роликов, а во втором невозможен непосредственный анализ уравнений движения системы. Уравнения движения симметричного экипажа по абсолютно шероховатой плоскости с учетом динамики роликов получены в \cite{ZobovaGerasimovPMM}. При рассмотрении динамики роликов отдельного внимания заслуживает момент перехода колеса с одного ролика на другой, поскольку вращение ролика, входящего в контакт, может не быть согласовано с условием отсутствия скольжения в контакте.

В данной работе проведено детальное рассмотрение момента смены ролика в контакте с учетом ударного характера взаимодействия с опорной плоскостью. Также, получены численные решения, состоящие из участков, определяемых уравнениями движения, и моментов смены контакта, моделируемых с точки зрения теории удара.