\documentclass{article}
\usepackage[russian]{babel}
\usepackage[utf8]{inputenc}
\usepackage{amsmath,amsfonts,amssymb,amscd,amsthm}
\newcommand{\bs}[1]{\boldsymbol{#1}}

\title{Удар во время смены роликов в контакте при движении одного омни-колеса по шероховатой плоскости}
\date{November 2017}

\usepackage{natbib}
\usepackage{graphicx}

\begin{document}

\maketitle

\section{Постановка задачи}
Рассмотрим движение одного свободного омниколеса по абсолютно шероховатой плоскости (всего экипажа нет). На колесо действует сила тяжести, а также реакция связи, приложенная в наинизшей точке колеса.

Примем следующие предположения:
\begin{itemize}
\item Время удара $\Delta t$ мало настолько, что за время удара координаты системы не меняются. При этом скорости меняются на конечную величину.
\item Изменение скоростей происходит вследствие действия реакции связи, сводящейся к нормальной реакции и силе трения, действующей на входящий в контакт ролик. Будем считать, что момент сил трения относительно граничной точки входящего в контакт ролика равен нулю. Сила трения имеет обе горизонтальные компоненты -- вдоль плоскости колеса и перпендикулярно ей.
\item Все время движения колеса на его диск действует некоторый горизонтальный момент, сохраняющий плоскость диска колеса вертикальной.
\item Ролик усечен, так что вектор, идущий из центра ролика в точку контакта, не коллинеарен оси ролика (это необходимо, чтобы сила трения смогла раскрутить ролик до нужной нам скорости)
\item Трения в оси ролика нет
\item Трения в оси колеса нет
\item За время $\Delta t$ угловая скорость вторго ролика и скорость собственного вращения колеса меняются так, что проскальзывание в точке контакта исчезает (то есть в момент окончания удара становятся выполнеными дифференциальные уравнения связи. В течение удара $[0,\Delta t)$ они нарушены).
\end{itemize}


Теорема об изменении импульса всего колеса
$$
M\frac{d\bs{v}_C}{dt} = \bs{R} + m\bs{g}
$$
дает
$$
M(\dot{z}_C^+ - \dot{z}_C^-) = N-mg\Delta t
$$
\begin{equation}
    M(\nu_1^+ - \nu_1^-) = R_\xi
\end{equation}
\begin{equation}
M(\nu_2^+ - \nu_2^-) = R_\eta
\end{equation}

$(\bs{M},\bs{e}_z) = 0$, $(\bs{M},\bs{e}_\eta) = 0$ -- второе предположение
\begin{equation}
(\frac{d\bs{K}_C}{dt},\bs{e}_z) = (mom_C R,\bs{e}_z) = 0\Longrightarrow \nu_3^+ - \nu_3^- = 0
\end{equation}
\begin{equation}
(\frac{d\bs{K}_C}{dt},\bs{e}_\eta) = (mom_C R,\bs{e}_\eta) \Longrightarrow J_1(\dot{\chi}^+ - \dot{\chi}^-) = -r R_\xi
\end{equation}

$\bs{K}_S=\lambda\dot{\phi}_2$ -- кинетический момент входящего в контакт ролика, $\bs{e}$ --- его ось. $S$ -- центр ролика, $P$ -- точка на границе ролика.
\begin{equation}
\lambda(\dot{\phi}_2^+-\dot{\phi}_2^-) = ([\bs{SP},\bs{R}],\bs{e}) = dR_\eta\sin(?)
\end{equation}

До контакта известны $\nu_1^-, \nu_2^-, \nu_3^-, \dot\chi^-, \dot\phi_2^-$. 
Из кинематических связей после контакта известны $\dot{\chi}^+, \dot{\phi}_2^{+}$ как функции $\nu_1^{+}, \nu_2^{+}, \nu_3^{+}$.

$$
\Delta T = ? 
$$
Неизвестные: $R_\xi, R_\eta, \nu_1^{+}, \nu_2^{+}, \nu_3^{+}$





Вариант Кирилла:
$$
m \Delta \vec{v_C} = \vec{R} + \vec{N} + \vec{F_{тр}},
$$
$$
J \Delta \vec{\omega} = \vec{CK} \times( \vec{N} + \vec{F} ) + M_1 + M_2
$$
 -- относительно центра ролика,
 
\newpage

\textbf{Удар трением, если кинетическая энергия -- квадратичная форма с постоянной матрицей}

Пусть
$$ T = \frac{1}{2}\mathbf{\nu}^T M \mathbf{\nu} $$
Тогда уравнения движения:
$$ M\dot{\mathbf{\nu}} = \mathbf{Q}\delta(t) $$
интегрируем:
$$ \nu^+ = \nu^- + M^{-1}\mathbf{Q} $$
смотрим, что с энергией:
$$ \Delta T = ... = \frac{1}{2}\mathbf{Q}^T \left( M^{-1}Q + 2\nu^{-} \right) $$

Таким образом, энергия убывает тогда и только тогда, когда $Q$ направлены против скоростей, но не слишком велики.

Если рассмотреть движение точки по наклонной плоскости с шероховатым участко под действием силы тяжести, можно заметить, что в момент достижения шероховатости точка испытывает удар трением $F$:
$$ m\Delta\dot{x} = F $$
и изменение энергии:
$$ \Delta T = F(\frac{F}{2} + \dot{x}^-), $$
т.е.
$$ \Delta T < 0 <=> 0 > F > -2\dot{x}^-, $$
и получается, что если начальная скорость близка к нулю, либо если сила трения зависит пропорциональня массе и масса велика, то энергия совершенно спокойно увеличится. Значит ли это, что ударов трением Кулона не бывает !?

\newpage

\textbf{Перевод аннотации}


  
\end{document}


\newpage

