\begin{adjustwidth}{1cm}{1cm}
\hspace{1cm} Рассматривается динамика симметричного экипажа с роликонесущими колесами, движущегося по  неподвижной горизонтальной абсолютно шероховатой плоскости в следующих предположениях: масса каждого ролика ненулевая, контакт между роликами и плоскостью точечный, проскальзывания нет. Уравнения движения, составленные с помощью системы символьных вычислений Maxima, содержат дополнительные члены, пропорциональные осевому моменту инерции ролика и зависящие от углов поворота колес. Масса роликов учитывается в тех фазах движения, когда не происходит смены роликов в контакте. При переходе колес с одного ролика на другой масса роликов считается пренебрежимо малой. Показано, что ряд движений, существующих в безынерционной модели (т.е. не учитывающей массу роликов), пропадает, так же как и линейный первый интеграл. Проведено сравнение основных типов движения симметричного трехколесного экипажа, полученных численным интегрированием уравнений движения с результатами, полученными на основании безынерционной модели.
\end{adjustwidth} 
\textit{Ключевые слова:} омни-колесо, массивные ролики, неголономная связь, лаконичная форма уравнений движения Я.В. Татаринова
\vspace{1.5cm}