\RequirePackage{fix-cm}

\documentclass[12pt]{extarticle}

% \smartqed  % flush right qed marks, e.g. at end of proof

\usepackage{amsmath}
\usepackage{amsfonts}

% Specially for PMM: to have straight (upright) greek letters for vectors
%--------------------------------------
% \usepackage{upgreek}
% \usepackage[artemisia]{textgreek}
\usepackage[euler]{textgreek} %use upsilon instead of nu
%--------------------------------------

% Russian-specific packages
%--------------------------------------
\usepackage[T2A]{fontenc}
\usepackage[utf8]{inputenc}
\usepackage[russian]{babel}
%--------------------------------------

% Asymptote for pictures
%--------------------------------------
\usepackage{asymptote} %% comes with options inline and attach
%--------------------------------------

% graphicx for graphs
%--------------------------------------
\usepackage{graphicx}
%--------------------------------------

% so that it was possible to fix a figure's placement with [H]: \begin{figure}[H]
%--------------------------------------
\usepackage{float}
%--------------------------------------

% to put Fig.N on the margins
%-------------------------------------
\usepackage{marginnote}
\reversemarginpar
%-------------------------------------


%--------------------------------------
% Specially for PMM: make all imported EPS grayscale:
%--------------------------------------
\usepackage[gray]{epspdfconversion}
%--------------------------------------

% \usepackage{subfig} % incompatible with subcaption package
% \graphicspath{ % not used here
    % {./pic/,./asy/}
% }
%--------------------------------------

% subcaption for many figures under one big caption
% each having its own small caption
%--------------------------------------
%\usepackage{caption}
%\usepackage{subcaption}
%--------------------------------------

% so that refs were [1-10], not [1,2,3,4,5,...]
%--------------------------------------
\usepackage{cite}
%--------------------------------------

% to get \bigstar
%--------------------------------------
\usepackage{amssymb}
%--------------------------------------

% \ddfrac command to show big fractions, not cramped up
% https://tex.stackexchange.com/questions/173899/
%--------------------------------------
\newcommand\ddfrac[2]{\displaystyle\frac{\displaystyle #1}{\displaystyle #2}}
%--------------------------------------

% \vsp command to make a spacey newline
% useful for equations arrays
%--------------------------------------
\newcommand\vsp[1][10]{\\[#1pt]}
%--------------------------------------

% to customize itemize
%--------------------------------------
\usepackage{enumitem}
%--------------------------------------

% to set text width
%--------------------------------------
% \usepackage{geometry}
\usepackage{changepage}
%--------------------------------------



% partial derivatives (can \usepackage{physics}, but only one command so far, so no)
%--------------------------------------
\newcommand\pd[2]{\frac{\partial #1}{\partial #2}}
\newcommand\ddpd[2]{\ddfrac{\partial #1}{\partial #2}}
\newcommand\ddt[1]{\frac{d #1}{dt}}
\newcommand\ddddt[1]{\ddfrac{d #1}{dt}}
%--------------------------------------

% unbreakable space parenthesized reference
%--------------------------------------
\newcommand\upr[1]{~(\ref{#1})}
%--------------------------------------

% Nice letters
%--------------------------------------
\newcommand\M[0]{\mathcal{M}} % Matrix of intertia
\newcommand\const{\mathrm{const}} %константа
\newcommand\AntiU[0]{\mathcal{U}} % Helper antisymmetric matrix for eqs' RHS
\newcommand\Rhs[0]{\mathcal{R}} % RHS
\newcommand\Prhs[0]{\mathcal{P}} % The family of matrices for RHS
\newcommand\prhs[0]{\mathbf{p}} % Poisson brackets
\newcommand\vnu[0]{\text{\textbf{\textupsilon}}} % Upright greek vector nu for PMM
%--------------------------------------

% change line spacing mid doc (affects global line spacing)
%--------------------------------------
% \usepackage{setspace}
%--------------------------------------


\renewcommand{\vec}[1]{\boldsymbol{\mathbf{#1}}}
%\renewcommand{\figurename}{Фиг.}
\usepackage[labelsep=period]{caption}
\addto\captionsrussian{\renewcommand{\figurename}{Фиг. }}

\newtheorem{stmt}{Утверждение}
\newtheorem{prblm}{Затруднение}

% biblio hacks -- noindent bibitems
\makeatletter
\renewenvironment{thebibliography}[1]
      {\section*{\refname}%
      \@mkboth{\MakeUppercase\refname}{\MakeUppercase\refname}%
      \list{\@biblabel{\@arabic\c@enumiv}}%
            {\settowidth\labelwidth{\@biblabel{#1}}%
             \leftmargin\labelwidth
             \advance\leftmargin-25pt% change 20 pt according to your needs
             \advance\leftmargin\labelsep
             \setlength\itemindent{25pt}% change using the inverse of the length used before
             \@openbib@code
             \usecounter{enumiv}%
             \let\p@enumiv\@empty
             \renewcommand\theenumiv{\@arabic\c@enumiv}}%
      \sloppy
      \clubpenalty4000
      \@clubpenalty \clubpenalty
      \widowpenalty4000%
      \sfcode`\.\@m}
      {\def\@noitemerr
        {\@latex@warning{Empty `thebibliography' environment}}%
      \endlist}
\renewcommand\newblock{\hskip .11em\@plus.33em\@minus.07em}
\makeatother


\voffset=-15mm \textwidth=17cm \textheight=24cm
\oddsidemargin=0cm \topmargin=+0cm \headsep=10pt \evensidemargin=0mm
\renewcommand{\baselinestretch}{2}

\makeatletter \@addtoreset{equation}{section} \makeatother
\makeatletter

% bibliography hacks
\renewcommand{\@biblabel}[1]{#1. \hfill}
\makeatother
\addto\captionsrussian{\def\refname{Литература}}
\renewcommand{\refname}{}

\renewcommand{\thesection}{\arabic{section}}
\renewcommand{\theequation}{\arabic{section}.\arabic{equation}}


\begin{document}

% ------------------------------------------------------
\section{Введение}

Известны модели либо неголономные без роликов, либо с роликами и трением, но без уравнений.
Построим неголономную модель с роликами и понятными уравнениями, но и с реалистичной сменой контакта.

% ------------------------------------------------------
\section{Постановка задачи}

\begin{itemize}
    \item рисунки, обозначения и предположения
    \item уравнения движения
    \item усечение и перекрытие роликов
    \item снятие связей перед сменой контакта
    \item эффект рамки и ожидания об убывании энергии
\end{itemize}

% ------------------------------------------------------
\section{Удар}

% ------------------------------------------------------
\subsection{Способ 1. Классическая механика}

Мотивация -- получить величины реакций.

\begin{itemize}
    \item введение реакций в контакте, рисунки
    \item составление линейной системы, ранг
    \item решение системы
\end{itemize}

% ------------------------------------------------------
\subsection{Способ 2. Аналитическая механика}

Мотивация -- строго определяется матрицами кинетической энергии и связей, не требует произвольного введения векторов реакций.

\begin{itemize}
    \item постановка задачи теории удара как проецирования вектора обобщенных скоростей на плоскость в пространстве виртуальных перемещений, определяемую вновь налагаемыми связями
    \item выражение для скоростей после удара
\end{itemize}

% ------------------------------------------------------
\subsection{Изменение кинетической энергии}

\begin{itemize}
    \item проверка, что оба способа дают один результат
    \item проверка, что результат соответствует теореме Карно (потеря энергии равна энергии потерянных скоростей)
\end{itemize}

% ------------------------------------------------------
\section{Примеры движений}

\begin{itemize}
    \item вокруг себя
    \item по прямой
    \item с закруткой
\end{itemize}

Обратить внимание на монотонное убывание кинетической энергии.

% ------------------------------------------------------
\section{Бонус: реакции}

Получить, в каких конусах остаются реакции и сравнить с конусами трения Кулона.

% ------------------------------------------------------
\section{Бонус: время движения системы}

Энергия всякий раз убывает на ненулевую конечную величину из-за эффекта рамки. Оценить время движения или количество ``шагов'', исходя из значений этой величины.

\end{document}

