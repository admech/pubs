
\chapter*{Выводы}

\begin{enumerate}

\item
Получены уравнения движения экипажа на омни-колесах по абсолютно шероховатой плоскости с учетом инерции роликов.

\item
Изучены свойства этих уравнений движения и проведено сравнение их с уравнениями движения безынерционной модели~\cite{Zobova2011}.

\item
Построен способ расчета изменения обобщенных скоростей при смене ролика в контакте, в предположении о мгновенном выполнении условия отсутствия проскальзывания между роликом и опорной плоскостью.

\item
Получены численные решения для симметричной конфигурации экипажа с омни-колесами с учетом ударного взаимодействия роликов и опорной плоскости.

\item
Построена динамическая модель экипажа с омни-колесами на плоскости с регуляризованным сухим трением. Показана возможность гладкого безударного переключения роликов в контакте в процессе качения/скольжения омни-колеса. Модель описывает оба варианта омни-колес: обыкновенные, с осями роликов в плоскости колеса, и \textit{mecanum}, где оси роликов повернуты вокруг радиус-векторов их центров.

\item
Выполнена верификация динамической модели омни-экипажа с использованием безынерционной модели, рассмотренной в работе~\cite{Borisov2011}, в качестве предельного случая (когда суммарная масса роликов равна нулю).

\end{enumerate}
