
\chapter*{Результаты, выносимые на защиту}

\begin{enumerate}

        \item {
            Построены модели экипажа с омни-колесами, движущегося по горизонтальной плоскости по инерции: неголономная с идеальными связями и голономная с неидеальными. Обе модели учитывают инерцию роликов омни-колес.
        }
        \item {
            Первая модель получена в предположении, что ролик омни-колеса не проскальзывает относительно плоскости (связи идеальны). Уравнения движения на гладких участках  (т.е. между сменой ролика в контакте) получены аналитически в псевдоскоростях и представляют собой уравнения 33 порядка для экипажа с 3 колесами и 5 роликами на каждом колесе. С помощью теории удара расчет изменения обобщенных скоростей при смене ролика в контакте сведен к решению системы линейных алгебраических уравнений, имеющей единственное решение в т.ч. при кратном ударе.
        }
        \item {
            % Обоснована корректность неголономной модели с массивными роликами. Для этого а
            Aналитически показано, что при равенстве осевого момента инерции ролика нулю, её уравнения движения совпадают с уравнениями движения безынерционной модели~\cite{Zobova2011}.
        }
        \item {
            Показано, что линейный первый интеграл, существующий в безынерционной модели, разрушается при осевом моменте инерции, отличном от нуля. При этом скорость изменения значения этого интеграла пропорциональна осевому моменту инерции ролика. Найдены линейные интегралы, связывающие угловую скорость платформы экипажа и скорости собственного вращения роликов, не находящихся в контакте.
        }
        \item {
            В ходе численных экспериментов обнаружен эффект быстрого убывания скорости центра масс по сравнению с угловой скоростью платформы экипажа.
            % Данный эффект противоположен эффекту быстрого убывания скорости верчения шара по сравнению со скоростью его центра при движении по шероховатой плоскости.
        }
        \item {
            Модель с неидеальными голономными связями реализована в системе автоматического построения численных динамических моделей для вязкого трения и для регуляризованного сухого трения. Для этого найдены геометрические условия контакта роликов омни- и mecanum-колес и опорной плоскости.
        }
        \item {
            % Обоснована корректность динамической модели с трением. Для этого ч
            Численно показано, что при стремлении осевого момента инерции ролика к нулю, движения системы с трением стремятся к движениям безынерционной модели. Обнаружено качественное сходство траекторий системы с вязким трением с достаточно большим коэффициентом трения с движениями модели, рассмотренной в главах 1 и 2.
        }
        % \item {
        %     Численные эксперименты показали, что проскальзывание между роликом в контакте и опорной плоскостью заканчивается за время, существенно меньшее, чем время нахождения ролика в контакте. Это служит обоснованием применимости теории удара в неголономной модели. Также обнаружено, что время проскальзывания тем меньше, чем меньше осевой момент инерции ролика.
        % }
        \item {
            Как-то красиво -- о единстве трех разных моделей экипажа: с ударами, с сухим трением и с вязким трением.
        }
        
% \item
% Получены уравнения движения экипажа на омни-колесах по абсолютно шероховатой плоскости с учетом инерции роликов.

% \item
% Изучены свойства этих уравнений движения и проведено сравнение их с уравнениями движения безынерционной модели~\cite{Zobova2011}.

% \item
% Построен способ расчета изменения обобщенных скоростей при смене ролика в контакте, в предположении о мгновенном выполнении условия отсутствия проскальзывания между роликом и опорной плоскостью.

% \item
% Получены численные решения для симметричной конфигурации экипажа с омни-колесами с учетом ударного взаимодействия роликов и опорной плоскости.

% \item
% Построена динамическая модель экипажа с омни-колесами на плоскости с регуляризованным сухим трением. Показана возможность гладкого безударного переключения роликов в контакте в процессе качения/скольжения омни-колеса. Модель описывает оба варианта омни-колес: обыкновенные, с осями роликов в плоскости колеса, и \textit{mecanum}, где оси роликов повернуты вокруг радиус-векторов их центров.

% \item
% Выполнена верификация динамической модели омни-экипажа с использованием безынерционной модели, рассмотренной в работе~\cite{Borisov2011}, в качестве предельного случая (когда суммарная масса роликов равна нулю).

\end{enumerate}
