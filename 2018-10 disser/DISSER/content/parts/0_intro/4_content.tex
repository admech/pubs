
СОДЕРЖАНИЕ РАБОТЫ

Настоящая работа состоит из введения, трех глав, заключения и списка литературы.

Во введении дано описание предметной области и цели данной работы, выполнен обзор литературы об омни-колесах и экипажах, оснащенных ими, о динамике систем тел, в том числе, с односторонними связями, ударами и трением, а также составлено краткое содержание работы.

Целью работы является изучение неуправляемого движения роликонесущего экипажа по горизонтальной плоскости с учетом инерции роликов и трения в двух постановках. В первой постановке опорная плоскость абсолютно шероховата, т.е. проскальзывание между роликом в контакте и плоскостью отсутствует. При этом предполагается, что при смене ролика в контакте происходит мгновенное согласование скоростей системы в соответствии с новыми связями (удар связями). Во второй между контактным роликом и опорной плоскостью действует сила сухого трения скольжения Кулона, либо сила вязкого трения скольжения. 

Экипаж состоит из несущей платформы и трех одинаковых омни-колес, центры которых расположены в вершинах правильного треугольника, а плоскости препендикулярны биссектрисам соответствующих углов. Омни-колесо моделируется абсолютно твердым диском и некоторым количеством распределенных по его окружности весомых роликов, свободно вращающихся вокруг своих осей, направленных вдоль касательных к окружности.

В первой и второй главе работы рассматривается движение  экипажа по абсолютно  шероховатой плоскости, т.е. предполагается, что проскальзывание между опорным роликом и плоскостью отсутствует. Уравнения движения получены в явном виде с использованием формализма лаконичных уравнений Я.В. Татаринова. Изучена структура уравнений, найдены первые интегралы и проведено сравнение с уравнениями движения безынерционной модели. Показано, что если момент инерции ролика относительно его оси равен нулю, то уравнения совпадают с уравнениями безынерционной модели. Результаты первой главы изложены в \cite{GerasimovZobovaPMM2018}.

Во второй главе рассмотрена задача о смене ролика в контакте: при повороте колеса вокруг своей оси в контакт с плоскостью приходит новый ролик, скорость которого, вообще говоря, не согласована со связями, при этом возникает проскальзывание. Предполагается, что проскальзывание прекращается за бесконечно малый промежуток времени. Составлены линейные алгебраические уравнения, определяющие обобщенные скорости после смены ролика в контакте в соответствии с теорией удара. Таким образом, численное моделирование движения экипажа состоит из решения задачи Коши уравнений, полученных в первой главе, пока в контакте находится один и тот же ролик, и решения линейных алгебраических уравнений при смене ролика для получения начальных условий для следующего гладкого участка. Получены и проанализированы численные решения для симметричной конфигурации экипажа. По результатам второй главы опубликована статья  \cite{GerasimovZobovaTrudyMAI2018}.

В третьей главе построена динамическая модель экипажа на плоскости с сухим трением Кулона-Амонтона, регуляризованным в окрестности нуля по скоростям участком линейной функции насыщения с достаточно большим угловым коэффициентом. Особое внимание уделяется вопросу моделирования связи в контакте ролика и горизонтальной плоскости, отслеживанию точки контакта ролика и опорной плоскости, а также алгоритмической реализации процесса переключения контакта от ролика к ролику при качении омни-колеса. Динамическая модель построена в формализме объектно-ориентированного моделирования на языке \texttt{Modelica}. Выполнена верификация динамической модели с использованием безынерционной модели. Также строится модель экипажа на плоскости с вязким трением. Проводится качественное сравнение данной модели с моделью экипажа на абсолютно шероховатой плоскости, построенной в первой и второй главах. Результаты третьей главы опубликованы в работах \cite{KosenkoGerasimovNd2016,KosenkoGerasimovJsme2016}, а также доложены на конференциях и опубликованы в их сборниках трудов \cite{Kosenko2014unilateral,KosenkoGerasimov2014,Kosenko201construction,Kosenko2015verification,Kosenko2015hierarchy,KosenkoGerasimov2015,Kosenko2016testbench}.

В заключении перечислены основные положения, выносимые на защиту.

% По результатам работы опубликованы и приняты к печати в рецензируемых журналах, реферируемых в международных базах WebOfScience, Scopus, RSCI и входящих в список ВАК, следующие статьи \cite{KosenkoGerasimovNd2016,GerasimovZobovaPMM2018,GerasimovZobovaTrudyMAI2018,KosenkoGerasimovJsme2016,KosenkoGerasimov2015,Kosenko201construction}

% Результаты докладывались автором на ряде международных и всероссийских конференциях: Международная конференция по дифференциальным уравнениям и динамическим системам 2018, Суздаль, Россия, 6-11 июля 2018; 20-е Международное рабочее совещание по компьютерной алгебре (Дубна, 21-22 мая 2018), Дубна, Россия, 21-22 мая 2018; Ломоносовские Чтения - 2018, МГУ имени М.В. Ломоносова, Россия, 16-25 апреля 2018; Ломоносовские чтения - 2017, МГУ имени М.В. Ломоносова, Россия, 17-26 апреля 2017; 11th International Modelica Conference Versailles, France, September 21-23, 2015. Результаты также были представлены аспирантом на семинаре Аналитическая механика и теория устойчивости (имени В.В. Румянцева) под руководством д.ф.-м.н. проф. А.В. Карапетяна в 2017 и 2018 г. и на семинаре Динамика относительного движения под руководством д.ф.-м.н. проф. В.Е. Павловского на механико-математическом факультете МГУ им. М.В. Ломоносова в 2018 г.

% По результатам работы опубликованы в рецензируемых журналах, индексируемых в международных базах WebOfScience, Scopus и RSCI, следующие статьи:

\begin{enumerate}
    \item Косенко И.И., Герасимов К.В. Физически-ориентированное моделирование динамики омнитележки // Нелин. дин. 2016. Т.~12, No~2. С.~251--262 (и.-ф. РИНЦ 0,394)
    \item Герасимов К.В., Зобова А.А. Движение симметричного экипажа на омни-колесах с массивными роликами // ПММ. 2018. Т.~82, No~4., стр. 427--440 (и.-ф. РИНЦ 0,831)
    \item Kosenko I.I., Gerasimov K.V. Object-oriented approach to the cons\-truc\-tion of an omni vehicle dynamical model // Journal of Mechanical Science and Technology. — 2015. — Vol.~29, No.~7. — P.~2593--2599 (SJR 0,55)
\end{enumerate}
Опубликована статья в журнале, входящем в список ВАК:
\begin{enumerate}
    \addtocounter{enumi}{3}
    \item Герасимов К.В., Зобова А.А. Динамика экипажа на омни-ко\-ле\-сах с массивными роликами с учетом смены ролика в контакте с опорной плоскостью // Труды МАИ. 2018. No 101. (и.-ф. РИНЦ 0,445)
\end{enumerate}
Также опубликованы статьи в сборниках трудов международных конференций, включенных в международные базы Scopus либо Web Of Science:
\begin{enumerate}
    \addtocounter{enumi}{4}
    \item Kosenko I.I., Stepanov S.Y., Gerasimov K.V. Improved contact tracking algorithm for the omni wheel in general case of roller orientation // The Proceedings of the Asian Conference on Multi\-bo\-dy Dynamics. 2016.8. The Japan Society of Mechanical Engineers. — 2017. — no. July 01 — P. 2424-2985
    \item Kosenko I.I., Gerasimov K.V. Omni vehicle dynamics model: Object-oriented implementation and verification // Proceedings of the International Conference on Numerical Analysis and Applied Mathe\-ma\-tics 2014 (ICNAAM-2014), volume 1648 of AIP Conference Pro\-cee\-dings, College Park, Md., United States — 2015 — P. 1–4.
    \item Kosenko I. I., Gerasimov K. V., Stavrovskiy M. E. Contact types hierarchy and its object-oriented implementation // B. Schrefler, E. Onate and M. Papadrakakis (Eds), Proceedings of the VI Inter\-na\-tional Conference on Coupled Problems in Science and Engineering, San Servolo, Venice, Italy, May 18–20, 2015. — 2015. — P. 191–202.
\end{enumerate}
Результаты докладывались соискателем на ряде международных и всероссийских конференциях:
\begin{enumerate}
    \item Международная конференция по дифференциальным уравнениям и динамическим системам 2018, Суздаль, Россия, 6-11 июля 2018
    \item Двадцатое международное рабочее совещание по компьютерной алгебре, Дубна, Россия, 21-22 мая 2018
    \item Ломоносовские Чтения - 2018, МГУ имени М.В. Ломоносова, Россия, 16-25 апреля 2018
    \item Ломоносовские чтения - 2017, МГУ имени М.В. Ломоносова, Россия, 17-26 апреля 2017
    \item 11th International Modelica Conference, Версаль, Франция, 21-23 сентября 2015
\end{enumerate}
Результаты также были представлены диссертантом на следующих научных семинарах механико-математического факультета МГУ им. М.В. Ломоносова:
\begin{enumerate}
    \item Семинар по аналитической механике и теории устойчивости имени В.В. Румянцева под руководством д.ф.-м.н. проф. А.В.~Карапетяна (2017, 2018 г.)
    \item Семинар имени В.В.Белецкого по динамике относительного движения под руководством д.ф.-м.н. проф. Ю.Ф. Голубева, д.ф.-м.н. проф. В.Е.Павловского, к.ф.-м.н. доц. К.Е. Якимовой, к.ф.-м.н. доц. Е.В. Мелкумовой в 2018 г.
\end{enumerate}


