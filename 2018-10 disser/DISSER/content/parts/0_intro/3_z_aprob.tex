По результатам работы опубликованы в рецензируемых журналах, индексируемых в международных базах WebOfScience, Scopus и RSCI, следующие статьи:

\begin{enumerate}
    \item Косенко И.И., Герасимов К.В. Физически-ориентированное моделирование динамики омнитележки // Нелин. дин. 2016. Т.~12, No~2. С.~251--262.
    \item Герасимов К.В., Зобова А.А. Движение симметричного экипажа на омни-колесах с массивными роликами // ПММ. 2018. Т.~82, No~4., стр. 427--440
    \item Kosenko I.I., Gerasimov K.V. Object-oriented approach to the cons\-truc\-tion of an omni vehicle dynamical model // Journal of Mechanical Science and Technology. — 2015. — Vol.~29, No.~7. — P.~2593--2599.
\end{enumerate}
Опубликована статья в журнале, входящем в список ВАК:
\begin{enumerate}
    \addtocounter{enumi}{3}
    \item Герасимов К.В., Зобова А.А. Динамика экипажа на омни-ко\-ле\-сах с массивными роликами с учетом смены ролика в контакте с опорной плоскостью // Труды МАИ. 2018. No 101.
\end{enumerate}
Также опубликованы статьи в сборниках трудов международных конференций, включенных в международные базы Scopus либо Web Of Science:
\begin{enumerate}
    \addtocounter{enumi}{4}
    \item Kosenko I.I., Stepanov S.Y., Gerasimov K.V. Improved contact tracking algorithm for the omni wheel in general case of roller orientation // The Proceedings of the Asian Conference on Multi\-bo\-dy Dynamics. 2016.8. The Japan Society of Mechanical Engineers. — 2017. — no. July 01 — P. 2424-2985
    \item Kosenko I.I., Gerasimov K.V. Omni vehicle dynamics model: Object-oriented implementation and verification // Proceedings of the International Conference on Numerical Analysis and Applied Mathe\-ma\-tics 2014 (ICNAAM-2014), volume 1648 of AIP Conference Pro\-cee\-dings, College Park, Md., United States — 2015 — P. 1–4.
    \item Kosenko I. I., Gerasimov K. V., Stavrovskiy M. E. Contact types hierarchy and its object-oriented implementation // B. Schrefler, E. Onate and M. Papadrakakis (Eds), Proceedings of the VI Inter\-na\-tional Conference on Coupled Problems in Science and Engineering, San Servolo, Venice, Italy, May 18–20, 2015. — 2015. — P. 191–202.
\end{enumerate}
Результаты докладывались соискателем на ряде международных и всероссийских конференциях:
\begin{enumerate}
    \item Международная конференция по дифференциальным уравнениям и динамическим системам 2018, Суздаль, Россия, 6-11 июля 2018
    \item Двадцатое международное рабочее совещание по компьютерной алгебре, Дубна, Россия, 21-22 мая 2018
    \item Ломоносовские Чтения - 2018, МГУ имени М.В. Ломоносова, Россия, 16-25 апреля 2018
    \item Ломоносовские чтения - 2017, МГУ имени М.В. Ломоносова, Россия, 17-26 апреля 2017
    \item 11th International Modelica Conference, Версаль, Франция, 21-23 сентября 2015
\end{enumerate}
Результаты также были представлены диссертантом на следующих научных семинарах механико-математического факультета МГУ им. М.В. Ломоносова:
\begin{enumerate}
    \item Семинар по аналитической механике и теории устойчивости имени В.В. Румянцева под руководством д.ф.-м.н. проф. А.В.~Карапетяна (2017, 2018 г.)
    \item Семинар имени В.В.Белецкого по динамике относительного движения под руководством д.ф.-м.н. проф. Ю.Ф. Голубева, д.ф.-м.н. проф. В.Е.Павловского, к.ф.-м.н. доц. К.Е. Якимовой, к.ф.-м.н. доц. Е.В. Мелкумовой в 2018 г.
\end{enumerate}
