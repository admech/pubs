
\textbf{Методы исследования}

Для получения уравнения движения экипажа по абсолютно шероховатой плоскости используется метод, предложенный Я.В. Татариновым. Рассмотрение перехода колеса с одного ролика на другой проведено с точки зрения теории удара. Решения составленной системы уравнений получены для случая трех колес и пяти роликов на каждом с помощью системы компьютерной алгебры \texttt{Maxima}. При построении модели экипажа на плоскости с сухим трением кинематика вращательного движения твердых тел описывается в алгебре кватернионов. Сила сухого трения Кулона в точечном твердотельном контакте роликов и опорной плоскости регуляризуется, основываясь на теории И.В. Новожилова. Компьютерная модель экипажа с трением построена с помощью технологии объектно-ориентированного моделирования \texttt{Modelica}.
% СТОИТ ЛИ О СЛЕДУЮЩЕМ ??? Верификация данной модели в сравнении с безынерционной моделью проведена, опираясь на теорию А.В. Карапетяна о близости решений неголономных систем и систем с достаточно сильным вязким трением.