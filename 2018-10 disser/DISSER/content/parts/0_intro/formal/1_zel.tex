
\textbf{Цель работы}

Целью работы является изучение неуправляемого движения роликонесущего экипажа по горизонтальной плоскости с учетом инерции роликов и трения в двух постановках. В первой постановке опорная плоскость абсолютно шероховата, т.е. проскальзывание между роликом в контакте и плоскостью отсутствует. При этом предполагается, что при смене ролика в контакте происходит мгновенное согласование скоростей системы в соответствии с новыми связями (удар связями). Во второй между контактным роликом и опорной плоскостью действует сила сухого трения Кулона, либо сила вязкого трения.