
\textbf{Актуальность темы}

Роликонесущие колеса – это колеса особой конструкции, позволяющей экипажу двигаться в произвольном направлении, не поворачиваясь вокруг вертикальной оси. Колеса при этом вращаются лишь вокруг их собственных осей и не поворачиваются вокруг вертикали. Экипажи с такими колесами обладают повышенной маневренностью и используются, например, в качестве погрузочных платформ в авиастроении. Ранее была рассмотрена динамика роликонесущих экипажей с использованием упрощенной модели омни-колес без учета инерции и формы роликов (безынерционная модель). В этой модели колеса  представляют собой жесткие диски, которые могут скользить в одном направлении и катиться без проскальзывания в другом. Динамические эффекты, порождаемые вращением роликов, не учитываются. В другой части работ по динамике роликонесущего экипажа используются  алгоритмы для построения численных моделей систем твердых тел. При этом явный вид уравнений движения оказывается скрытым, что делает невозможным непосредственный анализ уравнений и затрудняет оценку влияния разных факторов на динамику системы. В этом случае авторами также применяются существенные упрощения геометрии роликов, либо их динамика также не учитывается.В настоящей работе динамические эффекты, возникающие из-за собственного вращения роликов, учитываются, и форма роликов приближена к применяемой на практике.