% \chapter*{Введение}
\section*{Введение}

Омни-колеса (в русской литературе также используется название роликонесущие колеса) -- это колеса особой конструкции, позволяющей экипажу двигаться в произвольном направлении, вращая колеса вокруг их собственных осей и не поворачивая их вокруг вертикали. На ободе такого колеса располагаются ролики, которые могут свободно вращаться вокруг своих осей, жестко закрепленных в диске колеса. Существуют два варианта расположения осей роликов: первый  (собственно омни-колеса \cite{formalskii}) -- оси роликов являются касательными к ободу колеса и, следовательно, лежат в его плоскости; второй (\textit{mecanum wheels} \cite{Ilon}) -- оси роликов развернуты вокруг радиусов-векторов их центров на постоянный угол, обычно $\pi/4$.

За счет своей конструкции омни-колесо может катиться вперед или назад, как обыкновенное колесо, кроме того, диск колеса может двигаться поступательно под углом к своей плоскости за счет вращения ролика, контактирующего с опорной поверхностью. Благодаря этому мобильные экипажи, оснащенные такими колесами, обладают повышенной маневренностью и применяются для выполнения различных работ в условиях стесненного пространства, например, для перемещения грузов.
% использовать следующее предложение при 1 интервале
Свойства омни-колес мотивируют инженеров и ученых находить новые и новые их применения.
% , подчас самые неожиданные.
% использовать при недостатке актуальности
% , обнаруживая широкий спектр практических применений -- от промышленных погрузчиков до роботов, предназначенных для освоения труднодоступных территорий.
% использовать следующее предложение при 1 интервале
Сложность их устройства, обусловленная наличием роликов, на протяжении последних десятков лет делает системы с омни-колесами предметом исследований, в которых движение описывается все более точно.

% использовать следующее предложение при 1 интервале
Большинство известных моделей омниколесных экипажей построено в приближении, не учитывающем движение роликов. Как правило, это обусловлено двумя причинами. Во-первых, относительно большое количество твердых тел в полной системе затрудняет аналитическое исследование и замедляет численное.
% Во-вторых, при рассмотрении динамики роликов возникает необходимость моделировать переход колес экипажа с одного ролика на другой.
Во-вторых, при моделировании роликов как отдельных твердых тел, ролик в контакте с опорной плоскостью меняется. В силу этого система уравнений становится не непрерывной, а гибридной. Кроме того, ролик, не находящийся в контакте, может совершать такие движения, что при входе его в контакт возникнет некоторый переходный процесс, который также требуется моделировать.
% использовать следующее предложение при 1.5 интервале
% Большинство известных моделей не учитывает движение роликов, поскольку относительно большое количество твердых тел в полной системе затрудняет аналитическое исследование и замедляет численное, а также требует моделирования фаз движения, в которых колеса экипажа переходят с одного ролика на другой.
Настоящая работа посвящена построению динамических моделей омни-колесного экипажа, учитывающих движение всех роликов.
%, а также описывающих переход с одного ролика на другой.
%В настоящей работе строятся динамические модели полной системы тел, составляющих омни-экипаж.
%, а именно, платформы, $N$ колес и $n$ роликов на каждом колесе. 
При построении таких моделей требуется описать движение большого количества твердых тел%-- $1 + N(n + 1)$
, важен способ задания контакта роликов и опорной поверхности, требуется моделирование перехода колеса экипажа с одного ролика на другой.
Поэтому обзор литературы построен следующим образом: сначала будут описаны работы, непосредственно посвященные исследованиям омни-экипажей; далее будут упомянуты работы по формализмам построения уравнений систем многих твердых тел, близко относящиеся к представленной работе; будут даны необходимые ссылки на работы по теории удара; в конце обзора обсуждаются работы по языку Modelica, используемому в последней главе работы.


% Цель настоящей работы -- изучение движения по инерции экипажа с омни-колесами с массивными роликами. Задача рассматривается в постановке с дифференциальными связями отсутствия проскальзывания в контакте, где составляются уравнения движения Я.В. Татаринова \cite{Tatarinov} и проводится исследование их свойств и сравнение поведения такой системы с поведением системы в безынерционном случае \cite{Zobova2011}, причем моменты перехода колес с одного ролика на другой изучаются с точки зрения теории удара; а также в постановке с трением, где строится численная динамическая модель экипажа и проводится ее верификация в сравнении с безынерционным случаем \cite{borisov}.