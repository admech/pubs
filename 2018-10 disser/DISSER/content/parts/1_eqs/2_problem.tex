\section{Постановка задачи}

\blfootnote{Данная глава основана на статье \cite{GerasimovZobovaPMM2018}.}

Рассмотрим экипаж с омни-колесами, движущийся по инерции по неподвижной абсолютно шероховатой горизонтальной опорной плоскости в поле силы тяжести. 

Омни-колесо -- это система абсолютно твердых тел, включающая в себя плоский диск колеса и $n$ массивных роликов. Схема колеса приведена на фиг.~\ref{fig:wheel}; ролики на ней показаны в виде затемненной области и областей, ограниченных штриховой линией. Плоскость, содержащую диск колеса будем называть плоскостью колеса. Каждый ролик может свободно вращаться вокруг оси, неподвижной относительно диска колеса. Оси роликов -- это прямые, лежащие в плоскости колеса и касательные к его окружности радиуса $r$. Поверхность ролика является поверхностью вращения дуги окружности радиуса $l > r$, лежащей в плоскости колеса, с центром в центре диска, вокруг хорды, лежащей на оси ролика. Центры $K_j, j = 1 \ldots n$ роликов расположены на окружности диска колеса в вершинах правильного $n$-угольника. Отметим, что при такой конфигурации, если плоскость колеса вертикальна, то расстояние от оси колеса до опорной плоскости постоянно во все время движения, и в частности, при переходе колеса с одного ролика на другой (последнее рассмотрено в главе 2).

Экипаж состоит из платформы и $N$ одинаковых омни-колес. Платформа экипажа -- это плоский диск радиуса $R$. Оси колес неподвижны относительно платформы и лежат в ее плоскости. Таким образом, система состоит из $N(n+1) + 1$ абсолютно твердых тел.

Введем неподвижную систему отсчета $OXYZ$ так, что ось $OZ$ направлена вертикально вверх, а плоскость $OXY$ совпадает с опорной плоскостью. Введем также подвижную систему отсчета $S\xi\eta Z$ с началом в центре $S$ платформы экипажа, жестко связанную с платформой экипажа так, что плоскость $S\xi\eta$ горизонтальна и содержит центры $P_i, i = 1 \ldots N$ дисков всех колес. Будем рассматривать конфигурацию экипажа, в которой оси колес совпадают с прямыми, соединяющими центр $S$ платформы и центры $P_i$ дисков колес (см. фиг.~\ref{fig:vehicle}), расстояния от $P_i$ до $S$ одинаковы и равны $R$, и центры дисков колес расположены в вершинах правильного $N$-угольника. Геометрию установки колес на платформе зададим углами $\alpha_i$ между осью $S\xi$ и осями колес (см. фиг.~\ref{fig:wheel}). Заметим, что при такой конфигурации $\sum\limits_{k} \cos\alpha_k = \sum\limits_{k}\sin\alpha_k = 0$.

Будем считать, что все тела однородны. Тогда центр масс всей системы совпадает с точкой $S$, а центры масс колес -- с центрами их дисков $P_i$.

Для каждого колеса введем также тройку единичных взаимно-ортогональных векторов (ортов), жестко связанных с дисками колес: орт оси $i$-го колеса $\vec{n}_i = \vec{SP}_i/|\vec{SP}_i|$ и орты $\vec{n}_i^\perp$ и $\vec{n}_i^z$, лежащие в плоскости диска колеса. Положения центров $K_j$ роликов относительно дисков колес определим углами $\kappa_j$ между их радиусами-векторами $\overrightarrow{P_iK_j}$ относительно центра колеса $P_i$ и направлением, противоположным вектору $\vec{n}_i^z$. Определим угол поворота колеса $\chi_i$ вокруг его собственной оси $SP_i$ как угол между опорной плоскостью и вектором $\vec{n}_i^\perp$, отсчитываемый против часовой стрелки, если смотреть с конца вектора $\vec{n}_i$, так, что вектор $\vec{n}_i^\perp$ горизонтален при $\chi_i = 0$.

\begin{figure}[htb]
    \centering
    \minipage{0.45\textwidth}
        \centering
        \asyinclude{./content/pic/asy/pic_wheel.asy}
        \caption{Омни-колесо. Индексы $i$, означающие номер колеса, опущены.}
        \label{fig:wheel}
    \endminipage
    \quad
    \minipage{0.45\textwidth}
        \centering
        \asyinclude{./content/pic/asy/pic_cart.asy}
        \caption{Экипаж симметричной конфигурации}
        \label{fig:vehicle}
    \endminipage
\end{figure}

Положение экипажа будем задавать следующими обобщенными координатами:
$x, y$ --- координаты точки $S$ на плоскости $OXY$, $\theta$ -- угол между осями $OX$ и $S\xi$ (будем называть $\theta$ углом курса),
$\chi_i$ ($i = 1,\dots,N$) -- углы поворота колес вокруг их осей, и $\phi_j$ -- углы поворота роликов вокруг их собственных осей.
Таким образом, вектор обобщенных координат имеет вид
\begin{equation}\label{eq:q}
    vec{q} = (
        x, y, \theta,
        \left\{\chi_i\right\}|_{i=1}^N,
        \left\{\phi_k\right\}|_{k=1}^N,
        \left\{\phi_s\right\}|_{s=N + 1}^{Nn}
    )^{\mathop{T}}\in\mathbb{R}^{N(n+1) + 3},    
\end{equation}
 где координаты сгруппированы таким образом, что сначала указаны углы поворота $\phi_k$ роликов, находящихся в данный момент в контакте с опорной плоскостью, a затем -- остальных, ``cвободных'', роликов. Индекс $s$ используется для сквозной нумерации свободных роликов.

Введем псевдоскорости
$$
    \vnu = (\nu_1, \nu_2, \nu_3, \left\{\nu_s\right\}),
$$
\begin{equation}\label{eq:vsnu3}
    \vec{v}_S = R\nu_1\vec{e}_\xi + R\nu_2\vec{e}_\eta, \quad \nu_3 = \Lambda\dot{\theta},
\end{equation}
\begin{equation}\label{eq:nus}
    \nu_s = \dot{\phi}_s, \quad s = N + 1,\ldots,Nn
\end{equation}

Их механический смысл таков: $\nu_1$, $\nu_2$ --- проекции скорости точки $S$ на оси системы $S\xi$ и $S\eta$, связанные с платформой, $\nu_3$ --- проекция угловой скорости платформы на вертикальную ось $SZ$ с точностью до безразмерного множителя $\Lambda$, $\nu_s$ --- угловые скорости свободных роликов. Безразмерный множитель $\Lambda$ вводится для удобства вывода уравнений и равен $\ddfrac{I_S}{M l^2}$, где $I_S = \const$ -- момент инерции всей системы относительно оси $SZ$, $M$ -- масса всей системы. Число независимых псевдоскоростей системы $K = N(n-1)+3$.

В определении псевдоскоростей $\nu_s$, равных угловым скоростям собственного вращения свободных роликов, используется индекс $s$, как и для углов $\phi_s$ поворота роликов вокруг их осей. Индекс $s$ для угловой скорости $\nu_s$ собственного вращения ролика с номером $j$ на колесе $i$ принимает значения $4,\ldots,3+N(n-1)$ и по определению задается формулой
\begin{equation}\label{eq:num}
    s(i, j) = 3 + (n-1)(i-1) + j - 1, \enspace i = 1,\ldots,N, \enspace j = 2,\ldots,n
\end{equation}

Поскольку $\vec{v}_S = \dot{x}\vec{e}_X + \dot{y}\vec{e}_Y$, из\upr{eq:vsnu3} имеем
\begin{equation}\label{eq:constraints_xy}
    \dot{x} = R \nu_1\cos\theta-R\nu_2\sin\theta, \hspace{15pt} \dot{y} = R\nu_1\sin\theta+R\nu_2\cos\theta
\end{equation}

В главах 1 и 2 будем считать, что проскальзывания между опорной плоскостью и роликами в точках контакта не происходит, т.е. скорости точек роликов $C_i$, находящихся в контакте (см. фиг.~\ref{fig:wheel}) равны нулю:
\begin{equation}\label{eq:constraints_vec}
    \vec{v}_{C_i} = 0,\quad i = 1,\dots, N    
\end{equation}

Проектируя скорости точек контакта на оси системы $S\xi\eta$ и выражая через введенные псевдоскорости, получим:
\begin{eqnarray}
    \dot{\phi_k} &=& \frac{R}{\rho_k }\left(\nu_1\cos\alpha_k + \nu_2\sin\alpha_k\right); \quad \rho_k  = l\cos\chi_k - r \label{constraint_roller_contact} \\
    \dot{\chi}_i &=& \frac{R}{l}\left(\nu_1\sin\alpha_i - \nu_2\cos\alpha_i - \frac{\nu_3}{\Lambda}\right)\label{constraint_wheel_contact}
\end{eqnarray}
Заметим, что знаменатель $\rho_k$ в формуле (\ref{constraint_roller_contact}) -- расстояние от оси ролика до точки контакта, обращающееся в нуль на стыке роликов, то есть в случае, когда точка контакта $C_i$ оказывается на оси ролика (см. левую часть фиг.~\ref{fig:wheel}). Данное обстоятельство может приводить к разрывам второго рода функций в правых частях уравнений движения. Эта проблема будет рассмотрена отдельно ниже.

Таким образом, введенные псевдоскорости независимы, и количество степеней свободы системы равно $K$. Выражение обобщенных скоростей через псевдоскорости, учитывающее связи, наложенные на систему~\eqref{eq:vsnu3}--\eqref{eq:constraints_xy} и~\eqref{constraint_roller_contact},~\eqref{constraint_wheel_contact}, можно записать в матричном виде:
\begin{equation}\label{constraints_V}
    \dot{\vec{q}} = \cstr\vnu,\quad \cstr = \cstr(\theta,\chi_i),
\end{equation}

где матрица $\cstr$ имеет вид:
$$
\cstr = \begin{bmatrix}
        \widetilde{V}  & O_1  \\[6pt]
        O_2       & E         \\[6pt]
    \end{bmatrix};
\quad
\widetilde{V} = \begin{bmatrix}
        R\cos\theta                    & -R\sin\theta                    & 0                      \\[6pt]
        R\sin\theta                    &  R\cos\theta                    & 0                      \\[6pt]
        0                              & 0                               & \ddfrac{1}{\Lambda}    \\[6pt]
        \ddfrac{R}{l}\sin\alpha_i      & -\ddfrac{R}{l}\cos\alpha_i      & -\ddfrac{R}{\Lambda l} \\[6pt]
        \ddfrac{R}{\rho_k}\cos\alpha_k &  \ddfrac{R}{\rho_k}\sin\alpha_k & 0                      \\[6pt]
    \end{bmatrix}
$$
Здесь $O_1$ и $O_2$ -- нулевые $(3+2n \times N(n-1))$- и $(N(n-1) \times 3)$-матрицы, $E$ -- единичная матрица размера $N(n-1)$.

\vspace{15pt}

В литературе широко известна модель экипажа, не учитывающая инерцию роликов \cite{ZobovaTatarinovAspecty2006,zobova2008svobodnye8020851,ZobovaTatarinovPMM,Zobova2011,MartynenkoFormalskii2007,formalskii,Martynenko2010_rus,Borisov2011,KilinBobykin2014}. Такую модель экипажа будем называть безынерционной. Описание безынерционной модели и ее уравнения движения приведены в Приложении к главе 1. Отметим здесь, что уравнение (\ref{constraint_wheel_contact}) совпадает с уравнением связи в безынерционной модели в форме, приведенной в~\cite{Zobova2011}. В настоящей работе строятся модели, учитывающие инерцию роликов, и сравниваются с безынерционной моделью.
