\section{Метод построения механической модели системы твердых тел}

Для описания динамики системы воспользуемся следующим формализмом. Положение каждого твердого тела будем задавать радусом-вектором центра масс тела $\vec{r}$ и кватернионом $\vec{q}$, задающим ориентацию тела; распределение скоростей описывается скоростью центра масс $\vec{v}$ и угловой скоростью тела $\vec{\omega}$.
Динамика этого твердого тела описывается уравнениями Ньютона-Эйлера:
$$ m\dot{\vec{v}} = \vec{F} + \vec{R}, \quad \dot{\vec{r}} = \vec{v} $$
$$ J\dot{\vec{\omega}} + [ \vec{\omega}, J\vec{\omega} ] = \vec{M} + \vec{L}, \quad \dot{\vec{q}} = (0, \enspace \vec{\omega}), $$
где это -- это это.
        
Для получения замкнутой системы уравнений требуется ввести также уравнения связей 
$$ f(\vec{r}, \vec{v}, \vec{q}, \vec{\omega}) = 0 $$

!!!!!!!!!!либо фразу про то, что в формуле есть еще и радиус-векторы других тел, либо индексы ранее.

и модель реакций связей, в частности, контактного взаимодействия:
$$ g(\vec{R}, \vec{r}, \vec{v}, \vec{L}, \vec{q}, \vec{\omega}) = 0. $$
!!!!!!!!!!либо фразу про то, что в формуле есть еще и радиус-векторы других тел, либо индексы ранее.

Отметим, что при необходимости изменения закона контактного взаимодействия достаточно изменить только уравнения 2 и 3.

В построенной системе уравнений используются избыточные координатах, а сами уравнения, вообще говоря, не разрешены относительно первых производных фазовых переменных. Для эффективного численного интегрирования такой системы требуется представить ее в виде

$$ \textit{УТВЕРДИТЬ С И.И.}, $$

что может быть сделано в полуавтоматическом режиме с помощью инструментов инфраструктуры языка \texttt{Modelica}. Эти инструменты выполняют редукцию индекса построенной системы дифференциально-алгебраических уравнений ССЫЛКА НА ПАНТЕЛИДЕСА И ФРИЦСОНА и строят разностную схему  численного решения.

Далее, в параграфе ... описана геометрия и  отдельно -- и чрезмерно подробно -- опишем также все детали этой кинематики для омни- и меканум колес.

Проведено сравнение движений безынерционной модели экипажа и модели экипажа на плоскости с сухим трением.
В таблице приведены величины отличий угла курса $\theta$ экипажа и координат центра масс $x, y$ к моменту безразмерного времени $t = 10$.
Отличия уменьшаются с уменьшением порядка величины отношения массы одного ролика $m_{\text{рол}}$ к суммарной массе колеса $m_{\text{к}}$.

\begin{table}[]
    \begin{tabular}{l|l|l}
     & Движение $1$ & Движение $2$ \\ \hline
    $\frac{m_{\text{рол}}}{m_{\text{к}}}$ &
    $\Delta \theta$ &
    $\max(|\Delta x|, |\Delta y|)$ \\ \hline
    $10^{-1}$ & $\approx 1$       & $\approx 1$       \\
    $10^{-2}$ & $\approx 10^{-1}$ & $\approx 0.5$     \\
    $10^{-3}$ & $\approx 10^{-2}$ & $\approx 10^{-1}$ \\
    $10^{-4}$ & $\approx 10^{-3}$ & $\approx 10^{-2}$ \\
    $10^{-5}$ & $\approx 10^{-3}$ &                   \\
    $10^{-6}$ & $\approx 10^{-4}$ & 
    \end{tabular}
\end{table}
