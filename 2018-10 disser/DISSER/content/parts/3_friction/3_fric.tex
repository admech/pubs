\section{Моделирование трения в контакте}

Конструкция омни-колеса такова, что в каждый момент времени имеется имеется только один ролик, контактирующий с опорной поверхностью. Для остальных роликов при этом во время расчета алгоритм отслеживания контакта продолжает работать, генерируя в нулевые реакции.

В случае фактического выполнения контакта помимо нормальной реакции вычисляется также её касательная составляющая, сила трения. Для касательного контактного усилия имеется множество различных моделей. Мы остановились на реализации двух случаев при одноточечном твердотельном контакте:
\begin{enumerate}
    \item {
        регуляризованное сухое трение
        $$
            \vec{F}_{\text{тр}} = -\mu N \vec{v}_{C}
                \left\{
                    \begin{array}{ll}
                        \ddfrac{1}{\delta}, \enspace |\vec{v}_{C}| < \delta \ll 1 \vspace{7pt}\\
                        \ddfrac{1}{|\vec{v}_{C}|}\enspace \text{иначе}
                    \end{array}
                \right.
        $$
    }
    \item {
        вязкое трение
        $$
            \vec{F}_{\text{тр}} = -\gamma\vec{v}_{C}
        $$
    }
\end{enumerate}

Как известно, идеальный <<сухой>> случай реализовать не удается из-за разрыва в правой части уравнений движения, поэтому вместо разрывной функции от касательной скорости относительного скольжения контактирующих поверхностей, в первом случае используется её регуляризованный в нуле вариант. Вместо функции $sign$ применяется функция насыщения, представляющая собой линейный участок с достаточно большим угловым коэффициентом в окрестности нуля. Для таких функций известен результат~\cite{Novozhilov1991} о близости аппроксимирующего движения и движения, соответствующего точному случаю разрывной функции $sign$. В целом, реализация модели неудерживающей связи основана на результатах~\cite{Kosenko2006unilat}.
