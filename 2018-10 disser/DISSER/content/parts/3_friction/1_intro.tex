\chapter{Динамика экипажа на омни-колесах с трением}

В предыдущих главах рассмотрена динамика омни-колесного экипажа на абсолютно шероховатой плоскости. Однако, как было показано ранее, в этом случае на интервалах движения без смены ролика в контакте полная механическая энергия сохраняется, что никогда не происходит в реальных системах. Отдельного рассмотрения заслуживает вопрос о том, прекращается ли в реальных системах скольжение вновь вошедшего в контакт ролика. Таким образом, 
интерес представляет изучение динамики экипажа по плоскости с трением.

В данной главе вместо идеальных неголомных связей отсутствия проскальзывания используются голономные неидеальные неудерживающие связи. Для касательных составляющих реакций опорной плоскости используются две модели: сухое трение Амонтона -- Кулона, регуляризованного в окрестности нуля по скорости проскальзывания линейной функцией с насыщением; вязкое трением.

Построение модели выполнено таким образом, что изменение в ней модели контактного взаимодействия требует изменения всего одной алгебраической формулы. 
Также подробно рассмотрен вопрос отслеживания контакта роликов и горизонтальной плоскости и алгоритмической реализации процесса  переключения контакта от ролика к ролику при качении  роликонесущего колеса.

Динамические свойства построенной модели экипажа проиллюстрированы при помощи численных экспериментов.
Проведена верификация построенной модели с сухим трением в сравнении с безынерционной моделью при стремлении суммарной массы роликов к нулю. Модель с вязким трением представлена в сравнивнении с неголономной моделью, построенной в главах 1 и 2.