\chapter{Динамика экипажа на омни-колесах с трением}

Наряду с постановкой задачи движения омни-колесного экипажа по абсолютно шероховатой плоскости, интерес представляет его динамика на плоскости с трением.

В данной главе строится модель экипажа с сухим трением Амонтона -- Кулона, регуляризованным в окрестности нуля по скоростям участком линейной функции насыщения с достаточно большим угловым коэффициентом, то есть достаточно сильным вязким трением.

Вначале моделируется динамика ролика, совершающего свободное движение в поле сил тяжести.
При этом предполагается, что на ролик может быть наложена неудерживающая связь --- твердотельный контакт с горизонтальной плоскостью. 
Оказалось, что в упомянутых условиях возможно применение упрощенного и эффективного алгоритма отслеживания контакта.
На следующем этапе реализуется модель омни-колеса, а затем -- экипажа в целом.
Геометрия экипажа та же, что и в предыдущих главах.

Особое внимание уделяется вопросу конструирования неудерживающей связи в контакте ролика и горизонтальной плоскости,
отслеживанию точки контакта ролика и опорной плоскости,
а также алгоритмической реализации процесса 
переключения контакта от ролика к ролику при качении омни-колеса.
% Для удобства описания модели отслеживания контакта, в этой главе вводятся специальные обозначения, не связанные с принятыми в предыдущих главах.

Динамические свойства результирующей модели экипажа иллюстрируются при помощи численных экспериментов.
Проводится верификация построенной модели в сравнении с безынерционной моделью при стремлении суммарной массы роликов к нулю.
