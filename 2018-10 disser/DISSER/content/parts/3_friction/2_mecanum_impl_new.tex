Для решения задачи отслеживания контакта снова требуется указать два выражения: формулу для нахождения координат точки контакта и условие наличия контакта как такового. В случае \textit{mecanum}, в отличие от случая омни-колеса, точка контакта $C$, вообще говоря, не лежит в плоскости, содержащей диск колеса. Способ нахождения радиуса-вектора $\vec{r}_C$ точки контакта, предлагаемый в настоящем разделе, основан на вычислении расстояния $\lambda$ от этой плоскости до $C$. Условие наличия контакта аналогично полученному в предыдущем разделе, но для его проверки требуется учесть отличия в геометрии ролика: $\psi \ne 0$.

\textbf{Неявный алгоритм отслеживания контакта}

Как и в случае $\psi = 0$, определим направление ``на центр колеса'' от центра ролика, т.е. вектор
$$
    \vecrho = \ddfrac
        { {\bf r}_{P}-{\bf r}_{K} }
        {\left| {\bf r}_{P}-{\bf r}_{K} \right|}.
$$
Таким образом, ролик находится в контакте, если только если
$$
     \vec{s} \cdot \vec{e}_z < \cos\ddfrac{\pi}{n} \enspace \text{и} \enspace z_C < l.
$$
Также определим вектор $\vec{k}$, направленный вдоль оси колеса (т.е. перпендикулярно плоскости колеса). Тогда радиус-вектор $\vec{r}_C$ точки контакта $C$ можно выразить следующим образом через введенные векторы и радиус-вектор $\vec{r}_K$ центра ролика $K$:
\begin{equation}\label{eq:cont_impl}
    \vec{r}_C = \vec{r}_K + r\vec{s} - l\vec{e}_z + \lambda\vec{k}.
\end{equation}
В этом выражении компоненты векторов $\vec{s}$ и $\vec{k}$, а также расстояние от плоскости колеса до точки контакта $\lambda$ не известны, их требуется вычислять. Для этого введем три вспомогательные системы отсчета и, пользуясь ими, получим необходимые выражения.

Введем систему отсчета $P{\bf i}{\bf j}{\bf k}$, жестко связанную с диском колеса (см. рис.~\ref{fig:mecanum}), с началом в его центре $P$. Пусть вектор ${\bf k}$ направлен вдоль оси колеса, ${\bf i}$ и ${\bf j}$ лежат в его плоскости. Дополнительно введем также системы отсчета $P{\bf i}_1{\bf j}_1{\bf k}_1$ и $K{\bf i}_2{\bf j}_2{\bf k}_2$, где $K$ -- центр ролика.

Вектор ${\bf i}_1$ направим вдоль линии пересечения плоскости колеса и горизонтальной плоскости. Пусть вектор ${\bf k}_1 = {\bf k}$ совпадает c вектором $\vec{k}$ базиса, связанного с колесом, ортогонален плоскости колеса и всегда горизонтален. 
$\vec{j}_1$ положим совпадающим с вектором восходящей вертикали $\vec{\gamma}$. Тогда $\vec{i}_1$ определим как ${\bf i}_1 = {\bf j}_1 \times {\bf k}_1$.

Вектор ${\bf i}_2$ направим вдоль оси собственного вращения ролика, см. рис.~\ref{fig:mecanum}. Этот вектор по определению не может принять вертикальное положение при $\psi = 0$, если ролик находится в контакте с опорной плоскостью, а, в случае \textit{mecanum} ролик повернут на постоянный угол $\psi > 0$ относительно оси $PK$, и потому соотношение ${\bf i}_2 \ne \vec{\gamma}$ верно во все время движения. Это обстоятельство позволяет определить вектор ${\bf c} = {\bf i}_2 \times \vec{\gamma}$, всегда отличный от нуля. Положим ${\bf k}_2 = \ddfrac{\vec{c}}{|\vec{c}|}$. Вектор ${\bf j}_2$ положим ортогональным ${\bf i}_2$ и лежащим в вертикальной плоскости как ${\bf j}_2 = {\bf k}_2 \times {\bf i}_2$.

Для определения компонент вектора $\vecrho$ теперь воспользуемся условиями ортогональности векторов, следующими из определений введенных систем отсчета:
$$
    \vecrho \cdot {\bf i}_2 = 0, \quad \vecrho \cdot {\bf k}_1 = 0,
$$
а точнее, их дифференциальными вариантами:
$$
    \dfrac{d}{dt} \vecrho \cdot {\bf i}_2 + \vecrho \cdot \dfrac{d}{dt} {\bf i}_2 = 0, \quad
    \dfrac{d}{dt} \vecrho \cdot {\bf k}_1 + \vecrho \cdot \dfrac{d}{dt} {\bf k}_1 = 0.
$$

Чтобы получить число $\lambda$, умножим уравнение\upr{eq:cont_impl} скалярно на ${\bf k}_2$. Поскольку ${\bf r}_C - {\bf r}_K$ лежит в вертикальном сечении осесимметричной поверхности ролика, и вектор ${\bf k}_2$ по построению ортогонален этому сечению, расстояние $\lambda$ от плоскости колеса до точки контакта $C$ определяется однозначно как
$$
    \lambda = \ddfrac
        { \left( r\vec{s} - l\vec{e}_z \right) \cdot \vec{k}_2 }
        { \vec{k}_1 \cdot \vec{k}_2 },
$$
а вместе с ним, и весь радиус-вектор $\vec{r}_C$. Аналогично предыдущему разделу, налагается связь вида~(\ref{eq:cont_vn},\ref{eq:cont_Dvn}).
