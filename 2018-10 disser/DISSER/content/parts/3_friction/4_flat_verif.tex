\section{Верификация}

\subsection{Примеры движений}
Задавая параметры экипажа, такие как массы его частей, их моменты инерции, геометрические размеры, положения, а также начальные данные - скорость центра масс и угловую скорость платформы, - и выполняя согласованные расчеты для двух реализаций - физической и идеальной - можно получить достаточно близкие движения при достаточно малой доле массы роликов.

\begin{figure}[!ht]
    \centering
    \includegraphics[width=0.95\textwidth]{content/parts/3_friction/diploma/img/art/my_exp_setup.png}
    \caption{Параметры экспериментов}
    \label{fig:my_exp_setup}
\end{figure}

При выполнении численных экспериментов массы платформы и колес, количество колес, количество роликов, геометрия системы были фиксированы (см. рис.~\ref{fig:my_exp_setup}). Изменялись начальные данные и доля массы роликов.

Испытания проводились, в частности, и для случая, когда относительная суммарная масса роликов приближается к нулю. В этом случае оказалось, что движение экипажа и омни-колес неограниченно приближаются к соответствующим функциям решения задачи Коши, получаемым в силу дифференциальных уравнений движения, используемых в работе~\cite{Borisov2011}, в которых динамика роликов не учитывается.

Рассмотрены два типа начальных условий $\vec{v}(0) = (v_0, 0, 0)^T, \omega(0) = \omega_0$ (см. рис.~\ref{fig:my_exp_setup}):
\begin{enumerate}
\item экипаж имеет начальную линейную скорость в направлении одного из колес и не закручен (ожидаемый результат - центр масс экипажа движется вдоль оси $Ox$, экипаж не вращается),
\item экипаж закручен вокруг вертикальной оси, проходящей через его центр масс, скорость центра масс равна нулю (ожидаемый результат - экипаж вращается вокруг своей вертикальной оси симметрии, и центр масс покоится).
\end{enumerate}

Значения отношения $\eta$ массы ролика к общей массе колеса принимали в обоих случаях значения от $10^{-6}n^{-1}$ до $10^{-1}n^{-1}$ с шагом $1$ по порядку малости (здесь $n$ - фиксированное количество роликов).

На рис.~\ref{fig:exp_examples} приведены примеры траектории центра масс $y(x)$ и зависимости $\psi(t)$ угла поворота $\psi$ платформы вокруг вертикальной оси, проходящей через её центр, для случаев 1) и 2). Кривые $y(x)$, изображающие траектории центра масс, соответствуют, в сущности, точке - началу координат - в случае $v_0 = 0, \omega_0 = 1$, и отрезку прямой, совпадающей с осью $x$, в случае $v_0 = 1, \omega_0 = 0$, ибо масштаб отображения таков, чтобы были видны отклонения от точных значений, возникающие в силу вычислительной погрешности, но сами эти отклонения имеют порядок малости, позволяющий считать их нулевыми. Аналогичное утверждение верно и для зависимости угла поворота платформы $\psi$ от времени в случае поступательного движения - полученная зависимость близка к постоянной.

Ниже представлены результаты нескольких численных экспериментов. Во всех случаях величины, изображенные на рис.~\ref{fig:exp_examples}, демонстрируют поведение, не различимое в масштабе рис.~\ref{fig:exp_examples}, и поэтому приведены лишь расхождения между построенной нами моделью и верификационной идеализацией, которые и представляют интерес. Также представлена абсолютная величина скорости скольжения в точке контакта в физической модели.

Графики зависимости скорости скольжения от времени показывают, что скольжение имеет место в окрестности момента смены роликов. Это объясняется тем, что для идеального качения в эти моменты ролику необходима бесконечная угловая скорость собственного вращения, ибо его размер вблизи вершины стремится к нулю. Видно, что с ростом доли массы роликов в общей массе колеса скольжение в контакте становится существеннее, изменяясь от пренебрежимо малого при $\eta = 10^{-6}$ до весьма существенного уже при $\eta = 10^{-3}$. Тем не менее, расхождения траектории и угла поворота платформы малы, а скольжение наблюдается лишь в точках колеса, которые в промышленных конструкциях не присутствуют (см. Обзор), что и позволяет считать верификацию проведенной.
\newpage

% EXAMPLES
\begin{figure}[h]
\centering
\begin{subfigure}{.47\textwidth}
    \centering
    \includegraphics[width=\textwidth]{content/parts/3_friction/diploma/img/res/example_v_0_0_omega_1_frac_1e-1_n_4_time_10s.png}
    \caption{$\eta = 0,1, v_0 = 0, \omega_0 = 1$}
    \label{fig:exp_example_omega}
\end{subfigure}%
\hspace{5pt}
\begin{subfigure}{.47\textwidth}
    \centering
    \includegraphics[width=\textwidth]{content/parts/3_friction/diploma/img/res/example_v_1_0_omega_0_frac_1e-1_n_4_time_10s.png}
    \caption{$\eta = 0,1, v_0 = 1, \omega_0 = 0$}
    \label{fig:exp_example_v}
\end{subfigure}
\caption{Примеры траекторий, характера изменения угла и смены номеров роликов в контакте для двух типов начальных условий. На нижнем графике - номер ролика в контакте, см. рис.~\ref{OmniWheel}}
\label{fig:exp_examples}
\end{figure}
\newpage

\begin{figure}[h]
\begin{center}\begin{equation*}\begin{array}{cc}
\includegraphics[width=7cm, viewport=0 0 395 745,clip]{content/parts/3_friction/diploma/img/res/comparison_v_0_0_omega_1_frac_1e-1_n_4_time_10s.png} & \includegraphics[width=7cm, viewport=0 0 395 745,clip]{content/parts/3_friction/diploma/img/res/comparison_v_0_0_omega_1_frac_1e-2_n_4_time_10s.png}\\
\eta = 0,1, v_0 = 0, \omega_0 = 1 & \eta = 0,01, v_0 = 0, \omega_0 = 1\\
\end{array}\end{equation*}\end{center}
\caption{Вращение экипажа с трением вокруг вертикальной оси}
\end{figure}
\newpage

\begin{figure}[h]
\begin{center}\begin{equation*}\begin{array}{cc}
\includegraphics[width=7cm, viewport=0 0 395 745,clip]{content/parts/3_friction/diploma/img/res/comparison_v_0_0_omega_1_frac_1e-3_n_4_time_10s.png} & \includegraphics[width=7cm, viewport=0 0 395 745,clip]{content/parts/3_friction/diploma/img/res/comparison_v_0_0_omega_1_frac_1e-4_n_4_time_10s.png}\\
\eta = 0,001, v_0 = 0, \omega_0 = 1 & \eta = 0,0001, v_0 = 0, \omega_0 = 1\\
\end{array}\end{equation*}\end{center}
\caption{Вращение экипажа с трением вокруг вертикальной оси}
\end{figure}
\newpage

\begin{figure}[h]
\begin{center}\begin{equation*}\begin{array}{cc}
\includegraphics[width=7cm, viewport=0 0 395 745,clip]{content/parts/3_friction/diploma/img/res/comparison_v_0_0_omega_1_frac_1e-5_n_4_time_10s.png} & \includegraphics[width=7cm, viewport=0 0 395 745,clip]{content/parts/3_friction/diploma/img/res/comparison_v_0_0_omega_1_frac_1e-6_n_4_time_10s.png}\\
\eta = 10^{-5}, v_0 = 0, \omega_0 = 1 & \eta = 10^{-6}, v_0 = 0, \omega_0 = 1\\
\end{array}\end{equation*}\end{center}
\caption{Вращение экипажа с трением вокруг вертикальной оси}
\end{figure}
\newpage

\begin{figure}[h]
\begin{center}\begin{equation*}\begin{array}{cc}
\includegraphics[width=7cm, viewport=0 0 395 745,clip]{content/parts/3_friction/diploma/img/res/comparison_v_1_0_omega_0_frac_1e-1_n_4_time_10s.png} & \includegraphics[width=7cm, viewport=0 0 395 745,clip]{content/parts/3_friction/diploma/img/res/comparison_v_1_0_omega_0_frac_1e-2_n_4_time_10s.png}\\
\eta = 0,1, v_0 = 1, \omega_0 = 0 & \eta = 0,01, v_0 = 1, \omega_0 = 0\\
\end{array}\end{equation*}\end{center}
\caption{Вращение экипажа с трением по прямой}
\end{figure}
\newpage

\begin{figure}[h]
\begin{center}\begin{equation*}\begin{array}{cc}
\includegraphics[width=7cm, viewport=0 0 395 745,clip]{content/parts/3_friction/diploma/img/res/comparison_v_1_0_omega_0_frac_1e-3_n_4_time_10s.png} & \includegraphics[width=7cm, viewport=0 0 395 745,clip]{content/parts/3_friction/diploma/img/res/comparison_v_1_0_omega_0_frac_1e-4_n_4_time_10s.png}\\
\eta = 0,001, v_0 = 1, \omega_0 = 0 & \eta = 0,0001, v_0 = 1, \omega_0 = 0\\
\end{array}\end{equation*}\end{center}
\caption{Вращение экипажа с трением по прямой}
\end{figure}
\newpage
