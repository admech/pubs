\chapter{Динамика экипажа на омни-колесах с трением}

+ интро
    + было абс шер но там эн сохр тчо не так
    + тут тр -- кул и вяз
    + легко менять
    + сделали расчеты и сравнили

+ как строили модель
    + н-э (1) св (2) мкс (3)
    + ред инд (ссыль)
    + к-во ст св и ур-й в расчетах
    + отсл конт -- затравка

отсл конт
    для (2) вообще надо пов-ти и град
    омни
        пов-ть
        упс -- острие. плохо для град, но ок для сил 
        ну таки возьмем да сложим вектора
        усл (4) на углы и высоту
        дифф вид для счета
    меканум
        обозн и что 4 поряд
        неявн -- вектора
        явн -- со ссылкой на СЯ
    трение
        вкл-выкл
        выч норм (с пом (4)) и кас реакц
        формулы для трения + про регуляриз по Нов
        
расчеты
    сух
        что вериф с безын при б к 0
        вар-ты н у
        как конкретно б к 0
        что траект -- точка и прямая и что не рис, зато табл
        что проскальз и тем мень чем мень
        что при вращ оно возн из-за интла а при пост еще и ибо просто
        картинка переключ
    вязк
        что кач ср с уд
        картинка спираль
    