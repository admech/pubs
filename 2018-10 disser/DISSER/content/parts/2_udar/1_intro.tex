\chapter{Смена ролика в контакте с опорной плоскостью}

\blfootnote{Данная глава основана на статье \cite{GerasimovZobovaTrudyMAI2018}.}

В настоящей главе рассмотрена задача о смене ролика в контакте. При повороте колеса вокруг своей оси на достаточно большой угол ролик, находящийся в контакте с опорной плоскостью, перестает контактировать, и в контакт с плоскостью приходит новый ролик, скорость которого, вообще говоря, не согласована со связями. При этом возникает переходный процесс, во время которого возможно скольжение ролика относительно опорной плоскости. К окончанию переходного процесса скольжение снова прекращается. В данной главе предполагается, что описанный переходный процесс прекращается за бесконечно малый промежуток времени. Составлены линейные алгебраические уравнения, определяющие обобщенные скорости после смены ролика в контакте в соответствии с теорией удара. Таким образом, для получения движений экипажа требуется найти решения задачи Коши уравнений, полученных в первой главе, на интервалах, где в контакте находится один и тот же ролик, и решения линейных алгебраических уравнений при сменах роликов для получения начальных условий для следующих гладких участков. Получены и проанализированы численные решения для симметричной конфигурации экипажа.

\begin{figure}
    \minipage{\textwidth}
        \centering
        \asyinclude{./content/pic/asy/pic_overlap.asy}
        \caption{Омни-колесо в рассматриваемой конфигурации. Концы роликов усечены. Имеется пренебрежимое пересечение тел роликов.}
        \label{fig:overlap}
    \endminipage
\end{figure}

В тех интервалах времени, когда ролик в контакте с опорной плоскостью не меняется, динамика системы описывается уравнениями движения системы \ref{eq:full_system} из главы 1. Смена контакта на $i$-том колесе происходит при значении угла $\chi_i = \ddfrac{2\pi}{n}$. При этом, во-первых, правая часть уравнений движения терпит разрыв второго рода из-за равенства нулю величин $\rho_i = l\cos\chi_i-r$ в знаменателе. Во-вторых, снимается условие отсутствия проскальзывания для ролика, выходящего из контакта, и условие отсутствия проскальзывания мгновенно налагается на вновь входящий в контакт ролик.

На практике первое обстоятельство никогда не реализуется, поскольку для закрепления роликов на колесах в реальных системах их концы усекаются. В некоторых вариантах омни-колес ролики располагают рядами в двух и более плоскостях, чтобы в каждый момент гладкий участок поверхности хотя бы одного ролика был в контакте с плоскостью. В данной главе рассматриваются усеченные ролики (см. рис.~\ref{fig:overlap}), но их оси расположены в плоскости колеса. Пересечением тел роликов в пространстве, возникающем в такой конфигурации, пренебрегается. Ось ролика находится на расстоянии $r = l\cos\ddfrac{\pi}{n-1}$ от центра колеса. Ролик представляет собой тело вращения относительно этой оси дуги окружности радиуса $l$ с углом раствора $\ddfrac{2\pi}{n}$.

В данной главе проведено детальное рассмотрение момента времени смены ролика в контакте с учетом ударного характера взаимодействия с опорной плоскостью. Проведено численное моделирование движения экипажа по инерции. Получены численные решения, состоящие из участков, определяемых уравнениями движения\upr{eq:full_system} из главы 1, сшитых с помощью решения уравнений теории удара.
