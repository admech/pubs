\section{Модель удара при смене ролика в контакте}

\begin{figure}[h]
    \minipage{0.5\textwidth}
        \centering
        \asyinclude{content/pic/asy/pic_react}
        \caption{Реакции}
        \label{fig:react}
    \endminipage
    \minipage{0.5\textwidth}
        \centering
        \asyinclude{content/pic/asy/pic_project}
        \caption{Проекция}
        \label{fig:project}
    \endminipage
\end{figure}

В реальной системе при смене контакта имеет место скольжение роликов относительно плоскости, и при этом полная энергия системы рассеивается. В данной главе будем считать, что трение при проскальзывании достаточно велико, и прекращение проскальзывания вошедшего в контакт ролика происходит достаточно быстро. Это взаимодействие будет рассматриваться как абсолютно неупругий удар, происходящий при мгновенном снятии и наложении связи. Освободившийся ролик на том же колесе начинает свободно вращаться вокруг своей оси.
Будем предполагать следующее:
\begin{itemize}
    \item ударное взаимодействие происходит за бесконечно малый интервал времени $\Delta t << 1$, так что изменения обобщенных координат пренебрежимо малы $|\Delta \q| \sim |\dotq\Delta t| << 1$, а изменения обобщенных скоростей конечны $|\Delta \dotq| < \infty$;
    \item взаимодействие экипажа с опорной плоскостью во время удара сводится к действию на экипаж в точках контакта нормальных и касательных реакций $\mathbf{R}_i = \mathbf{N}_i + \mathbf{F}_i$, где индекс $i$ равен номеру колеса;
    \item сразу после удара 
    % $t^*+\Delta t$
    уравнения связей, запрещающих проскальзывание, выполнены: $\dqposle = \cstr(\q)\nuposle$, т.е. за время $\Delta t$ проскальзывание вошедшего в контакт ролика заканчивается.
\end{itemize}

% Таким образом, решение задачи представляется в виде чередования гладких участков, определяемых уравнениями движения, и пересчетов значений обобщенных скоростей в моменты смен контакта.

Исходя из этих предположений, в следующих разделах получим системы алгебраических уравнений, связывающих при $\Delta t \rightarrow 0$ значения обобщенных скоростей непосредственно перед ударом $\dqdo$ и значения псевдоскоростей сразу после удара $\nuposle$.
% двумя разными способами: в первом случае эти уравнения получаются при определении ударных реакций, действующих в точках контакта, а во втором, при проектировании в кинетической метрике вектора обобщенных скоростей на пространство виртуальных перемещений, задаваемое уравнениями вновь налагаемых связей.

Таким образом, чтобы найти движение экипажа, необходимо найти решения задачи Коши системы обыкновенных дифференциальных уравнений в интервалах между моментами смены роликов в контактах и решения системы алгебраических уравнений в эти моменты для получения начальных условий для следующего безударного интервала.
