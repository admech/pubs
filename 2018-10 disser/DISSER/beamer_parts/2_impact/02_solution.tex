\begin{frame}{Глава 2. Нахождение обобщенных скоростей после удара}{В том числе, после кратного удара}
    \begin{columns}
        \hspace{15pt}
        \begin{column}{0.35\textwidth}
            $$ \eqnuposleproj $$
            
            Действительно, т.к.
            $$ \mke\deltadq = \Q, \enspace \Q^T\delta\qposle = 0,$$
            $$ \dqposle = \dqdo -\ \deltadq = \mathbf{V}\nuposle \in \subspace,$$
            то
            \begin{eqnarray*}
                0 & = & \dotp{\mke\deltadq}{\cstr} \\
                  & = & \dotp{\mke\left(\cstr\nuposle - \dqdo\right)}{\cstr} \\
                  & = & \dotp{\mke\cstr\nuposle - \mke\dqdo}{\cstr} \\
                  & = & \cstr^T\mke\cstr\nuposle - \cstr^T\mke\dqdo, \enspace \text{ч.т.д.}
            \end{eqnarray*}
        \end{column}
        \hspace{55pt}
        \begin{column}{0.65\textwidth}
            \begin{figure}
                % \centering
                \hspace{-65pt}
                \asyinclude{content/pic/asy/pic_project}
                \caption{
                    $\dqposle$ -- проекция $\dqdo$ на $\subspace$,\newline
                    ортогональная в метрике $\mke$
                }
            \end{figure}
        \end{column}
    \end{columns}
\end{frame}
