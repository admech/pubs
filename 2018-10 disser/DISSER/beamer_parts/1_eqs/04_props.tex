
\begin{frame}{Структура уравнений}{Свойства}    
\begin{enumerate}
    \item Интеграл энергии \quad $\frac{1}{2}\vec{\nu}^\mathrm{T}\M^*(\chi_i)\vec{\nu} = h = \mathrm{const}$\\
    (связи автономны, идеальны, силы консервативны)
    \item $\nu_1 = \nu_2 = \nu_3 = 0 \quad \implies \quad \nu_s = \mathrm{const}$
    \item $B = 0 \ \implies$ уравнения как в безынерционной модели.
    \item Интеграл $m_{33}^*\nu_3 = \mathrm{const}$ разрушается при $B \neq 0$. \ $\dot{\nu_3} \textprop B$.
    \item Первые интегралы:
    $$\nu_s + \ddfrac{1}{\Lambda}\sin\chi_{ij}\nu_3 = const.$$
    Вращение $\nu_1(0) = 0, \nu_2(0) = 0, \nu3(0) \neq 0$ неравномерно.
    \item Замена псевдоскоростей $\vec{\nu} \rightarrow \lambda\vec{\nu}, \lambda \neq 0$ эквивалентна замене времени $t \rightarrow \lambda t$.
\end{enumerate}
\end{frame}
