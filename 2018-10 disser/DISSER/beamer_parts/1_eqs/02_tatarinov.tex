\begin{frame}{Уравнения Я.В. Татаринова}
  \begin{itemize}
  \item {
    \begin{equation}\label{Tatarinov}
    \frac{d}{dt}\frac{\partial L^{*}}{\partial \nu_\alpha}  + \{P_\alpha, L^{*}\} = \{P_\alpha, \nu_\mu P_\mu\},
    \end{equation}
    $$ \nu_\mu P_\mu = \dot{q_i} p_i, \hspace{10pt} p_i = \frac{\partial L}{\partial \dot{q}_i} $$
  }
  \item {
    Лагранжиан и ``импульсы'' отличаются аддитивными членами:
    $$ L^{*} = \mathring{L}^{*} + BL^{*}_\Delta(\nu, \chi) $$
    $$ P_\alpha = \mathring{P_\alpha}(\theta, p_x, p_y, p_\chi) + P_\Delta(p_{\phi_i}, \chi) $$
  }

  \end{itemize}
\end{frame}