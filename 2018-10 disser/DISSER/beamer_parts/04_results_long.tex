\begin{frame}{Подход}
    \begin{itemize}
        \item {
            Построены модели экипажа с омни-колесами, движущегося по горизонтальной плоскости по инерции: неголономная с идеальными связями и голономная с неидеальными. Обе модели учитывают инерцию роликов омни-колес.
        }
        \item {
            Первая модель получена в предположении, что ролик омни-колеса не проскальзывает относительно плоскости (связи идеальны). Уравнения движения на гладких участках  (т.е. между сменой ролика в контакте) получены аналитически в псевдоскоростях и представляют собой уравнения 33 порядка для экипажа с 3 колесами и 5 роликами на каждом колесе. С помощью теории удара расчет изменения обобщенных скоростей при смене ролика в контакте сведен к решению системы линейных алгебраических уравнений, имеющей единственное решение (в т.ч. при кратном ударе).
        }
        \item {
            Обоснована (взаимная?) корректность неголономной модели с массивными роликами. Для этого аналитически показано, что при равенстве осевого момента инерции ролика нулю, её уравнения движения совпадают с уравнениями движения безынерционной модели.
        }
        \item {
            Показано, что линейный первый интеграл, существующий в безынерционной модели, разрушается при осевом моменте инерции, отличном от нуля. При этом скорость изменения значения этого интеграла пропорциональна осевому моменту инерции ролика. Найдены линейные интегралы, связывающие угловую скорость платформы экипажа и скорости собственного вращения роликов, не находящихся в контакте.
        }
        \item {
            В ходе численных экспериментов обнаружен эффект быстрого убывания скорости центра масс по сравнению с угловой скоростью платформы экипажа. Данный эффект противоположен эффекту быстрого убывания скорости верчения шара по сравнению со скоростью его центра при движении по шероховатой плоскости.
        }
        \item {
            Модель с неидеальными голономными связями реализована в системе автоматического построения численных динамических моделей для вязкого трения и для регуляризованного сухого трения. Для этого найдены геометрические условия контакта роликов омни- и mecanum-колес и опорной плоскости.
        }
        \item {
            Обоснована (взаимная?) корректность динамической модели с трением (и безынерционной модели). Для этого численно показано, что при стремлении осевого момента инерции ролика к нулю, движения системы стремятся к движениям безынерционной модели.
        }
        \item {
            Численные эксперименты показали, что проскальзывание между роликом в контакте и опорной плоскостью заканчивается за время, существенно меньшее, чем время нахождения ролика в контакте. Это служит обоснованием применимости теории удара в неголономной модели. Также обнаружено(?), что время проскальзывания тем меньше, чем меньше осевой момент ролика.
        }
    \end{itemize}
\end{frame}

