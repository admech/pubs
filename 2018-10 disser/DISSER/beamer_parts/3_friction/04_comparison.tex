\begin{frame}{Глава 3. Сравнение с безынерционной моделью}
    
    Проведено сравнение движений безынерционной модели экипажа и модели экипажа на плоскости с сухим трением.
    В таблице приведены величины отличий угла курса $\theta$ экипажа и координат центра масс $x, y$ к моменту безразмерного времени $t = 10$.
    Отличия уменьшаются с уменьшением порядка величины отношения массы одного ролика $m_{\text{рол}}$ к суммарной массе колеса $m_{\text{к}}$.
    
    \vspace{-12pt}
    \begin{table}[]
        \begin{tabular}{l|l|l}
         & Движение $1$ & Движение $2$ \\ \hline
        $\frac{m_{\text{рол}}}{m_{\text{к}}}$ &
        $\Delta \theta$ &
        $\max(|\Delta x|, |\Delta y|)$ \\ \hline
        $10^{-1}$ & $\approx 1$       & $\approx 1$       \\
        $10^{-2}$ & $\approx 10^{-1}$ & $\approx 0.5$     \\
        $10^{-3}$ & $\approx 10^{-2}$ & $\approx 10^{-1}$ \\
        $10^{-4}$ & $\approx 10^{-3}$ & $\approx 10^{-2}$ \\
        $10^{-5}$ & $\approx 10^{-3}$ &                   \\
        $10^{-6}$ & $\approx 10^{-4}$ & 
        \end{tabular}
    \end{table}
    
\end{frame}