\begin{frame}{Подход}
    \begin{itemize}
        \item {
            Построены модели омни-экипажа: неголономная с идеальными связями и голономная с неидеальными. Обе модели учитывают инерцию роликов.
        }
        \item {
            В модели 1 уравнения движения на гладких участках получены аналитически и представляют собой уравнения 33 порядка для экипажа с 3 колесами и 5 роликами на каждом колесе. Расчет изменения обобщенных скоростей проведен с помощью теории удара.
        }
        \item {
            Аналитически показано, что при равенстве осевого момента инерции ролика нулю, уравнения движения совпадают с уравнениями движения безынерционной модели.
        }
        \item {
            Показано, что линейный первый интеграл, безынерционной модели, разрушается при $B \new 0$. Скорость изменения значения этого интеграла пропорциональна осевому моменту инерции ролика. Найдены линейные интегралы, связывающие угловую скорость платформы экипажа и скорости собственного вращения роликов, не находящихся в контакте.
        }
        \item {
            В ходе численных экспериментов обнаружен эффект быстрого убывания скорости центра масс по сравнению с угловой скоростью платформы экипажа. Данный эффект противоположен эффекту быстрого убывания скорости верчения шара по сравнению со скоростью его центра при движении по шероховатой плоскости.
        }
        \item {
            Модель с неидеальными голономными связями реализована в системе автоматического построения численных динамических моделей для вязкого трения и для регуляризованного сухого трения. Для этого найдены геометрические условия контакта роликов омни- и mecanum-колес и опорной плоскости.
        }
        \item {
            Обоснована (взаимная?) корректность динамической модели с трением (и безынерционной модели). Для этого численно показано, что при стремлении осевого момента инерции ролика к нулю, движения системы стремятся к движениям безынерционной модели.
        }
        \item {
            Численные эксперименты показали, что проскальзывание между роликом в контакте и опорной плоскостью заканчивается за время, существенно меньшее, чем время нахождения ролика в контакте. Это служит обоснованием применимости теории удара в неголономной модели. Также обнаружено(?), что время проскальзывания тем меньше, чем меньше осевой момент ролика.
        }
    \end{itemize}
\end{frame}

