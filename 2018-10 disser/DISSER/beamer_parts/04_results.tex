\begin{frame}[allowframebreaks]{Результаты, выносимые на защиту}
    \small{
    \begin{itemize}
        \item {
            Построены модели экипажа с омни-колесами, движущегося по горизонтальной плоскости по инерции: неголономная с идеальными связями и голономная с неидеальными. Обе модели учитывают инерцию роликов омни-колес.
        }
        \item {
            Первая модель получена в предположении, что ролик омни-колеса не проскальзывает относительно плоскости (связи идеальны). Уравнения движения на гладких участках  (т.е. между сменой ролика в контакте) получены аналитически в псевдоскоростях и представляют собой уравнения 33 порядка для экипажа с 3 колесами и 5 роликами на каждом колесе. С помощью теории удара расчет изменения обобщенных скоростей при смене ролика в контакте сведен к решению системы линейных алгебраических уравнений, имеющей единственное решение в т.ч. при кратном ударе.
        }
        \item {
            % Обоснована корректность неголономной модели с массивными роликами. Для этого а
            Aналитически показано, что при равенстве осевого момента инерции ролика нулю, её уравнения движения совпадают с уравнениями движения безынерционной модели.
        }
        \item {
            Показано, что линейный первый интеграл, существующий в безынерционной модели, разрушается при осевом моменте инерции, отличном от нуля. При этом скорость изменения значения этого интеграла пропорциональна осевому моменту инерции ролика. Найдены линейные интегралы, связывающие угловую скорость платформы экипажа и скорости собственного вращения роликов, не находящихся в контакте.
        }
        \item {
            В ходе численных экспериментов обнаружен эффект быстрого убывания скорости центра масс по сравнению с угловой скоростью платформы экипажа.
            % Данный эффект противоположен эффекту быстрого убывания скорости верчения шара по сравнению со скоростью его центра при движении по шероховатой плоскости.
        }
        \item {
            Модель с неидеальными голономными связями реализована в системе автоматического построения численных динамических моделей для вязкого трения и для регуляризованного сухого трения. Для этого найдены геометрические условия контакта роликов омни- и mecanum-колес и опорной плоскости.
        }
        \item {
            % Обоснована корректность динамической модели с трением. Для этого ч
            Численно показано, что при стремлении осевого момента инерции ролика к нулю, движения системы с трением стремятся к движениям безынерционной модели. Обнаружено качественное сходство траекторий системы с вязким трением с достаточно большим коэффициентом трения с движениями модели, рассмотренной в главах 1 и 2.
        }
        \item {
            Численные эксперименты показали, что проскальзывание между роликом в контакте и опорной плоскостью заканчивается за время, существенно меньшее, чем время нахождения ролика в контакте. Это служит обоснованием применимости теории удара в неголономной модели. Также обнаружено, что время проскальзывания тем меньше, чем меньше осевой момент инерции ролика.
        }
    \end{itemize}
    }
    \vspace{-10pt}
    \center {
        \textcolor{Periwinkle}{Спасибо за внимание!}
    }
\end{frame}

% \begin{frame}[allowframebreaks]{Результаты, выносимые на защиту}
%     \small {
%     \begin{itemize}
%         \item {
%             Две модели экипажа с учетом инерции роликов: неголономная и с трением.
%         }
%         \item {
%             Аналитически получены уравнения движения первой модели (33 порядка для экипажа с 3 колесами и 5 роликами на каждом). С помощью теории удара построена система отн. обобщенных скоростей при смене ролика, имеющая единственное решение в т.ч. при кратном ударе.
%         }
%         \item {
%             Аналитически показано, что при $B = 0$, полученные уравнения движения совпадают с уравнениями движения безынерционной модели.
%         }
%         \item {
%             Показано, что линейный первый интеграл, безынерционной модели, разрушается при $B \neq 0$. При этом скорость его изменения $~B$. Найдены линейные интегралы, связывающие $\nu_3$ и $\nu_s$.
%         }
%         \item {
%             В ходе численных экспериментов обнаружен эффект быстрого убывания скорости центра масс по сравнению с угловой скоростью платформы экипажа.
%         }
%         \item {
%             Модели с сухим и вязким трением реализованы в системе автоматического построения численных динамических моделей. Найдены геометрические условия контакта для омни- и mecanum-колес.
%         }
%         \item {
%             Численно показано, что при стремлении осевого момента инерции ролика к нулю, движения системы с трением стремятся к движениям безынерционной модели, а при стремлении коэффициента трения к бесконечности -- к движениям модели с ударами.
%         }
%         \item {
%             Численно показано, что проскальзывание между роликом и плоскостью заканчивается за время, существенно меньшее, чем время нахождения ролика в контакте, и тем быстрее, чем меньше осевой момент инерции ролика.
%         }
%     \end{itemize}
%     }
% \end{frame}
