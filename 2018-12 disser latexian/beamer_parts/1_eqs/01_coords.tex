\begin{frame}{Глава 1. Постановка задачи}{Координаты, псевдоскорости, связи}
  \begin{itemize}
  \item {
    Обобщенные координаты: \\
    $q = (x, y, \theta, \chi_i, \phi_k, \phi_s),$ где $i,k = 1\dots N$, \\
    индекс $s$ означает ролики вне контакта.
  }
  \item{
    Псевдоскорости:\\
    \hspace{-15pt}
    $\nu = (\nu_1, \nu_2, \nu_3, \nu_s), \enspace \vec{v}_S = R\nu_1\vec{e}_\xi + R\nu_2\vec{e}_\eta, \enspace \nu_3 = \Lambda\dot{\theta}, \enspace \nu_s = \dot{\phi}_s$
  }
  \item {
    Связи:
	$$ \dot{x} = R \nu_1\cos\theta-R\nu_2\sin\theta, \hspace{15pt} \dot{y} = R\nu_1\sin\theta+R\nu_2\cos\theta,$$
	$$\dot{\theta} = \frac{\nu_3}{\Lambda}, \hspace{15pt} \dot{\chi}_i = \frac{R}{l}(\nu_1\sin\alpha_i - \nu_2\cos\alpha_i - \frac{\nu_3}{\Lambda}), $$
	$$ \dot{\phi_k} = \frac{R}{l\cos\chi_k-r}(\nu_1\cos\alpha_k + \nu_2\sin\alpha_k), \hspace{15pt} \dot{\phi}_s = \nu_s $$
  }

  \end{itemize}
\end{frame}