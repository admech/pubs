\begin{myposter}{
    Результаты, выносимые на защиту
}

    \headerbox
    {.}
    {name=first,column=0,row=0,span=3}
    {
        {
        % \LARGE
        \fontsize{0.7cm}{0.5cm}\selectfont
        \bf
            \vspace{10pt}

                \begin{enumerate}
                
                        \item {
                            Построены модели экипажа с омни-колесами, движущегося по горизонтальной плоскости по инерции: неголономная с идеальными связями и голономная с неидеальными. Обе модели учитывают инерцию роликов омни-колес.
                        }
                
                        \item {
                            Первая модель получена в предположении, что ролик омни-колеса не проскальзывает относительно плоскости (связи идеальны). Уравнения движения на гладких участках  (т.е. между сменой ролика в контакте) получены аналитически в псевдоскоростях и представляют собой уравнения 33 порядка для экипажа с 3 колесами и 5 роликами на каждом колесе. С помощью теории удара расчет изменения обобщенных скоростей при смене ролика в контакте сведен к решению системы линейных алгебраических уравнений, имеющей единственное решение в т.ч. при кратном ударе.
                        }
                
                        \item {
                            Aналитически показано, что при равенстве осевого момента инерции ролика нулю, её уравнения движения совпадают с уравнениями движения безынерционной модели.
                        }
                
                        \item {
                            Показано, что линейный первый интеграл, существующий в безынерционной модели, разрушается при осевом моменте инерции, отличном от нуля. При этом скорость изменения значения этого интеграла пропорциональна осевому моменту инерции ролика. Найдены линейные интегралы, связывающие угловую скорость платформы экипажа и скорости собственного вращения роликов, не находящихся в контакте.
                        }
                        
                        \item {
                            В ходе численных экспериментов обнаружен эффект быстрого убывания скорости центра масс по сравнению с угловой скоростью платформы экипажа.
                        }
                        
                        \item {
                            Модель с неидеальными голономными связями реализована в системе автоматического построения численных динамических моделей для вязкого трения и для регуляризованного сухого трения. Для этого найдены геометрические условия контакта роликов омни- и mecanum-колес и опорной плоскости.
                        }
                
                        \item {
                            Для различных моделей экипажа с омни-колесами, рассмотренных в работе -- безынерционной модели и модели экипажа с массивными роликами на плоскости с регуляризованным сухим трением, а также для моделей экипажа с массивными роликами на абсолютно шероховатой плоскости и на плоскости с вязким трением -- показана их взаимная согласованность
                            при стремлении параметров -- момента инерции ролика и коэффициента вязкого трения -- к нулю или к бесконечности, соответственно.
                        }
    
                \end{enumerate}

            \vspace{10pt}
        }
    }
    
\end{myposter}
