\section{Уравнения движения}\label{sect:eqs}

%{\bf 3. Уравнения движения.}
%\stepcounter{section}

Для вывода уравнений движения системы воспользуемся лаконичным методом получения уравнений движения для систем с дифференциальными связями, предложенным Я.В.~Татариновым \cite{Tatarinov,Zobova2011}. Уравнения имеют вид:
\begin{equation}\label{Tatarinov}
    \frac{d}{dt}\frac{\partial L^{*}}{\partial \nu_\beta}  + \{P_\beta, L^{*}\} = \sum\limits_{\mu = 1}^{K}\{P_\beta, \nu_\mu P_\mu\},\quad \beta = 1,\dots, K
\end{equation}
Здесь $L$ -- лагранжиан, $L^*$ -- он же с учетом связей (здесь и далее верхний индекс $*$ означает учет связей, то есть подстановку выражений обобщенных скоростей через псевдоскорости), $P_\beta$ -- линейные комбинации формальных канонических импульсов $p_i$, определяемые из соотношения
\begin{equation}\label{eq:tatimp}
    \sum\limits_{\mu=1}^{K}\nu_\mu P_\mu \equiv \sum\limits_{i=1}^{N(n+1)+3}\dot{q_i} p_i
\end{equation}
в котором $\dot{q}_i$ выражены через псевдоскорости $\nu_\mu$ в соответствии с формулами (\ref{constraints_V}); $\{\cdot, \cdot\}$ -- скобка Пуассона по $p_i$, $q_i$, после ее вычисления выполняется подстановка 
$$
    \hspace{10pt} p_i = \frac{\partial L}{\partial \dot{q}_i}.
$$
Здесь $q_i$ -- компоненты вектора обобщенных координат $\vec{q}$.

Как отмечено ранее, расстояния от осей колес до опорной плоскости постоянны во все время движения. В силу этого факта, однородности всех тел и геометрии расположения колес относительно платформы, потенциальная энергия системы во время движения не меняется, и лагранжиан равен кинетической энергии:
\begin{equation}\label{kin_en}
    \hspace{-8pt}
    2L = 2T = M\vec{v}_S^2 + I_S\dot{\theta}^2 + J\sum_i\dot{\chi}_i^2 + B\sum_{i,j}(\dot{\phi}_{ij}^2 + 2\dot{\theta}\sin(\kappa_j + \chi_i)\dot{\phi}_{ij})=\dot{\vec{q}}^\mathrm{T}\M\dot{\vec{q}}
\end{equation}
%\sout{Здесь полная масса системы -- $M = \mathring{M} + Nnm$, момент инерции всей системы относительно $SZ$ -- $I_S = \mathring{I_S} + Nn(\frac{A+B}{2} + mR^2 + \frac{mr^2}{2})$, момент инерции колеса (с роликами) относительно его оси $J = \mathring{J} + n(A + mr^2)$, где $\mathring{M}, \mathring{I_S}, \mathring{J}$ -- масса и моменты инерции системы и колес без учета роликов; $m$ -- масса ролика; $A$ -- момент инерции ролика относительно любой оси, перпендикулярной его оси собственного вращения и проходящей через его центр масс; $r$ -- радиус диска колеса (расстояние от центра колеса до центра ролика),}
Здесь $M,\ I_S,\ J$ --- массово-инерционные характеристики экипажа (его общая масса, момент инерции относительно оси $SZ$ и момент инерции диска колеса относительно его оси вращения соответственно), $B$ --- момент инерции ролика относительно его оси вращения. Матрица $\M$ кинетической энергии имеет следующий блочный вид:
$$
\M = \begin{bmatrix}
    \widetilde{\M}_{11}   & \widetilde{\M}_{12}   & \widetilde{\M}_{13} \\
    \widetilde{\M}_{12}^T & \widetilde{\M}_{22}   & \widetilde{\M}_{23} \\
    \widetilde{\M}_{13}^T & \widetilde{\M}_{23}^T & \widetilde{\M}_{33} \\
\end{bmatrix}
$$
где
$$
\widetilde{\M}_{11} = \text{diag}(M, M, I_S),
\quad
\widetilde{\M}_{22} = JE_{N \times N},
\quad
\widetilde{\M}_{33} = BE_{Nn \times Nn}
$$
$$
\widetilde{\M}_{12} = O_{3 \times N},
\quad
\widetilde{\M}_{13} = \begin{bmatrix}
        0                      & \cdots & 0                      \\
        0                      & \cdots & 0                      \\
        B\sin\chi_{11}         & \cdots & B\sin\chi_{Nn}         \\
    \end{bmatrix},
\quad
\widetilde{\M}_{23} = O_{N \times Nn}
\vsp
$$
Здесь $O_{(\boldsymbol{\cdot})}$ и $E_{(\boldsymbol{\cdot})}$ --- нулевые и единичные матрицы, в индексах которых указаны их размерности. В третьей строке матрицы $\widetilde{\M}_{13}$ сначала указаны элементы, соответствующие роликам, находящимся в контакте, а затем соответствующие ``свободным'' роликам; элементы упорядочены по возрастанию индексов, так что ролики одного колеса соседствуют, т.е. третья строка матрицы $\widetilde{\M}_{13}$ имеет вид:
$$
    \left( B\sin\chi_{11} \ldots, B\sin\chi_{N1}, B\sin\chi_{12}, \ldots, B\sin\chi_{1n}, B\sin\chi_{22}, \ldots B\sin\chi_{Nn} \right).
$$
Лагранжиан при учете связей определяется соотношением:
$$ 2L^{*}  = \vnu^\mathrm{T} V^\mathrm{T}\M V\vnu = \vnu^\mathrm{T} \M^*(\chi_i)\vnu $$

Структура симметрической матрицы $\M^*$ кинетической энергии, выраженной в псевдоскоростях, следующая:
$$
\M^* = 
    \begin{bmatrix}
        \widetilde{\M}^*_{11} & \widetilde{\M}^*_{12} \\
        \widetilde{\M}^{*T}_{12} & \widetilde{\M}^*_{22} \\[4pt]
    \end{bmatrix}
\vsp
$$
$$
\widetilde{\M}^*_{11} = 
    \left(
        m^*_{ij}
    \right)_{3\times3},
\qquad
\widetilde{\M}^*_{22} = 
    BE_{N(n-1) \times N(n-1)}
\vsp
$$
$$
\widetilde{\M}^*_{12} = 
    \begin{bmatrix}
        0&\ldots& 0 \\
        0&\ldots&0 \\
        B\Lambda^{-1}\sin\chi_{12}&\ldots& B\Lambda^{-1}\sin\chi_{Nn}\\[4pt]
    \end{bmatrix}_{3\times N(n-1)}
\vsp
$$
Здесь $E_{N(n-1) \times N(n-1)}$ --- единичная матрица, $\chi_{kl} = \chi_k+\kappa_l$ --- угол между вертикалью $OZ$ и осью ролика, где индекс $k = 1,\dots,N$ означает номер колеса, $l = 2,\ldots, n$ -- номер cвободного ролика на колесе ($l = 1$ --- ролик, находящийся в контакте).

Элементы $m^*_{ij}$ матрицы $\widetilde{\M}^*_{11}$ зависят только от координат $\chi_i$, которые входят в отношения $\ddfrac{B}{\rho_i^2(\chi_i)}$ и $B\ddfrac{\sin\chi_i}{\rho_i(\chi_i)}$, имеющие разрывы второго рода при смене роликов, т.е. при переходе с одного ролика на другой (см. равенство (\ref{constraint_roller_contact})). Действительно, явный вид элементов матрицы $\widetilde{\M}^*_{11}$ таков:
\begin{equation}\label{mstar}
    \hspace{-15pt}
    \begin{array}{rcl}
        m^*_{11} & = & MR^2 + \sum\limits_i \bigg( J\ddfrac{R^2}{l^2}\sin^2\alpha_i + B\ddfrac{R^2}{\rho_i^2}\cos^2\alpha_i\bigg)\quad(11 \leftrightarrow 22, \sin\alpha_i \leftrightarrow \cos\alpha_i)\vsp
        m^*_{33} & = & \ddfrac{1}{\Lambda}\bigg(I_S + \sum\limits_i J\ddfrac{R^2}{l^2}\bigg),\quad
        m^*_{12}  =  \sum\limits_i \bigg(-J\ddfrac{R^2}{l^2} + B\ddfrac{R^2}{\rho_i^2}\bigg)\sin\alpha_i\cos\alpha_i\vsp
        m^*_{13} & = & \ddfrac{1}{\Lambda}\sum\limits_i B\ddfrac{R}{\rho_i}\sin\chi_i\cos\alpha_i,
        \quad
        m^*_{23} \enspace = \enspace \ddfrac{1}{\Lambda}\sum\limits_i B\ddfrac{R}{\rho_i}\sin\chi_i\sin\alpha_i
    \end{array}
\end{equation}

Первое слагаемое в левой части равенства (\ref{Tatarinov}) получается дифференцированием лагранжиана:
\begin{equation}\label{DLStarDnuDdt}
    \hspace{-10pt}
    \frac{d}{dt}\frac{\partial L^{*}}{\partial \vnu} = \frac{d}{dt}(\M^*(\chi)\vnu) = 
    \M^*(\chi)\dot{\vnu} +
    \frac{d}{dt}(\M^*(\chi))\vnu =
    \M^*(\chi)\dot{\vnu} +
    \sum_{i=1}^{N}\dot{\chi}_i^*\M^*_i\vnu
\end{equation}
где $\M^*_i = \ddfrac{\partial \M^*}{\partial \chi_i}.$

Последнее слагаемое -- $\sum_{i=1}^{N}\dot{\chi}_i^*\M^*_i\vnu$ -- является суммой векторов размерности $K$, где $K = \dim\vnu$ -- число степеней свободы системы. Компоненты этих векторов с номерами $\beta = 4,\dots, K$, то есть компоненты, соответствующие свободным роликам, имеют вид $\dot{\chi}_i^*\nu_3B\Lambda^{-1}\cos\chi_{ij},$ где индексы $i,j$ связаны с индексом $\beta$ по формуле\upr{eq:num}: $\beta = s(i, j)$.

Найдем величины $P_\mu$ во втором слагаемом левой части~(\ref{Tatarinov}). Подставляя в (\ref{eq:tatimp}) выражения обобщенных скоростей через псевдоскорости~(\ref{constraints_V}) и приведя подобные слагаемые, получим:
\begin{equation}\label{P}
    \begin{array}{rcl}
        P_1 & = & R\bigg(p_x\cos\theta + p_y\sin\theta + \sum\limits_{i}\bigg(\ddfrac{p_{\chi_i}}{l}\sin\alpha_i +  \ddfrac{p_{\phi_{i1}}}{\rho_i}\cos\alpha_i\bigg)\bigg)\vsp
        P_2 & = & R\bigg(-p_x\sin\theta + p_y\cos\theta + \sum\limits_{i}\bigg(-\ddfrac{p_{\chi_i}}{l}\cos\alpha_i +  \ddfrac{p_{\phi_{i1}}}{\rho_i}\sin\alpha_i\bigg)\bigg)\vsp
        P_3 & = & \ddfrac{1}{\Lambda}\bigg(p_\theta - \sum\limits_{i}\ddfrac{R}{l}p_{\chi_i}\bigg)\vsp
        P_s & = & p_{\phi_s}
    \end{array}
\end{equation}

Из всех обобщенных координат лагранжиан $L^{*}$ зависит только от $\chi_i$. Можно показать, что его скобки Пуассона с $P_1$, $P_2$, $P_3$  --- квадратичные формы по псевдоскоростям, пропорциональные моменту инерции $B$ ролика. Коэффициенты этих форм зависят от $\chi_i$. Скобки Пуассона лагранжиана с $P_s$ равны нулю.
$$
\{P_1, L^*\} = -\frac{\partial P_1}{\partial p_{\chi_i}}\frac{\partial L^*}{\partial \chi_i} = -\frac{R}{2l}\vnu^\mathrm{T}\M^*_i\vnu\sin\alpha_i,
$$
$$
\{P_2, L^*\} = \frac{R}{2l}\vnu^\mathrm{T}\M^*_i\vnu\cos\alpha_i,\  
\{P_3, L^*\} = \frac{R}{2l\Lambda}\vnu^\mathrm{T}\M^*_i\vnu,\quad \{P_s,L^*\} = 0, s>3
$$

Для завершения вывода уравнений движения рассмотрим правую часть равенства (\ref{Tatarinov}). Она отлична от нуля лишь при $\beta = 1,\ 2,\ 3$. Сперва выразим импульсы $\vec{p}$ через псевдоскорости $\vnu$. Обозначим
$$
    \xi_\pm(\alpha) = \nu_1\cos\alpha \pm \nu_2\sin\alpha, \quad \eta_\pm(\alpha) = \nu_1\sin\alpha \pm \nu_2\cos\alpha.
$$
Тогда для импульсов $\vec{p} = \ddpd{L}{\dot{\vec{q}}}$ получим
\begin{equation}\label{p}
    \begin{array}{c}
        p_x  =  MR\xi_-(\theta), \ p_y = MR\eta_+(\theta),\\
        p_\theta  =  BR\sum\limits_i\ddfrac{\sin\chi_i}{\rho_i}\xi_+(\alpha_i) + \ddfrac{I_S}{\Lambda}\nu_3 + B\sum\limits_s\sin\chi_s\nu_s\vsp\\
        p_{\chi_i}  =  J\ddfrac{R}{l}(\eta_-(\alpha_i) - \ddfrac{1}{\Lambda}\nu_3), \ p_{\phi_{k1}}  =  \ddfrac{BR}{\rho_k}\xi_+(\alpha_k)+\ddfrac{B}{\Lambda}\sin\chi_k,\\
        p_{\phi_s}  =  \ddfrac{B}{\Lambda}\nu_3\sin\chi_s + B\nu_s
    \end{array}
\end{equation}

Для упрощения записи правой части системы\upr{Tatarinov} введем обозначение для операции $\sigma$, которую будем называть дискретной сверткой произвольной функции~$f(\alpha, \chi)$:
$$
\sigma[f(\alpha,\chi)] = \sum\limits_{k=1}^{N} f(\alpha_k,\chi_k) \frac{\sin\chi_k}{\rho_k^3(\chi_k)}
$$
Тогда скобки Пуассона в правой части системы\upr{Tatarinov} имеют вид (звездочкой обозначена подстановка импульсов $p_i$ из~\ref{p})
\hspace{-35pt}
\begin{eqnarray*}
    (\{P_1,P_2\})^* &=& \left(-\sum\limits_{k=1}^{N} R^2\tau_kp_{\phi_k}\right)^* \\
                    &=& -BR^2(R\nu_1 \sigma[\cos\alpha] + R\nu_2 \sigma[\sin\alpha] + \Lambda^{-1}\nu_3\sigma[\rho\sin\chi]) = \\
                    &=& -BR^2\prhs_{12}\vnu\\
    \prhs_{12}      &=& (\sigma[\cos\alpha], R\sigma[\sin\alpha], \Lambda^{-1}\sigma[\rho\sin\chi], 0,\dots,0) \\
    (\{P_1,P_3\})^* &=& {R}{\Lambda}^{-1}\left(-p_x\sin\theta + p_y\cos\theta - \sum\limits_{k=1}^{N} R\cos\alpha_k\tau_kp_{\phi_k}\right)^* \\
                    &=& {MR^2}{\Lambda}^{-1}\nu_2 - \\
                    &-& {BR^2}{\Lambda}^{-1}(R\nu_1 \sigma[\cos^2\alpha] + R\nu_2 \sigma[\sin\alpha\cos\alpha] +\\
                    &+& \Lambda^{-1}\nu_3\sigma[\rho\cos\alpha\sin\chi]) = \\
                    &=& {MR^2}{\Lambda}^{-1}\nu_2-{BR^2} \prhs_{13}\vnu \\
    \prhs_{13}      &=& \Lambda^{-1} (R\sigma[\cos^2\alpha], R\sigma[\sin\alpha\cos\alpha], \Lambda^{-1}\sigma[\rho\cos\alpha\sin\chi], 0,\dots,0) \\
    (\{P_2,P_3\})^* &=& {R}{\Lambda}^{-1}\left(-p_x\cos\theta - p_y\sin\theta - \sum\limits_{k=1}^{N} R\sin\alpha_k\tau_kp_{\phi_k}\right)^* = \\
                    &=& -{MR^2}{\Lambda}^{-1}\nu_1 - {BR^2}{\Lambda}^{-1}(R\nu_1\sigma[\sin\alpha\cos\alpha] + R\nu_2\sigma[\sin^2\alpha] \\
                    &+& \Lambda^{-1}\nu_3\sigma[\rho\sin\alpha\sin\chi] = \\
                    &=& -{MR^2}{\Lambda}^{-1}\nu_1 -{BR^2}\prhs_{23}\vnu\\
    \prhs_{23}      &=& \Lambda^{-1}(R\sigma[\sin\alpha\cos\alpha], R\sigma[\sin^2\alpha], \Lambda^{-1}\sigma[\rho\sin\alpha\sin\chi], 0,\dots,0)
\end{eqnarray*}

Окончательно, объединяя выражения для всех слагаемых\upr{Tatarinov} и разрешая полученные равенства относительно $\M^*\dot{\vnu}$, получаем, что уравнения движения имеют следующую структуру:

\begin{eqnarray*}
    \hspace{-65pt}
    \M^*\dot{\vnu} = 
    MR^2\Lambda^{-1}\begin{bmatrix}
        \nu_2\nu_3\\
        -\nu_1\nu_3\\
        0\\
        0\\
        \vdots
        \\
        0
    \end{bmatrix}
    +\vnu^\mathrm{T}
    \left(
    \frac{R}{2l}
    \begin{bmatrix}
        -\M^*_i \sin\alpha_i\\
        \M^*_i \cos\alpha_i\\
        \M^*_i \Lambda^{-1}\\
        0\\
        \vdots
        \\
        0
    \end{bmatrix}
    -BR^2
    \begin{bmatrix}
        \Prhs_1\\
        \Prhs_2\\
        \Prhs_3\\
        0\\
        \vdots
        \\
        0
    \end{bmatrix}
    \right)
    \vnu
    -B\begin{bmatrix}
        \scalebox{1.5}{$\star$}\\
        \scalebox{1.5}{$\star$}\\
        \scalebox{1.5}{$\star$}\\
        \fontdimen16\textfont2=5pt
        \ddfrac{\nu_3}{\Lambda}\dot{\chi}_1^*\cos\chi_{12}\\
        \vdots
        \\
        \ddfrac{\nu_3}{\Lambda}\dot{\chi}_N^*\cos\chi_{Nn}
    \end{bmatrix}
\end{eqnarray*}
\begin{equation}\label{eq:full_system}
\end{equation}
\fontdimen16\textfont2=1.79999pt
Поясним символ $\star$ в правой части уравнений движения. В равенствах (\ref{DLStarDnuDdt}) последнее слагаемое -- сумма векторов $\dot{\chi}_i^*\M^*_i\vnu, \enspace i = 1 \ldots N$. Символ $\star$ заменяет суммы компонент $\beta = 1,2,3$ этих векторов. Матрицы $\Prhs_\beta$ размера $K\times K$ составлены из строк $\prhs_{\alpha\beta}$, определенных выше и зависящих от геометрии экипажа и углов поворота колес $\chi_i$:
$$ 
\Prhs_1 = \left(
\begin{matrix}
 \vec{0} \\
 \prhs_{12}\\
\prhs_{13}\\
\vec{0}\\
\vdots\\
\vec{0}
\end{matrix}
\right),\quad
\Prhs_2 = \left(
\begin{matrix}
-\prhs_{12}\\
\vec{0}\\
\prhs_{23}\\
\vec{0}\\
\vdots\\
\vec{0}
\end{matrix}
\right),
\Prhs_3 = \left(
\begin{matrix}
-\prhs_{13}\\
-\prhs_{23}\\
\vec{0}\\
\vec{0}\\
\vdots\\
\vec{0}
\end{matrix}
\right)
$$
Поскольку матрицы $\M^*_i$ и $\Prhs_\alpha$ зависят от углов поворота колес $\chi_i$, для замыкания системы к этим уравнениям надо добавить уравнения
(\ref{constraint_wheel_contact}).

Система уравнений движения имеет следующие свойства.
\begin{enumerate}[wide]
    \item Система допускает интеграл энергии $\frac{1}{2}\vnu^\mathrm{T}\M^*(\chi_i)\vnu = h = \mathrm{const}$: так как система стеснена автономными идеальными связями, а силы консервативны, то полная энергия (в рассматриваемом здесь случае она равна кинетической энергии) сохраняется.
    % Доказательство этого факта можно провести двумя способами. Первый основан на общей теореме об изменении полной механической энергии: так как система стеснена автономными идеальными связями, а силы консервативны, то полная энергия (в нашем случае она равна кинетической энергии) сохраняется. Второй способ основан на стандартном приеме: умножение каждого из уравнений на $\nu_\alpha$ и их сложение. Действительно, левая часть вместе со второй группой слагаемых в правой части дадут полный дифференциал выражения $\vnu^\mathrm{T}\M^*(\chi_i)\vnu$, так как $\sum\limits_{\alpha}\{\nu_\alpha P_\alpha, L^*\} = \vnu^\mathrm{T} \frac{\partial\M^*}{\partial \chi_i}\dot{\chi_i}\vnu$. Слагаемые из правой части уравнений (\ref{Tatarinov}) при суммировании дадут тождественный ноль: $\sum\limits_{\alpha,\mu}\{\nu_\alpha P_\alpha, \nu_\mu P_\mu\} = 0$.
    
    \item Если платформа экипажа неподвижна, т.е. $\nu_1 = \nu_2 = \nu_3 = 0$, то свободные ролики сохраняют свою начальную угловую скорость: $\nu_s = \mathrm{const}$, чего и следовало ожидать.
   
    % \item Cистема допускает частное решение $\nu_1= 0, \nu_2 = 0, \nu_3 = const, \nu_s = 0$ --- равномерное вращение вокруг вертикальной оси проходящей через центр масс, при котором ни один ролик не вращается вокруг своей оси. Действительно, левая часть и первое слагаемое правой обращаются в ноль.
    % Вторая группа слагаемых правой части равна нулю, т.к. из \upr{mstar} $\frac{\partial m^*_{33}}{\partial \chi_i} = 0$. Третья группа слагаемых принимает вид
    % $$\nu_3^2\sum_{k}(\cos\alpha_k, \sin\alpha_k, 0, 0, \ldots, 0)^\mathrm{T}\ddfrac{\sin^2\chi_k}{\Lambda^2\rho^2_k}$$
    % Так как из (\ref{constraint_wheel_contact}) $\dot\chi_k = -R(l\Lambda)^{-1}\nu_3$, то в случае одинаковых начальных условий для всех колес, функции $\chi_k$ и $\rho_k$ для всех колес совпадают. Вынося этот множитель за знак суммы и исходя из геометрии экипажа ($\sum_k\cos\alpha_k =  \sum_k\sin\alpha_k = 0$), получим ноль в правой части.
    \item При $B = 0$ (ролики не имеют инерции) все слагаемые в правой части равенства~(\ref{eq:full_system}), кроме первого, обращаются в нуль, как и все члены, соответствующие свободным роликам, в его левой части. В этом случае существенными остаются лишь первые три уравнения системы\upr{Tatarinov} относительно $\nu_1,\ \nu_2,\ \nu_3$. Важно отметить, что оставшиеся нетривиальные уравнения в точности совпадают с уравнениями движения безынерционной модели экипажа \cite{ZobovaTatarinovPMM}.
    \item Существовавший в безынерционной модели линейный первый интеграл \cite{ZobovaTatarinovPMM} разрушается для модели с массивными роликами. При $B = 0$ он имеет вид $m_{33}^*\nu_3 = \mathrm{const}$ (причем $m^*_{33} = \mathrm{const}$) и получается непосредственно из третьего уравнения системы\upr{Tatarinov}. Из вида интеграла следует, что $\nu_3 = \const$. При $B \neq 0$ скорость изменения $\nu_3$ пропорциональна моменту инерции ролика~$B$.
    \item Поскольку скобки Пуассона в уравнениях движения для свободных роликов равны нулю, система допускает первые интегралы:
    \begin{equation}\label{int_free_roller}
        \nu_s + \ddfrac{\nu_3}{\Lambda}\sin\chi_{ij} = \const
    \end{equation}
    Механический смысл этих интегралов заключается в сохранении проекций угловых скоростей роликов на их оси вращения. Действительно, из определений псевдоскоростей (\ref{eq:vsnu3},\ref{eq:nus}) следует, что  $\nu_3 = \Lambda\dot{\theta}, \enspace \nu_s = \dot{\phi}_{ij}$, и абсолютная угловая скорость ролика равна $\dot{\theta}\vec{e}_Z + \dot{\phi}_{ij}\vec{e}_{ij}$, где $\vec{e}_{ij} = \cos\chi_{ij}\vec{n}_i^\perp + \sin\chi_{ij}\vec{n}_i^z$ --- единичный направляющий вектор оси ролика, и при этом $\vec{e}_{ij} \cdot \vec{e}_Z = \cos\left(\ddfrac{\pi}{2} - \chi_{ij}\right) = \sin\chi_{ij}$.
    Получаем, что в левой части равенств (\ref{int_free_roller}) стоит проекция абсолютной угловой скорости ролика на его ось $\vec{e}_{ij}$.
    Таким образом, псевдоскорость $\nu_3$, пропорциональная проекции угловой скорости платформы на вертикаль $OZ$, связана со скоростями собственного вращения свободных роликов вокруг их осей. В частности, вращение экипажа вокруг вертикальной оси, проходящей через его центр (т.е. движение с начальными условиями $\nu_1(0) = 0, \nu_2(0) = 0, \nu_3(0) \neq 0$), неравномерно, в отличие от выводов, основанных на безынерционной модели \cite{ZobovaTatarinovPMM}.
    \item При одновременном умножении начальных значений всех псевдоскоростей на отличное от нуля число $\lambda$ получаются такие же уравнения движения, как при умножении времени $t$ на $\lambda$:
    % Одновременное умножение начальных значений всех псевдоскоростей на отличное от нуля число $\lambda$ эквивалентно растяжению времени:
    $$
        \vnu \mapsto \lambda\vnu, \enspace \lambda \neq 0 \enspace \sim \enspace t \mapsto \lambda t.
    $$
\end{enumerate}

