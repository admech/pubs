\chapter{Динамика экипажа на~омни-колесах на~плоскости с трением}

\blfootnote{Данная глава изложена на основании статей \cite{KosenkoGerasimovNd2016,KosenkoGerasimov2014,KosenkoGerasimovJsme2016,KosenkoGerasimov2015}, опубликованных в соавторстве с научным руководителем, предложившим  постановку задачи и используемые методы. Все результаты главы получены соискателем лично.}

В предыдущих главах рассмотрена динамика омни-колесного экипажа на абсолютно шероховатой плоскости. Было показано, что в этом случае на интервалах движения без смены ролика в контакте полная механическая энергия сохраняется, что, однако, никогда не происходит в реальных системах.
% Отдельного рассмотрения заслуживает вопрос о том, прекращается ли в реальных системах скольжение вновь вошедшего в контакт ролика.
Отдельного рассмотрения заслуживает вопрос о том, как возникает и прекращается в реальных системах скольжение вновь вошедшего в контакт ролика, либо ролика, уже находящегося в контакте.
Таким образом, интерес представляет изучение динамики экипажа по плоскости с трением.

В данной главе для моделирования контактного взаимодействия вместо идеальных неголомных связей отсутствия проскальзывания используются голономные неидеальные связи. Для касательных составляющих реакций опорной плоскости используются две модели: сухое трение Амонтона -- Кулона, регуляризованное в окрестности нуля по скорости проскальзывания линейной функцией насыщения; а также вязкое трение.

Модель экипажа строится с помощью системы построения численных моделей \texttt{Modelica}. В формализме языка \texttt{Modelica} разработчик самостоятельно задает систему дифференциально-алгебраи\-чес\-ких уравнений, описывающих динамику системы. Затем записанная вручную система уравнений автоматически приводится к виду, подходящему для численного интегрирования.

Построение модели экипажа в настоящей главе выполнено таким образом, что замена в ней модели контактного взаимодействия на другую требует изменения всего одного алгебраического выражения. Также подробно рассмотрен вопрос отслеживания контакта роликов и горизонтальной плоскости и алгоритмической реализации процесса переключения контакта от ролика к ролику при качении  роликонесущего колеса.

Динамические свойства построенной модели экипажа проиллюстрированы при помощи численных экспериментов. Проведена верификация построенной модели с сухим трением в сравнении с безынерционной моделью при стремлении суммарной массы роликов к нулю. Модель с вязким трением представлена в сравнении с неголономной моделью, построенной в главах 1 и 2.