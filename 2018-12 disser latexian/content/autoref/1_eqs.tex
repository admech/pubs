%\section{В некотором пространстве, в некотором подпространстве...}

В \textbf{первой главе} рассматривается движение экипажа по абсолютно шероховатой плоскости, т.е. предполагается, что проскальзывание между опорным роликом и плоскостью отсутствует. Получены уравнения движения экипажа с омни-колесами как системы абсолютно твердых тел --- платформы, колес и всех роликов --- в явном виде с использованием формализма лаконичных уравнений Я.В. Татаринова. Изучена структура уравнений, найдены первые интегралы и проведено сравнение с уравнениями движения безынерционной модели. Показано, что если момент инерции ролика относительно его оси равен нулю, то уравнения совпадают с уравнениями модели экипажа с дифференциальными связями, не учитывающей инерцию роликов -- безынерционной модели.

Рассматривается экипаж с омни-колесами с массивными роликами, движущийся по инерции по неподвижной абсолютно шероховатой горизонтальной опорной плоскости в поле силы тяжести. 

Омни-колесо в рассматриваемой постановке -- это система абсолютно твердых тел, включающая в себя плоский диск колеса и $n$ массивных роликов. Схема колеса приведена на фиг.~\ref{fig:wheel}; ролики на ней показаны в виде затемненной области и областей, ограниченных штриховой линией. Плоскость, содержащую диск колеса, будем называть плоскостью колеса. Каждый ролик может свободно вращаться вокруг оси, неподвижной относительно диска колеса. Оси роликов -- это прямые, лежащие в плоскости колеса и касательные к его окружности радиуса $r$. Поверхность ролика является поверхностью вращения дуги окружности радиуса $l > r$, лежащей в плоскости колеса с центром в центре диска вокруг хорды, лежащей на оси ролика. Экипаж состоит из платформы и $N$ одинаковых омни-колес. Платформа экипажа -- это плоский диск радиуса $R$. Таким образом, система состоит из $N(n+1) + 1$ абсолютно твердых тел.

\begin{figure}[h]
    \minipage{0.4\textwidth}
        \centering
        \asyinclude{./content/pic/asy/pic_wheel.asy}
        \caption{Колесо}
        \label{fig:wheel}
    \endminipage
    \minipage{0.5\textwidth}
        \centering
        \asyinclude{./content/pic/asy/pic_cart.asy}
        \caption{Экипаж}
        \label{fig:vehicle}
    \endminipage
\end{figure}

\filbreak
Положение экипажа задается следующими обобщенными координатами:
$x, y$ --- координаты точки $S$ на плоскости $OXY$, $\theta$ -- угол между осями $OX$ и $S\xi$ (будем называть $\theta$ углом курса),
$\chi_i$ ($i = 1,\dots,N$) -- углы поворота колес вокруг их осей, и $\phi_j$ -- углы поворота роликов вокруг их собственных осей.
Таким образом, вектор обобщенных координат имеет вид
\begin{equation}\label{eq:q}
    \vec{q} = (
        x, y, \theta,
        \left\{\chi_i\right\}|_{i=1}^N,
        \left\{\phi_k\right\}|_{k=1}^N,
        \left\{\phi_s\right\}|_{s=N + 1}^{Nn}
    )^{\mathop{T}}\in\mathbb{R}^{N(n+1) + 3},    
\end{equation}
где координаты $\phi$ сгруппированы таким образом, что сначала указаны углы поворота $\phi_k$ роликов, находящихся в данный момент в контакте с опорной плоскостью, a затем -- остальных, ``cвободных'', роликов. Индекс $s$ используется для сквозной нумерации свободных роликов.

Вводятся псевдоскорости
$$
    \vnu = (\nu_1, \nu_2, \nu_3, \left\{\nu_s\right\}),
$$
\begin{equation}\label{eq:vsnu3}
    \vec{v}_S = R\nu_1\vec{e}_\xi + R\nu_2\vec{e}_\eta, \quad \nu_3 = \Lambda\dot{\theta},
\end{equation}
\begin{equation}\label{eq:nus}
    \nu_s = \dot{\phi}_s, \quad s = N + 1,\ldots,Nn
\end{equation}

Их механический смысл таков: $\nu_1$, $\nu_2$ --- проекции скорости точки $S$ на оси системы $S\xi$ и $S\eta$, связанные с платформой, $\nu_3$ --- проекция угловой скорости платформы на вертикальную ось $SZ$ с точностью до безразмерного множителя $\Lambda$, $\nu_s$ --- угловые скорости свободных роликов. Число независимых псевдоскоростей системы равно $K = N(n-1)+3$.

Предполагается, что проскальзывания между опорной плоскостью и роликами в контакте не происходит, т.е. скорости точек роликов $C_i$, находящихся в контакте (см. фиг.~\ref{fig:wheel}) равны нулю:
\begin{equation}\label{eq:constraints_vec}
    \vec{v}_{C_i} = 0,\quad i = 1,\dots, N    
\end{equation}
Отсюда получены уравнения связей:
\begin{eqnarray}
\dot{\phi_k} &=& \frac{R}{\rho_k }(\nu_1\cos\alpha_k + \nu_2\sin\alpha_k); \quad \rho_k  = l\cos\chi_k - r \label{constraint_roller_contact}\\
\dot{\chi}_i &=& \frac{R}{l}(\nu_1\sin\alpha_i - \nu_2\cos\alpha_i - \frac{\nu_3}{\Lambda})\label{constraint_wheel_contact}
\end{eqnarray}
Заметим, что знаменатель $\rho_k$ в формуле (\ref{constraint_roller_contact}) -- расстояние от оси ролика до точки контакта, обращающееся в нуль на стыке роликов, то есть в случае, когда точка контакта $C_i$ оказывается на оси ролика (см. левую часть фиг.~\ref{fig:wheel}). Данное обстоятельство может приводить к разрывам второго рода функций в правых частях уравнений движения. Эта проблема рассмотрена отдельно ниже.

Уравнение (\ref{constraint_wheel_contact}) совпадает с уравнением связи в случае безынерционной модели роликов. 

%Таким образом, выражение обобщенных скоростей через псевдоскорости, учитывающее связи, наложенные на систему, можно записать в матричном виде:
%\begin{equation}\label{constraints_V}
%    \dot{\vec{q}} = \cstr\vnu,\quad \cstr = \cstr(\theta,\chi_i),
%\end{equation}

%где компоненты матрицы $\cstr$ имеют вид:
%$$
%\cstr = \begin{bmatrix}
%        \widetilde{V}  & O_1  \\[6pt]
%        O_2       & E         \\[6pt]
%    \end{bmatrix};
%\quad
%\widetilde{V} = \begin{bmatrix}
%        R\cos\theta                    & -R\sin\theta                    & 0                      \\[6pt]
%        R\sin\theta                    &  R\cos\theta                    & 0                      \\[6pt]
%        0                              & 0                               & \ddfrac{1}{\Lambda}    \\[6pt]
%        \ddfrac{R}{l}\sin\alpha_i      & -\ddfrac{R}{l}\cos\alpha_i      & -\ddfrac{R}{\Lambda l} \\[6pt]
%        \ddfrac{R}{\rho_k}\cos\alpha_k &  \ddfrac{R}{\rho_k}\sin\alpha_k & 0                      \\[6pt]
%    \end{bmatrix}
%$$
%Здесь $O_1$ и $O_2$ -- нулевые $(3+2n \times N(n-1))$- и $(N(n-1) \times 3)$-матрицы, $E$ -- единичная матрица размерности $N(n-1)$.

%\section{Уравнения движения}
%{\bf 3. Уравнения движения.}
%\stepcounter{section}
В работе используется лаконичный метод получения уравнений движения для систем с дифференциальными связями, предложенным Я.В.~Татариновым:
% \cite{Tatarinov}
\begin{equation}\label{Tatarinov}
    \frac{d}{dt}\frac{\partial L^{*}}{\partial \nu_\beta}  + \{P_\beta, L^{*}\} = \sum\limits_{\mu = 1}^{K}\{P_\beta, \nu_\mu P_\mu\},\quad \beta = 1,\dots, K
\end{equation}
Здесь $L$ -- лагранжиан, $L^*$ -- он же с учетом связей (здесь и далее верхний индекс $*$ означает учет связей, то есть подстановку выражений обобщенных скоростей через псевдоскорости), $P_\beta$ -- линейные комбинации формальных канонических импульсов $p_i$, определяемые из соотношения
\begin{equation}\label{eq:tatimp}
    \sum\limits_{\mu=1}^{K}\nu_\mu P_\mu \equiv \sum\limits_{i=1}^{N(n+1)+3}\dot{q_i} p_i
\end{equation}
в котором $\dot{q}_i$ выражены через псевдоскорости $\nu_\mu$ в соответствии с формулами связей; $\{\cdot, \cdot\}$ -- скобка Пуассона по $p_i$, $q_i$, после ее вычисления выполняется подстановка 
$$
    \hspace{10pt} p_i = \frac{\partial L}{\partial \dot{q}_i}.
$$

В силу симметрии системы и однородности всех тел, потенциальная энергия системы во время движения не меняется, и лагранжиан равен кинетической энергии:
\begin{equation}\label{kin_en}
    \hspace{-10pt}
    2L = 2T = M\vec{v}_S^2 + I_S\dot{\theta}^2 + J\sum_i\dot{\chi}_i^2 + B\sum_{i,j}(\dot{\phi}_{ij}^2 + 2\dot{\theta}\sin(\kappa_j + \chi_i)\dot{\phi}_{ij})=\dot{\vec{q}}^\mathrm{T}\M\dot{\vec{q}}
\end{equation}
Здесь $M,\ I_S,\ J$ --- массово-инерционные характеристики экипажа (его общая масса,  суммарный момент инерции системы относительно оси $SZ$ и момент инерции колеса относительно его оси вращения соответственно), $B$ --- момент инерции ролика относительно его оси вращения.
% Все слагаемые равенства (\ref{Tatarinov}) получаются непосредственным подсчетом и при необходимости, подстановкой связей. Окончательно 
Уравнения движения системы имеют следующую структуру:
\begin{eqnarray*}
    \hspace{-25pt}
    \M^*\dot{\vnu} = 
    \frac{MR^2}{\Lambda}\begin{bmatrix}
        \nu_2\nu_3\\
        -\nu_1\nu_3\\
        0\\
        0\\
        \vdots
        \\
        0
    \end{bmatrix}
    +\vnu^\mathrm{T}
    \left(
    \frac{R}{2l}
    \begin{bmatrix}
        -\M^*_i \sin\alpha_i\\
        \M^*_i \cos\alpha_i\\
        \M^*_i \Lambda^{-1}\\
        0\\
        \vdots
        \\
        0
    \end{bmatrix}
    -BR^2
    \begin{bmatrix}
        \Prhs_1\\
        \Prhs_2\\
        \Prhs_3\\
        0\\
        \vdots
        \\
        0
    \end{bmatrix}
    \right)
    \vnu
    -B\begin{bmatrix}
        \scalebox{1.5}{$\star$}\\
        \scalebox{1.5}{$\star$}\\
        \scalebox{1.5}{$\star$}\\
        \fontdimen16\textfont2=5pt
        \ddfrac{\nu_3}{\Lambda}\dot{\chi}_1^*\cos\chi_{12}\\
        \vdots
        \\
        \ddfrac{\nu_3}{\Lambda}\dot{\chi}_N^*\cos\chi_{Nn}
    \end{bmatrix}
\end{eqnarray*}
\begin{equation}\label{eq:full_system}
\end{equation}
\fontdimen16\textfont2=1.79999pt

Показано, что уравнения движения имеют следующие свойства:
\begin{enumerate}[wide]
    \item Система допускает интеграл энергии $\frac{1}{2}\vnu^\mathrm{T}\M^*(\chi_i)\vnu = h = \mathrm{const}$: так как система стеснена автономными идеальными связями, а силы консервативны, то полная энергия (в рассматриваемом здесь случае она равна кинетической энергии) сохраняется.
    
    \item Если платформа экипажа неподвижна, т.е. $\nu_1 = \nu_2 = \nu_3 = 0$, то свободные ролики сохраняют свою начальную угловую скорость: $\nu_s = \mathrm{const}$, чего и следовало ожидать.
   
    \item При $B = 0$ (ролики не имеют инерции) все слагаемые в правой части равенства~(\ref{eq:full_system}), кроме первого, обращаются в нуль, как и все члены, соответствующие свободным роликам, в его левой части. В этом случае существенными остаются лишь первые три уравнения системы относительно $\nu_1,\ \nu_2,\ \nu_3$. Важно отметить, что оставшиеся нетривиальные уравнения в точности совпадают с уравнениями движения безынерционной модели экипажа.
    \item Существовавший в безынерционной модели линейный первый интеграл разрушается для модели с массивными роликами. При $B = 0$ он имеет вид $m_{33}^*\nu_3 = \mathrm{const}$ (причем $m^*_{33} = \mathrm{const}$) и получается непосредственно из третьего уравнения системы\upr{eq:full_system}. Из вида интеграла следует, что $\nu_3 = \const$. При $B \neq 0$ скорость изменения $\nu_3$ пропорциональна моменту инерции ролика~$B$.
    \item Система допускает первые интегралы:
    \begin{equation}\label{int_free_roller}
        \nu_s + \ddfrac{\nu_3}{\Lambda}\sin\chi_{ij} = \const
    \end{equation}
    Механический смысл этих интегралов заключается в сохранении проекций угловых скоростей роликов на их оси вращения.
    % Таким образом, псевдоскорость $\nu_3$, пропорциональная проекции угловой скорости платформы на вертикаль $OZ$, связана со скоростями собственного вращения свободных роликов вокруг их осей. В частности, вращение экипажа вокруг вертикальной оси, проходящей через его центр (т.е. движение с начальными условиями $\nu_1(0) = 0, \nu_2(0) = 0, \nu_3(0) \neq 0$), неравномерно, в отличие от выводов, основанных на безынерционной модели.
    \item При одновременном умножении начальных значений всех псевдоскоростей на отличное от нуля число $\lambda$ получаются такие же уравнения движения, как при умножении времени $t$ на $\lambda$:
    $$
        \vnu \mapsto \lambda\vnu, \enspace \lambda \neq 0 \enspace \sim \enspace t \mapsto \lambda t.
    $$
\end{enumerate}