% \ddfrac command to show big fractions, not cramped up
% https://tex.stackexchange.com/questions/173899/
%--------------------------------------
\newcommand\ddfrac[2]{\displaystyle\frac{\displaystyle #1}{\displaystyle #2}}
%--------------------------------------

% \vsp command to make a spacey newline
% useful for equations arrays
%--------------------------------------
\newcommand\vsp[1][10]{\\[#1pt]}
%--------------------------------------

% partial derivatives (can \usepackage{physics}, but only one command so far, so no)
%--------------------------------------
\newcommand\pd[2]{\frac{\partial #1}{\partial #2}}
\newcommand\ddpd[2]{\ddfrac{\partial #1}{\partial #2}}
\newcommand\ddt[1]{\frac{d #1}{dt}}
\newcommand\ddddt[1]{\ddfrac{d #1}{dt}}
%--------------------------------------

% unbreakable space parenthesized reference
%--------------------------------------
\newcommand\upr[1]{~(\ref{#1})}
%--------------------------------------

% Nice letters
%--------------------------------------
\newcommand\M[0]{\mathcal{M}} % Matrix of intertia
\newcommand\const{\mathrm{const}} %константа
\newcommand\AntiU[0]{\mathcal{U}} % Helper antisymmetric matrix for eqs' RHS
\newcommand\Rhs[0]{\mathcal{R}} % RHS
\newcommand\Prhs[0]{\mathcal{P}} % The family of matrices for RHS
\newcommand\prhs[0]{\mathbf{p}} % Poisson brackets
\newcommand\vnu[0]{\text{\textbf{\textupsilon}}} % Upright greek vector nu for PMM
%--------------------------------------

% change line spacing mid doc (affects global line spacing)
%--------------------------------------
% \usepackage{setspace}
%--------------------------------------


\renewcommand{\vec}[1]{\boldsymbol{\mathbf{#1}}}

\newtheorem{stmt}{Утверждение}
\newtheorem{prblm}{Затруднение}

\renewcommand{\thesection}{\arabic{section}}
\renewcommand{\theequation}{\arabic{section}.\arabic{equation}}

\linespread{1.2}

\newcounter{Section}
\setcounter{Section}{0}
\newcommand{\sect}[1]{\stepcounter{Section}\textbf{\arabic{Section}. #1}}


% \newcommand{\imm}[1]{ \ifmmode #1 \fi }

% ------------------------------------------------------------------
% SYMBOLS
% ------------------------------------------------------------------

\newcommand{\q}{\mathbf{q}}
\newcommand{\qdo}{\q^-}
\newcommand{\qposle}{\q^+}

\newcommand{\dotq}{\dot{\mathbf{q}}}
\newcommand{\dqdo}{\dotq^-}
\newcommand{\dqposle}{\dotq^+}

\newcommand{\nuvec}{\vec{\nu}} % {\vec{\nu}}
\renewcommand{\vnu}{\nuvec}
\newcommand{\nudo}{\nuvec^-}
\newcommand{\nuposle}{\nuvec^+}

\newcommand{\subspace}{\widetilde{V}}
\newcommand{\deltadq}{\Delta\dotq}

\newcommand{\cstr}{\mathbf{V}}
\newcommand{\mke}{\mathbf{M}}
\newcommand{\mK}{\mathbf{K}}
\renewcommand{\O}{\mathbf{O}}
\newcommand{\E}{\mathbf{E}}

% not bold so that matrices themselves and their blocks were denoted uniformly
% \newcommand{\cstr}{V}
% \newcommand{\mke}{M}
% \newcommand{\mK}{K}

\newcommand{\mvk}{\left(\mke\cstr \enspace -\mK\right)}

\newcommand{\dotp}[2]{\left( #1, \enspace #2 \right)}
\newcommand{\mdotp}[2]{(#1)^T #2}
\newcommand{\inv}[1]{\left( #1 \right)^{-1}}

\newcommand{\ke}{\mathbf{T}}

\newcommand{\Q}{\mathbf{Q}}
\newcommand{\F}{\mathbf{F}}

\newcommand{\J}{\mathbf{J}}
\newcommand{\A}{\mathbf{A}}
\newcommand{\lagr}{\vec{\lambda}}

% ------------------------------------------------------------------
% EQUATIONS
% ------------------------------------------------------------------

% Impact -- classical

\newcommand
{ \eqDeltaqQ }
{ \mathbf{M} (\dot{\mathbf{q}}^+ - \dot{\mathbf{q}}^-) = \mathbf{Q} }

\newcommand
{ \eqQKF }
{ \mathbf{Q} = \mathbf{K}\mathbf{F} }

\newcommand
{ \eqQKFmat }
{ \eqQKF = \K \Fcomp }
% { \eqQKF, \quad \mK = \mKblock, \quad \eqF }

\newcommand
{ \eqqnu }
{ \dqposle = \mathbf{V}\nuposle }

\newcommand
{ \eqMVnuKF }
{ \mathbf{M}\mathbf{V}\nuposle - \mathbf{K}\mathbf{F} = \mathbf{M}\dqdo }

\newcommand{\nuFT}{\left(\nuposle ; \mathbf{F}\right)^T}

\newcommand
{ \eqMVnuKFmat }
{ \mvk \nuFT = \mathbf{M}\dqdo }

\newcommand
{ \eqMVnuKFmatres }
{ \nuFT = \mvk^{-1} \mke\dqdo }

% Impact -- analytical

\newcommand
{ \eqDelta }
{ \deltadq = \cstr\nuposle - \dqdo }

\newcommand
{ \logicDeltaOrth }
{ 0 = \cstr^T\mke\deltadq = \cstr^T\mke(\cstr\nuposle - \dqdo) = \cstr^T\mke\cstr\nuposle - \cstr^T\mke\dqdo }

\newcommand
{ \eqnuposleproj }
{ \nuposle = \inv{\cstr^T\mke\cstr}\cstr^T\mke\dqdo }

% Energy

\newcommand
{ \eqTquad }
{ \ke = \ddfrac{1}{2}\dotp{\mke\dotq}{\dotq} }

\newcommand
{ \eqDqVnu }
{ \dqposle = \cstr\nuposle }

\newcommand
{ \edQposleDelta }
{ \dqposle = \dqdo + \deltadq }

\newcommand
{ \eqDqposleInSubspace }
{ \dqposle \in \subspace }

\newcommand
{ \eqDeltaOrth }
{ \deltadq \perp_\mke \subspace }

\newcommand
{ \edMkeSim }
{ \mke = \mke^T }

\newcommand
{\logicWorkZero}
{\dotp{\mke\dqposle}{\deltadq} = \dotp{\mke\deltadq}{\dqposle} = \dotp{\Q}{\dqposle} = 0}

\newcommand
{\logicDeltaT}
{ 2\Delta\ke & = & 2\left(\ke^+ - \ke^-\right) = \dotp{\mke\dqposle}{\dqposle} - \dotp{\mke\dqdo}{\dqdo} = \dotp{\mke\deltadq}{\deltadq} + 2\dotp{\mke\dqdo}{\deltadq} \\
 & = & -\dotp{\mke\deltadq}{\deltadq} + 2\dotp{\mke\dqposle}{\deltadq} = -\dotp{\mke\deltadq}{\deltadq} }

\newcommand
{\eqDeltaT}
{ \Delta\ke = -\ddfrac{1}{2}\dotp{\mke\deltadq}{\deltadq} \leq 0 }

% ------------------------------------------------------------------
% EXPLICIT EXPRESSIONS
% ------------------------------------------------------------------

\newcommand
{\Kthree}
{
\begin{bmatrix}
1 & 0 & 1 & 0 & 1 & 0\cr
0 & 1 & 0 & 1 & 0 & 1\cr
-\mathrm{sin}\left( \theta+{\alpha_1}\right)  R & \mathrm{cos}\left( \theta+{\alpha_1}\right)  R & -\mathrm{sin}\left( \theta+{\alpha_2}\right)  R & \mathrm{cos}\left( \theta+{\alpha_2}\right)  R & -\mathrm{sin}\left(\theta+{\alpha_3}\right)  R & \mathrm{cos}\left(\theta+{\alpha_3}\right)  R \cr
-l \mathrm{sin}\left(\theta+{\alpha_1}\right)  & l \mathrm{cos}\left(\theta+{\alpha_1}\right)  & 0 & 0 & 0 & 0\cr
0 & 0 & -l \mathrm{sin}\left(\theta+{\alpha_2}\right)  & l \mathrm{cos}\left(\theta+{\alpha_2}\right)  & 0 & 0\cr 0 & 0 & 0 & 0 & -l \mathrm{sin}\left(\theta+{\alpha_3}\right)  & l \mathrm{cos}\left(\theta+{\alpha_3}\right) \cr
-\rho_1  \mathrm{cos}\left(\theta+{\alpha_1}\right)  & -\rho_1  \mathrm{sin}\left(\theta+{\alpha_1}\right)  & 0 & 0 & 0 & 0\cr
0 & 0 & -\rho_2  \mathrm{cos}\left(\theta+{\alpha_2}\right)  & -\rho_2  \mathrm{sin}\left(\theta+{\alpha_2}\right)  & 0 & 0\cr
0 & 0 & 0 & 0 & -\rho_3  \mathrm{cos}\left(\theta+{\alpha_3}\right)  & -\rho_3  \mathrm{sin}\left(\theta+{\alpha_3}\right)
\end{bmatrix}
}

\newcommand
{\K}
{
\begin{bmatrix}
1 & 0 & \ldots & 1 & 0 \\
0 & 1 & \ldots & 0 & 1 \\
-R\sin\left( \theta + \alpha_1 \right) & R\cos\left( \theta + \alpha_1 \right) & \ldots & -R\sin\left( \theta + \alpha_N \right) & R\cos\left( \theta + \alpha_N \right) \\
-l\sin\left( \theta + \alpha_1 \right) & l \cos\left( \theta + \alpha_1 \right) & \ldots & 0 & 0 \\
0 & 0 & \ddots & 0 & 0 \\
0 & 0 & \ldots & -l \sin\left( \theta + \alpha_N \right) & l \cos\left( \theta + \alpha_N \right) \\
-\rho_1 \cos\left( \theta + \alpha_1 \right) & -\rho_1 \sin\left( \theta + \alpha_1 \right) & \ldots & 0 & 0 \\
0 & 0 & \ddots & 0 & 0 \\
0 & 0 & \ldots & -\rho_N \cos\left( \theta + \alpha_N \right) & -\rho_N \sin\left( \theta + \alpha_N \right) \\
0 & 0 & \ldots & 0 & 0 \\
0 & 0 & \ddots & 0 & 0 \\
0 & 0 & \ldots & 0 & 0 \\
\end{bmatrix}
}

\newcommand {\Fix} { F^x_i }
\newcommand {\Fiy} { F^y_i }
% \newcommand {\eqQiOne} { Q^i_1 & = & \Fix }
% \newcommand {\eqQiTwo} { Q^i_2 & = & \Fiy }
% \newcommand {\eqQiTheta} { Q^i_\theta & = & R \left(-\Fix\sin(\theta + \alpha_i) + \Fiy\cos(\theta + \alpha_i)\right) }
\newcommand {\eqQiOne} { Q_1 & = & \sum_{i=1}^N \Fix }
\newcommand {\eqQiTwo} { Q_2 & = & \sum_{i=1}^N \Fiy }
\newcommand {\eqQiTheta} { Q_\theta & = & \sum_{i=1}^N Q^i_\theta \enspace = \enspace \sum_{i=1}^N R \left(-\Fix\sin(\theta + \alpha_i) + \Fiy\cos(\theta + \alpha_i)\right) }
\newcommand {\eqQChii} { Q_{\chi_i} & = & \ddfrac{l}{R} Q^i_\theta }
\newcommand {\eqQPhii} { Q_{\phi_i} & = & -\rho_i \left( \Fix\cos(\theta + \alpha_i) + \Fiy\sin(\theta + \alpha_i) \right) }
\newcommand {\eqQs} { Q_s & = & 0 }

\newcommand
{ \eqQ }
% { \Q = \left( \enspace \sum_{i=1}^N Q^i_1, \enspace \sum_{i=1}^N Q^i_2, \enspace \sum_{i=1}^N Q^i_\theta, \enspace Q_{\chi_i}|_{i=1}^N, \enspace Q_{\phi_i}|_{i=1}^N, \enspace Q_s \enspace \right)^T }
{ \Q = \left( \enspace Q_1, \enspace Q_2, \enspace Q_\theta, \enspace Q_{\chi_i}|_{i=1}^N, \enspace Q_{\phi_i}|_{i=1}^N, \enspace Q_s \enspace \right)^T }

\newcommand
{ \eqF }
{ \F = \left( \enspace F_x^1, \enspace F_y^1, \enspace \ldots, \enspace F_x^N, \enspace F_y^N \enspace \right)^T }

\newcommand
{ \Fcomp }
{
    \begin{bmatrix}
        F_x^1 \\
        F_y^1 \\
        \vdots \\
        F_x^N \\
        F_y^N
    \end{bmatrix}
}

\newcommand
{ \mKblock }
{ \left( \enspace \mK_1, \enspace \mK_N \enspace \right) }

\newcommand
{ \mKone }
{
    \begin{bmatrix}
        1 & 0 \\
        0 & 1 \\
        -R\sin\left( \theta + \alpha_1 \right) & R\cos\left( \theta + \alpha_1 \right) \\
        -l\sin\left( \theta + \alpha_1 \right) & l\cos\left( \theta + \alpha_1 \right) \\
        0 & 0 \\
        \ldots & \\
        0 & 0 \\
        -\rho_1 \cos\left( \theta + \alpha_1 \right) & -\rho_1 \sin\left( \theta + \alpha_1 \right) \\
        0 & 0 \\
        \ldots & \\
        0 & 0 
    \end{bmatrix}
}

\newcommand
{ \mKtwo }
{
    \begin{bmatrix}
        1 & 0 \\
        0 & 1 \\
        -R\sin\left( \theta + \alpha_2 \right) & R\cos\left( \theta + \alpha_2 \right) \\
        0 & 0 \\
        -l\sin\left( \theta + \alpha_2 \right) & l\cos\left( \theta + \alpha_2 \right) \\
        0 & 0 \\
        0 & 0 \\
        -\rho_2 \cos\left( \theta + \alpha_2 \right) & -\rho_2 \sin\left( \theta + \alpha_2 \right) \\
        0 & 0 
    \end{bmatrix}
}

\newcommand
{ \mKN }
{
    \begin{bmatrix}
        1 & 0 \\
        0 & 1 \\
        -R\sin\left( \theta + \alpha_N \right) & R\cos\left( \theta + \alpha_N \right) \\
        0 & 0 \\
        \ldots & \\
        0 & 0 \\
        -l\sin\left( \theta + \alpha_N \right) & l\cos\left( \theta + \alpha_N \right) \\
        0 & 0 \\
        \ldots & \\
        0 & 0 \\
        -\rho_N \cos\left( \theta + \alpha_N \right) & -\rho_N \sin\left( \theta + \alpha_N \right)
    \end{bmatrix}
}


% \newcommand{\subf}[2]{\minipage{#1} #2 \endminipage}
\newcommand{\subf}[2]{\begin{subfigure}[H]{#1} #2 \end{subfigure}}

\newcommand{\rk}{\mathrm{rk}}