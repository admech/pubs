\chapter*{Введение}

Омни-колеса (в русской литературе также используется название роликонесущие колеса) -- это колеса особой конструкции, позволяющей экипажу двигаться в произвольном направлении, вращая колеса вокруг их собственных осей и не поворачивая их вокруг вертикали. На ободе такого колеса располагаются ролики, которые могут свободно вращаться вокруг своих осей, жестко закрепленных в диске колеса. Существуют два варианта расположения осей роликов: первый  (собственно омни-колеса) -- оси роликов являются касательными к ободу колеса и, следовательно, лежат в его плоскости; второй (\textit{mecanum wheels} \cite{mecanum}) -- оси роликов развернуты вокруг нормали к ободу колеса на постоянный угол, обычно $\pi/4$.

Ранее была рассмотрена динамика омни-экипажей с использованием упрощенных моделей омни-колес без учета инерции и формы роликов \cite{ZobovaTatarinovPMM, formalskii, borisov, ZobovaTatarinovAspecty2006, zobova2008svobodnye8020851, Martynenko2010}, колеса (без роликов) моделируются как жесткие диски, которые могут скользить в одном направлении и катиться без проскальзывания в другом. Далее будем называть такую модель безынерционной, в том смысле, что инерция собственного вращения роликов в ней не учитывается. В другой части работ по динамике омни-экипажа \cite{KosenkoGerasimov, Tobolar, Williams2002, Ashmore2002} используются некоторые формализмы для построения численных моделей систем тел. При этом явный вид уравнений движения оказывается скрытым, что делает невозможным непосредственный  анализ уравнений и затрудняет оценку влияния разных факторов на динамику системы.


Цель настоящей работы -- получение в явном виде уравнений движения по инерции экипажа с омни-колесами с массивными роликами в неголономной постановке с помощью подхода Я.В. Татаринова \cite{Tatarinov}, исследование их свойств и сравнение поведения такой системы с поведением системы в безынерционном случае \cite{Zobova2011}.