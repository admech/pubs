% \filbreak
ОБЗОР ЛИТЕРАТУРЫ
% \begin{itemize}
%     \item Омни-колеса
%     \begin{itemize}
%         \item Определение омни-колеса
%         \item Работы по омни-экипажам в постановке без роликов
%         \item Работы Адамова в постановке с одним роликом
%         \item Работы о технических реализациях и front-to-back
%         \item Работы об управлении
%         \item Работы о шарах
%     \end{itemize}
%     \item Системы тел и удары
%     \begin{itemize}
%         \item Сложностей с количеством роликов и сменами контакта
%         \item Подходы к описанию динамики систем тел
%         \item Основания теории удара
%         \item Односторонние связи и удары в системах тел
%         \item Формализм языка Modelica
%         \item Работы по омни-колесам на Modelica
%     \end{itemize}
% \end{itemize}


Одной из широких областей, в которых омни-колеса находят применение, является изучение мобильных роботов \cite{Seeni2010,Martynenko2005,Martynenko2007,GolubevSnake2004}. Экипажи с омни-колесами и колесами \textit{mecanum} \cite{Ilon} подробно описаны в обзорных работах в этой области \cite{Campion1996,Zimmermann2009,ChungIagnemma2016,Kanjanawanishkul2015,Adascalitei2011} как с точки зрения теоретической механики, так и с позиции технической реализации систем. Геометрия поверхности ролика отдельно рассмотрена в \cite{Gfrerrer2008}, где также выполнена аппроксимация последней поверхности тором и сформулировано условие корректности конфигурации омниколесного экипажа: оси всех роликов, контактирующих с опорной плоскостью, не должны проходить через одну точку или быть параллельны, иначе экипаж не способен совершать ряд движений.

Движение экипажа с омни-колесами по абсолютно шероховатой плоскости рассмотрено в \cite{ZobovaTatarinovAspecty2006,ZobovaTatarinov2009,zobova2008svobodnye8020851,ZobovaTatarinovPMM,Zobova2011e}. Уравнения движения произвольной конфигурации экипажа получены, например, в \cite{ZobovaTatarinov2009}, где также найдены их интегралы и инвариантная мера. Для конфигурации с тремя колесами в двух перпендикулярных плоскостях изучается устойчивость движений. Составлены уравнения управляемого движения методом Я.В.Татаринова (уравнения в лаконичной форме) \cite{Tatarinov,Tatarinov2005}. В \cite{Zobova2011} этот метод получения уравнений движения  проиллюстрирован на примере омниколесного экипажа, а также рояльного колеса и экипажа с дифференциальным приводом.

В работе \cite{MartynenkoFormalskii2007,formalskii} рассмотрена симметричная конфигурация экипажа с омни-колесами, в которой центры колес расположены в углах правильного треугольника, а их плоскости вертикальны и перпендикулярны радиусам-векторам центров колес, выпущенным из центра треугольника. Описана ее кинематика, и рассматриваются движения по инерции и при постоянных напряжениях, подаваемых на моторы постоянного тока, установленные в осях колес. Дана оценка мощности, потребляемой моторами, и показано, что она наименьшая при движении в направлении оси одного из колес. Строится алгоритм отслеживания направления движения экипажа. Развитием этой работы стало рассмотрение экипажа со смещенным центром масс \cite{Martynenko2010_rus,Martynenko2010}, где построены траектории свободного движения и изучены вопросы существования движений по прямой и по окружности.

Движение экипажа произвольной конфигурации по инерции по абсолютно шероховатой плоскости рассматривается и в \cite{Borisov2011}, где также строятся различные примеры движения по инерции, в том числе, периодического; получены уравнения движения омниколесного экипажа на сфере. Позже в \cite{KilinBobykin2014} изучена управляемость экипажа произвольной конфигурации на абсолютно шероховатой плоскости и осуществимость движения по любой наперед заданной траектории.

Кроме экипажей, двигающихся за счет взаимодействия колес и опорной поверхности, изучаются и другие конструкции, например, шарообразные роботы, управляемые изнутри симметричным омниколесным экипажем \cite{Karavaev2015},
% Подходы к управлению: квази-статический и с переходными этапами.
либо двумя омниколесами,  установленными на сфере меньшего радиуса, находящейся внутри внешней сферы \cite{Ivanov2015a}. Последняя статья содержит также более широкий обзор литературы о роботах-шарах. Кроме того, в ней найдены условия, при которых возможно движение робота-шара вдоль произвольной траектории. Работа выполнена в формализме алгебры кватернионов. Известны и конструкции шарообразных роботов, управляемых двумя обычными колесами \cite{Zhan2011}, однако роботы, рассмотренные в \cite{Ivanov2015a}, имеют преимущество в способности нести полезный груз. Омни-колеса можно применять не только для перемещения в пространстве, но и для изменения ориентации тел. К примеру, в \cite{Weiss2015,Plumpton2014} предлагается использовать сферу, приводимую в движение омни-колесами, касающимися ее извне, в качестве корпуса тренажера для пилотов. В работе \cite{Plumpton2014} рассматривается точечный контакт колес и сферы, в \cite{Weiss2015} их взаимодействие задается в контактной модели Герца.

Инерцией движения роликов в большинстве работ  пренебрегают. Однако в работе \cite{Adamov2018}, вышедшей летом 2018 года, рассматривающей движение экипажа с четырьмя колесами \textit{mecanum}, учтено движение контактного ролика с учетом вязкого трения в осях колес и роликов. Строятся уравнения движения Аппеля. При этом предполагается, что точка контакта всегда находится строго под центром колеса. Изучается структура управляющих моментов на примере движения экипажа по окружности, а также устойчивость движения в линейном приближении. Отдельной областью интересов является определение коэффициентов в уравнениях движения \cite{Adamov2018a} в случаях, когда технические характеристики систем оказываются неизвестны, либо изменяются в процессе движения.

Отметим отдельно многочисленные работы по омни-роботам, содержащие описания практических реализаций  экипажей. Такие экипажи часто используются на соревнованиях мобильных роботов. К примеру, в \cite{Indiveri2007} описывается кинематика и строится управление симметричным омниколесным экипажем в условиях ограниченности моментов, прилагаемых двигателями. В \cite{Wada2007} строится гибридный экипаж с двумя обычными колесами и двумя роликонесущими. Весьма распространены работы, описывающие  низкоуровневую техническую реализацию экипажей, такие как \cite{Mohamed2017,Krishnaraj2017,SalamAl-Ammri2010}. В практике мобильных роботов необходимой задачей является навигация. В \cite{Eng2010} рассматривается способ навигации омни-колесных экипажей с помощью так называемого многочастичного фильтра \cite{Gordon1993}, широко распространенного в робототехнике метода решения нелинейных задач оценивания \cite{DelMoral1997}. Для омни-колес важен характер поверхности, по которой экипаж движется. Поэтому в работе \cite{Vicente2015} строится метод определения типа материала опорной поверхности с помощью оценивания вибраций при движении с целью адаптации управления: движение по мягким поверхностям естественным образом оказывается медленнее, а движение по жестким вызывает б\textit{о}льшие вибрации, что требуется компенсировать управлением. Предлагается также модель, в которой колеса экипажа подпружинены для компенсации неровности поверхности \cite{NguenMAI2012}.

Следующие работы посвящены оптимальному управлению движением омни-экипажей.
В работе \cite{Ashmore2002a} показано, что перемещение омни-экипажа между двумя точками на плоскости происходи быстрее всего не по прямой, а по дуге окружности. В \cite{Balkcom2006} оптимальные по времени траектории рассмотрены существенно детальнее, их построение проводится с помощью принципа максимума Понтрягина, и строится минимальная полная классификация таких траекторий. Класс работ о построении управляемых движений весьма широк \cite{Huang2015,Bramanta2017,Kalmar-Nagy2016,Szayer2017}. Встречаются работы, рассматривающие ситуацию частичного отказа приводов \cite{Field2017,Ivanov2015a}.

Интересны работы, описывающие все стадии разработки робототехнической платформы с омни-колесами, от кинематики и уравнений движения до построения (оптимального) управления, включающие также технические реализации экипажей \cite{Williams2002,Purwin2006,Li2009}. В \cite{Galicki2009} построено управление с объездом препятствий. В \cite{Lin2013} рассматривается адаптивное управление с учетом переменных коэффициентов трения в точках контакта, а также массы платформы.

Перейдем теперь к обзору формализмов, используемых при построении динамических уравнений систем твердых тел. 

Для описания систем многих тел, в том числе, систем, организованных иерархически, известны различные подходы \cite{Wittenburg2008,EberhardSchiehlen,Jain2011,Roberson1988,Jerkovsky1977}. Классические подходы основаны на теории графов \cite{Wittenburg2008,Jerkovsky1977}. Разработаны рекурсивные методы для описания древовидных структур \cite{EberhardSchiehlen}. Весьма обширный обзор существующих методов для описания систем тел, в том числе, с замкнутыми кинематическими цепями, проведен в \cite{Jain2011}. Работа \cite{Roberson1988}, кроме непосредственно методов описания систем тел, уделяет отдельное внимание историческому контексту развития данной области, в частности, констатируя слабопреодолимые затруднения, возникающие в аналитическом исследовании из-за нелинейностей и количества тел в системах, а также подробно освещая их разрешение с помощью вычислительной техники, как основной метод их изучения и проектирования.

В отношении контактного взаимодействия твердых тел, при описании динамики омни-колес и экипажей можно либо идти по пути наложения дифференциальных связей отсутствия проскальзывания, либо вводить силу трения в контакте. Динамика систем с дифференциальными связями подробно описана, например, в \cite{Chaplygin1949,NejmarkFufaev1967}. В \cite{NejmarkFufaev1967} и \cite{karapetyan1981negolonom}, в частности, подробно обсуждается вопрос обоснованности подобных идеализаций. В главах 1 и 2 настоящей работы принимается модель точечного контакта ролика и опорной плоскости. Для получения уравнений движения таких систем часто пользуются методами Аппеля либо Лагранжа первого рода \cite{KarapetyanKugushev2010,Appel1,AppelTwo1960}. В силу объема требуемых выкладок в рассматриваемой системе мы применили метод получения уравнений движения в лаконичной форме Я.В. Татаринова \cite{Tatarinov,Tatarinov2005}.

В отсутствии проскальзывания отдельного рассмотрения требует смена ролика в контакте с опорной плоскостью. Явления, возникающие при подобных движениях, описываются теорией удара \cite{AppelTwo1960,Vilke,KozlovTreshevBilliardsBook1991}. В \cite{KozlovTreshevBilliardsBook1991} приведено формальное построение этой теории, и обсуждается ее физическая обоснованность. Всестороннее современное рассмотрение механики систем с односторонними связями с учетом их моделирования с помощью вычислительной техники содержится в \cite{BrogliatoBook1999}, а также в серии работ \cite{PfeifferGlockerBook,PfeifferGlocker1995,Pfeiffer1999a,PfeifferGlockerSymposium1999,Pfeiffer1997,Glocker1999,Pfeiffer2001,Brogliato2002,Pfeiffer2003,FloresGlocker2011,Zbiciak2014}. Уравнения движения для систем с односторонними связями в интегральной форме построены в \cite{Kugushev2003}, и там же рассмотрены удары, в частности, о дифференциальные связи.

Постановка задачи с сухим трением в контакте между роликом и опорной плоскстью приводит к дальнейшему усложнению задачи. Методы исследования истем с сухим трением описаны в \cite{PfeifferGlocker1993,Pfeiffer1996,Anitescu1997,Lacoursiere2011,Charles2014,Paoli2015,Moreau1988}, а также \cite{Novozhilov1991} Исследование омниколесного экипажа с учетом динамики всех роликов и трения возможно лишь численно, и здесь, кроме понимания природы трения, требуется подходящий формализм для создания компьютерной модели экипажа. Нами был выбран формализм языка Modelica \cite{ModelicaSpec,Dymola,Fritzson}, примененный ранее к задачам динамики систем тел \cite{Kosenko1998,KosenkoQuaternionRus,Kosenko2006unilat,Kosenko2006,Kosenko2007,KosenkoGraphs2009,KosenkoGusev2012,KosenkoRolling,Kosenko2006,KosenkoAlexandrov,KosenkoKuznetzova}. Этот метод уже использовался и при изучении динамики омниэкипажей \cite{Kalman2012assistant,Kalman2013control,Kalman2013braking,Kalman2013practical}. В этих работах рассматривается контакт с трением, но ролики имеют существенно упрощенную форму -- считаются массивными цилиндрами либо конусами. Автор утверждает, что полученная им модель слишком медленна и сложна, и потому строит еще одну упрощенную модель, полностью пренебрегая инерцией роликов, но учитывая  трение в направлении оси ролика. В \cite{Kalman2013braking} изучается вопрос заноса омни-колесных экипажей при торможении в упрощенной модели.

Первая глава представленной работы, где получены уравнения движения экипажа с массивными роликами без проскальзывания, принята к печати \cite{GerasimovZobovaPMM2018}. Вторая глава, содержащая модель перехода с одного ролика на другой с позиции теории удара, принята к печати \cite{GerasimovZobovaTrudyMAI2018}. В работах \cite{KosenkoGerasimovNd2016,KosenkoGerasimovJsme2016} опубликованы результаты третьей главы, в которой экипаж моделируется в постановке с регуляризованным сухим трением в точечном контакте ролика и опорной плоскости. Эти результаты доложены  конференциях и опубликованы в их сборниках трудов \cite{Kosenko2014unilateral,KosenkoGerasimov2014,Kosenko201construction,Kosenko2015verification,Kosenko2015hierarchy,KosenkoGerasimov2015,Kosenko2016testbench,Kosenko2016ND}.\\

% Ранее была рассмотрена динамика омни-экипажей с использованием упрощенных моделей омни-колес без учета инерции и формы роликов \cite{ZobovaTatarinovPMM, formalskii, borisov, ZobovaTatarinovAspecty2006, zobova2008svobodnye8020851, Martynenko2010}, колеса (без роликов) моделируются как жесткие диски, которые могут скользить в одном направлении и катиться без проскальзывания в другом. Далее будем называть такую модель безынерционной, в том смысле, что инерция собственного вращения роликов в ней не учитывается. В другой части работ по динамике омни-экипажа \cite{KosenkoGerasimov, Tobolar, Williams2002, Ashmore2002} используются некоторые формализмы для построения численных моделей систем тел. При этом явный вид уравнений движения оказывается скрытым, что делает невозможным непосредственный  анализ уравнений и затрудняет оценку влияния разных факторов на динамику системы.

