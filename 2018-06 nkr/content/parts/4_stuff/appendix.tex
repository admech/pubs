\section{Приложение}

\stepcounter{section}
Матрица кинетической энергии:
%$$
%\M = \begin{bmatrix}
%    M &   &               &   &        &   &                        &        &                        \\
%      & M &               &   &        &   &                        &        &                        \\
%      &   & I_S           &   & \cdots &   & B\sin(\chi_k+\kappa_1) & \cdots & B\sin(\chi_N+\kappa_n) \\
%      &   &               & J &        &   &                        &        &                        \\
%      &   &               &   & \ddots &   &                        &        &                        \\
%      &   &               &   &        & J &                        &        &                        \\
%      &   & \text{\huge*} &   &        &   & B                      &        &                        \\
%      &   &               &   &        &   &                        & \ddots &                        \\
%      &   &               &   &        &   &                        &        & B                      \\
%\end{bmatrix},
%$$
$$
\M = \begin{bmatrix}
    \widetilde{\M}_{11}   & \widetilde{\M}_{12}   & \widetilde{\M}_{13} \\
    \widetilde{\M}_{12}^T & \widetilde{\M}_{22}   & \widetilde{\M}_{23} \\
    \widetilde{\M}_{13}^T & \widetilde{\M}_{23}^T & \widetilde{\M}_{33} \\
\end{bmatrix}
$$
где
$$
\widetilde{\M}_{11} = \text{diag}(M, M, I_S),
\quad
\widetilde{\M}_{22} = JE_{N \times N},
\quad
\widetilde{\M}_{33} = BE_{Nn \times Nn}
$$
$$
\widetilde{\M}_{12} = O_{3 \times N},
\quad
\widetilde{\M}_{13} = \begin{bmatrix}
        0                      & \cdots & 0                      \\
        0                      & \cdots & 0                      \\
        B\sin\chi_{11}         & \cdots & B\sin\chi_{Nn}         \\
    \end{bmatrix},
\quad
\widetilde{\M}_{23} = O_{N \times Nn}
\vsp
$$
% $$
% \M = \begin{bmatrix}
%     M & 0 & 0             & 0 & \cdots & 0 & 0                      & \cdots &0                       \\
%       & M & 0             & 0 & \cdots & 0 & 0                      & \cdots & 0                      \\
%       &   & I_S           & 0 & \cdots & 0 & B\sin\chi_{11}         & \cdots & B\sin\chi_{Nn}         \\
%       &   &               & J &        &   &                        &        &                        \\
%       &   &               &   & \ddots &   &                        & \text{\huge 0}&                 \\
%       &   &               &   &        & J &                        &        &           \\
%       &   & \scalebox{1.5}{$\star$} &   &        &   & B                      &        &                        \\
%       &   &               &   &        &   &                        & \ddots &                        \\
%       &   &               &   &        &   &                        &        & B                      \\
% \end{bmatrix}
% $$
Здесь $O_{(\boldsymbol{\cdot})}$ и $E_{(\boldsymbol{\cdot})}$ --- нулевые и единичные матрицы, в индексах которых указаны их размерности. В третьей строке $\widetilde{\M}_{13}$ сначала указаны элементы, соответствующие роликам, находящимся в контакте, а затем соответствующие ``свободным'' роликам; элементы упорядочены по возрастанию индексов, так что ролики одного колеса соседствуют.

Матрица связей:
$$
V = \begin{bmatrix}
        \widetilde{V}  & O_1  \\[6pt]
        O_2       & E         \\[6pt]
    \end{bmatrix};
\quad
\widetilde{V} = \begin{bmatrix}
        R\cos\theta                    & -R\sin\theta                    & 0                      \\[6pt]
        R\sin\theta                    &  R\cos\theta                    & 0                      \\[6pt]
        0                              & 0                               & \ddfrac{1}{\Lambda}    \\[6pt]
        \ddfrac{R}{l}\sin\alpha_i      & -\ddfrac{R}{l}\cos\alpha_i      & -\ddfrac{R}{\Lambda l} \\[6pt]
        \ddfrac{R}{\rho_k}\cos\alpha_k &  \ddfrac{R}{\rho_k}\sin\alpha_k & 0                      \\[6pt]
    \end{bmatrix}
$$
Здесь $O_1$ и $O_2$ -- нулевые $(3+2n \times N(n-1))$- и $(N(n-1) \times 3)$-матрицы, $E$ -- единичная матрица размерности $N(n-1)$.

Элементы матрицы кинетической энергии при учете связей:
\begin{equation}\label{mstar}
    \begin{array}{rcl}
        m^*_{11} & = & MR^2 + \sum\limits_i \bigg( J\ddfrac{R^2}{l^2}\sin^2\alpha_i + B\ddfrac{R^2}{\rho_i^2}\cos^2\alpha_i\bigg)\quad(11 \leftrightarrow 22, \sin\alpha_i \leftrightarrow \cos\alpha_i)\vsp
        m^*_{33} & = & \ddfrac{1}{\Lambda}\bigg(I_S + \sum\limits_i J\ddfrac{R^2}{l^2}\bigg),\quad
        m^*_{12}  =  \sum\limits_i \bigg(-J\ddfrac{R^2}{l^2} + B\ddfrac{R^2}{\rho_i^2}\bigg)\sin\alpha_i\cos\alpha_i\vsp
        m^*_{13} & = & \ddfrac{1}{\Lambda}\sum\limits_i B\ddfrac{R}{\rho_i}\sin\chi_i\cos\alpha_i,
        \quad
        m^*_{23} \enspace = \enspace \ddfrac{1}{\Lambda}\sum\limits_i B\ddfrac{R}{\rho_i}\sin\chi_i\sin\alpha_i
    \end{array}
\end{equation}

Обозначая
$$\xi_\pm(\alpha) = \nu_1\cos\alpha \pm \nu_2\sin\alpha, \quad \eta_\pm(\alpha) = \nu_1\sin\alpha \pm \nu_2\cos\alpha$$
для импульсов $\vec{p} = \ddpd{L}{\dot{\vec{q}}}$ получим
\begin{equation}\label{p}
    \begin{array}{c}
        p_x  =  MR\xi_-(\theta), \ p_y = MR\eta_+(\theta),\ 
        p_\theta  =  BR\sum\limits_i\ddfrac{\sin\chi_i}{\rho_i}\xi_+(\alpha_i) + \ddfrac{I_S}{\Lambda}\nu_3 + B\sum\limits_s\sin\chi_s\nu_s\vsp
        p_{\chi_i}  =  J\ddfrac{R}{l}(\eta_-(\alpha_i) - \ddfrac{1}{\Lambda}\nu_3), \ p_{\phi_{k1}}  =  \ddfrac{BR}{\rho_k}\xi_+(\alpha_k)+\ddfrac{B}{\Lambda}\sin\chi_k,\ %\vsp
        p_{\phi_s}  =  \ddfrac{B}{\Lambda}\nu_3\sin\chi_s + B\nu_s
    \end{array}
\end{equation}

Для упрощения записи правой части системы\upr{Tatarinov} введем обозначение для операции дискретной свертки произвольной функции $f$:
$$
\sigma[f(\alpha,\chi)] = \sum\limits_{k=1}^{N} f(\alpha_k,\chi_k) \frac{\sin\chi_k}{\rho_k^3}
$$
Тогда скобки Пуассона в правой части системы\upr{Tatarinov} имеют вид (звездочкой обозначена подстановка импульсов $p_i$)
%; дополнительно введено обозначение $\tau_k = \ddfrac{\sin\chi_k}{\rho_k^2}$):
\begin{eqnarray*}
(\{P_1,P_2\})^* &=& \left(-\sum\limits_{k=1}^{N} R^2\tau_kp_{\phi_k}\right)^* =
-BR^2(R\nu_1 \sigma[\cos\alpha] + R\nu_2 \sigma[\sin\alpha] + \Lambda^{-1}\nu_3\sigma[\rho\sin\chi]) = \\
&=& -BR^2\prhs_{12}\vnu\\
\prhs_{12} &=&
(\sigma[\cos\alpha], R\sigma[\sin\alpha], \Lambda^{-1}\sigma[\rho\sin\chi], 0,\dots,0)
\\
(\{P_1,P_3\})^* &=& {R}{\Lambda}^{-1}\left(-p_x\sin\theta + p_y\cos\theta - \sum\limits_{k=1}^{N} R\cos\alpha_k\tau_kp_{\phi_k}\right)^* = {MR^2}{\Lambda}^{-1}\nu_2 -\\
&-& {BR^2}{\Lambda}^{-1}(R\nu_1 \sigma[\cos^2\alpha] + R\nu_2 \sigma[\sin\alpha\cos\alpha] + \Lambda^{-1}\nu_3\sigma[\rho\cos\alpha\sin\chi])=\\
&=& {MR^2}{\Lambda}^{-1}\nu_2-{BR^2} \prhs_{13}\vnu\\
\prhs_{13} &=& \Lambda^{-1}
(R\sigma[\cos^2\alpha], R\sigma[\sin\alpha\cos\alpha], \Lambda^{-1}\sigma[\rho\cos\alpha\sin\chi], 0,\dots,0)
\\
(\{P_2,P_3\})^* &=& {R}{\Lambda}^{-1}\left(-p_x\cos\theta - p_y\sin\theta - \sum\limits_{k=1}^{N} R\sin\alpha_k\tau_kp_{\phi_k}\right)^*  = -{MR^2}{\Lambda}^{-1}\nu_1 -\\
&-& {BR^2}{\Lambda}^{-1}(R\nu_1\sigma[\sin\alpha\cos\alpha] + R\nu_2\sigma[\sin^2\alpha]+\Lambda^{-1}\nu_3\sigma[\rho\sin\alpha\sin\chi] =
\\
&=&-{MR^2}{\Lambda}^{-1}\nu_1 -{BR^2}\prhs_{23}\vnu\\
\prhs_{23} &=& \Lambda^{-1}
(R\sigma[\sin\alpha\cos\alpha], R\sigma[\sin^2\alpha], \Lambda^{-1}\sigma[\rho\sin\alpha\sin\chi], 0,\dots,0)
\end{eqnarray*}
%$$
%S_1^3 = \sum\limits_{k=1}^{N}\sin\alpha_k\frac{\cos\alpha_k}{\rho_k}\tau_k + \frac{M}{BR},\
%S_2^3 = \sum\limits_{k=1}^{N}\sin\alpha_k\frac{\sin\alpha_k}{\rho_k}\tau_k,
%$$$$
%S_3^3 = \sum\limits_{k=1}^{N}\sin\alpha_k\sin\chi_k\tau_k,
%$$
%$$
%\{P_\alpha, P_\beta\} =0,\ \alpha,\beta >3
%$$
%$$
%(\{P_1,P_2\})^* = (-\sum\limits_{k=1}^{N} R^2\tau_kp_{\phi_k})^* =
%-BR^2(R\nu_1 S_1^1 + R\nu_2 S_2^1 + \Lambda^{-1}\nu_3S_3^1)$$
%$$
%S_1^1 = \sum\limits_{k=1}^{N}\frac{\cos\alpha_k}{\rho_k}\tau_k,\
%S_2^1 = \sum\limits_{k=1}^{N}\frac{\sin\alpha_k}{\rho_k}\tau_k,\
%S_3^1 = \sum\limits_{k=1}^{N}\sin\chi_k\tau_k,
%$$
%\begin{eqnarray*}
%(\{P_1,P_3\})^* &=& \frac{R}{\Lambda}\left(-\sin\theta p_x + \cos\theta p_y - \sum\limits_{k=1}^{N} %R\cos\alpha_k\tau_kp_{\phi_k}\right)^* =\\
%&=& -\frac{BR^2}{\Lambda}(R\nu_1 S_1^2 + R\nu_2 S_2^2 + \Lambda^{-1}\nu_3S_3^2),
%\end{eqnarray*}
%$$
%S_1^2 = \sum\limits_{k=1}^{N}\cos\alpha_k\frac{\cos\alpha_k}{\rho_k}\tau_k,\
%S_2^2 = \sum\limits_{k=1}^{N}\cos\alpha_k\frac{\sin\alpha_k}{\rho_k}\tau_k - \frac{M}{BR},
%$$$$
%S_3^2 = \sum\limits_{k=1}^{N}\cos\alpha_k\sin\chi_k\tau_k,
%$$
%\begin{eqnarray*}
%(\{P_2,P_3\})^* &=& \frac{R}{\Lambda}\left(-\cos\theta p_x - \sin\theta p_y - \sum\limits_{k=1}^{N} %R\sin\alpha_k\tau_kp_{\phi_k}\right)^*  =\\
%&=& -\frac{BR^2}{\Lambda}(R\nu_1 S_1^3 + R\nu_2 S_2^3 + \Lambda^{-1}\nu_3S_3^3),
%\end{eqnarray*}
%$$
%S_1^3 = \sum\limits_{k=1}^{N}\sin\alpha_k\frac{\cos\alpha_k}{\rho_k}\tau_k + \frac{M}{BR},\
%S_2^3 = \sum\limits_{k=1}^{N}\sin\alpha_k\frac{\sin\alpha_k}{\rho_k}\tau_k,
%$$$$
%S_3^3 = \sum\limits_{k=1}^{N}\sin\alpha_k\sin\chi_k\tau_k,
%$$
%$$
%\{P_\alpha, P_\beta\} =0,\ \alpha,\beta >3
%$$
