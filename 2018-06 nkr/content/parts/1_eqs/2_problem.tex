\section{Постановка задачи}

Рассмотрим экипаж с омни-колесами, движущийся по инерции по неподвижной абсолютно шероховатой горизонтальной плоскости. Экипаж состоит из платформы и $N$ одинаковых омни-колес, плоскости которых относительно платформы неподвижны. Каждое колесо может свободно вращаться относительно платформы вокруг собственной оси, расположенной горизонтально, и на каждом из них установлено $n$ массивных роликов, так что оси роликов параллельны касательным к контурам дисков колес (см. левую часть фиг.~\ref{fig:wheel}). Ролики расположены по контуру дисков колес и показаны в виде затемненных областей, либо областей, ограниченных штриховой линией. Они пронумерованы от $1$ до $n$.
Таким образом, система состоит из $N(n+1) + 1$ абсолютно твердых тел. 

Введем неподвижную систему отсчета так, что ось $OZ$ направлена вверх, а плоскость $OXY$ совпадает с опорной плоскостью.
Введем также подвижную систему отсчета $S\xi\eta Z$, жестко связанную с платформой экипажа так, что плоскость $S\xi\eta$ 
горизонтальна и содержит центры всех колес $P_i$. Будем считать, что оси колес лежат на лучах, соединяющих центр масс $S$ платформы и центры колес (см. правую часть фиг.~\ref{fig:wheel}), а расстояния от центров колес до $S$ одинаковы и равны $R$. Геометрию установки колес на платформе зададим углами $\alpha_i$ между осью $S\xi$ и осями колес (см. левую часть фиг.~\ref{fig:wheel}). Будем считать, что центр масс всей системы совпадает с точкой $S$ (отсюда следует, что $\sum\limits_{k} \cos\alpha_k = \sum\limits_{k}\sin\alpha_k = 0$). Введем также три орта, жестко связанных с дисками колес: орт оси $i$-го колеса $\vec{n}_i = \vec{SP}_i/|\vec{SP}_i|$ и орты $\vec{n}_i^\perp$ и $\vec{n}_i^z$, лежащие в плоскости диска колеса, причем вектор $\vec{n}_i^z$ вертикален при нулевом повороте колеса $\chi_i$. Положения центров роликов на колесе определим углами $\kappa_j$ между ними и направлением, противоположным вектору $\vec{n}_i^z$. 

\begin{figure}[h]
    \minipage{0.4\textwidth}
        \centering
        \asyinclude{./content/pic/asy/pic_wheel.asy}
        \caption{Колесо}
        \label{fig:wheel}
    \endminipage
    \minipage{0.5\textwidth}
        \centering
        \asyinclude{./content/pic/asy/pic_cart.asy}
        \caption{Экипаж}
        \label{fig:vehicle}
    \endminipage
\end{figure}

Положение экипажа будем задавать следующими координатами:
$x, y$ --- координаты точки $S$ на плоскости $OXY$, $\theta$ -- угол между осями $OX$ и $S\xi$ (угол курса),
$\chi_i$ ($i = 1,\dots,N$) -- углы поворота колес вокруг их осей, отсчитываемые против часовой стрелки, если смотреть с конца вектора $\vec{n}_i$, и $\phi_j$ -- углы поворота роликов вокруг их собственных осей.
Таким образом, вектор обобщенных координат имеет вид
$$\vec{q} = (
    x, y, \theta,
    \left.\{\chi_i\}\right|_{i=1}^N ,
    \left.\{\phi_k\}\right|_{k=1}^N,
    \left.\{\phi_s\}\right|_{s=1}^{N(n - 1)}
)^{\mathop{T}}\in\mathbb{R}^{N(n+1) + 3}$$ 
где сначала указаны углы поворота $\phi_k$ роликов, находящихся в данный момент в контакте с опорной плоскостью, a затем -- остальных, ``cвободных'', роликов. Индекс $s$ используется для сквозной нумерации свободных роликов и связан с номером колеса $i$ и ролика на колесе $j$ по формуле
\begin{equation}\label{eq:num}
    s(i, j) = (n-1)(i-1) + j - 1
\end{equation}

Введем псевдоскорости
$$\vnu = (\nu_1, \nu_2, \nu_3, \nu_s), \quad \vec{v}_S = R\nu_1\vec{e}_\xi + R\nu_2\vec{e}_\eta, \quad \nu_3 = \Lambda\dot{\theta},\quad \nu_s = \dot{\phi}_s, \quad s = 1,\ldots,N(n-1)$$

Их механический смысл таков: $\nu_1$, $\nu_2$ --- проекции скорости точки $S$ на оси системы $S\xi\eta$, связанные с платформой, $\nu_3$ --- с точностью до множителя угловая скорость платформы, $\nu_s$ --- угловые скорости свободных роликов. Число независимых псевдоскоростей системы $K = N(n-1)+3$. Таким образом, имеем
$$ \dot{x} = R \nu_1\cos\theta-R\nu_2\sin\theta, \hspace{15pt} \dot{y} = R\nu_1\sin\theta+R\nu_2\cos\theta$$

Будем считать, что проскальзывания между опорной плоскостью и роликами в контакте не происходит, т.е.
скорости точек $C_i$ контакта равны нулю:
\begin{equation}\label{eq:constraints_vec}
    \vec{v}_{C_i} = 0,\quad i = 1,\dots, N    
\end{equation}

Выражая скорость точек контакта через введенные псевдоскорости и проектируя на оси системы $S\xi\eta$, получим:
\begin{eqnarray}
\dot{\phi_k} &=& \frac{R}{\rho_k }(\nu_1\cos\alpha_k + \nu_2\sin\alpha_k); \quad \rho_k  = l\cos\chi_k - r \label{constraint_roller_contact}\\
\dot{\chi}_i &=& \frac{R}{l}(\nu_1\sin\alpha_i - \nu_2\cos\alpha_i - \frac{\nu_3}{\Lambda})\label{constraint_wheel_contact}
\end{eqnarray}
Заметим, что знаменатель $\rho_k$ в формуле (\ref{constraint_roller_contact}) -- расстояние от оси ролика до точки контакта, обращающееся в нуль на стыке роликов (см. левую часть фиг.~\ref{fig:wheel}). Это обстоятельство приводит к разрывам второго рода функций в правых частях уравнений движения и будет рассмотрено отдельно ниже.
Уравнение (\ref{constraint_wheel_contact}) совпадает с уравнением связи в случае безынерционной модели роликов. 

Таким образом, выражение обобщенных скоростей через псевдоскорости, учитывающее связи, наложенные на систему, можно записать в матричном виде:
\begin{equation}\label{constraints_V}
    \dot{\vec{q}} = \cstr\vnu,\quad \cstr = \cstr(\theta,\chi_i),
\end{equation}

где компоненты матрицы $\cstr$ имеют вид:
$$
\cstr = \begin{bmatrix}
        \widetilde{V}  & O_1  \\[6pt]
        O_2       & E         \\[6pt]
    \end{bmatrix};
\quad
\widetilde{V} = \begin{bmatrix}
        R\cos\theta                    & -R\sin\theta                    & 0                      \\[6pt]
        R\sin\theta                    &  R\cos\theta                    & 0                      \\[6pt]
        0                              & 0                               & \ddfrac{1}{\Lambda}    \\[6pt]
        \ddfrac{R}{l}\sin\alpha_i      & -\ddfrac{R}{l}\cos\alpha_i      & -\ddfrac{R}{\Lambda l} \\[6pt]
        \ddfrac{R}{\rho_k}\cos\alpha_k &  \ddfrac{R}{\rho_k}\sin\alpha_k & 0                      \\[6pt]
    \end{bmatrix}
$$
Здесь $O_1$ и $O_2$ -- нулевые $(3+2n \times N(n-1))$- и $(N(n-1) \times 3)$-матрицы, $E$ -- единичная матрица размерности $N(n-1)$.

